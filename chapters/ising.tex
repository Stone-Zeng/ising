\chapter{Ising 模型}

\section{Ising 模型的统计物理}
\label{sec:ising-physics}

Ising 模型是统计物理中描述铁磁系统的一种简单模型。Ising 模型由一系列自旋排列而成的点阵构成,其中的每
个自旋只有两种取向,即 $+1$ 或 $-1$。自旋与自旋之间的相互作用可以导致相变的出现,最典型的例子是二维
正方晶格的 Ising 模型,其严格解由 Onsager 给出
\cite{pathria,linzonghan,surukeng,onsager1944crystal}。

Ising 模型的 Hamilton 量为 \cite{pathria,linzonghan}:
\begin{equation}
  H(\q{\sigma_i}) = -\sum_{\nearest{ij}} J_{ij}\sigma_i\sigma_j - \mu \sum_{i} B_i\sigma_i.
\end{equation}
式中,“$\nearest{ij}$” 表示相邻的自旋。$J_{ij}$ 是耦合常数,通常情况下与自旋位置无关,即
$J_{ij}=J$,因而可以放在求和号外部。$J>0$ 时,代表体系处于铁磁态;$J<0$ 时,则代表处于反铁磁态。
$B_i$ 是外磁场,对于匀强磁场,有 $B_i=B$,即同样与位置无关。$\mu$ 是对应的磁矩。由于体系关于
$\sigma_i$ 的正负是对称的,方便起见不妨认为 $B \geqslant 0$。此时,我们有
\begin{equation}
  H(\q{\sigma_i}) = -J \sum_{\nearest{ij}} \sigma_i\sigma_j - \mu B \sum_{i} \sigma_i.
\end{equation}

我们可以为某一特定的自旋构型引入概率。温度 $T$ 时,构型 $\q{\sigma_i}$ 在平衡态中出现的概率由
Boltzmann 分布给出 \cite{exact}:
\begin{equation}
  \label{eq:ising-probability}
  P(\q{\sigma_i}) = \frac{\ee^{-\beta H(\q{\sigma_i})}}{Z_N}.
\end{equation}
其中的 $\beta=1/T$。而 $Z_N$ 称为配分函数,它使得概率 $P$ 满足归一化条件
\begin{equation}
  \sum_\q{\sigma_i} P(\q{\sigma_i}) = 1
  \implies Z_N = \sum_\q{\sigma_i} \ee^{-\beta H(\q{\sigma_i})}.
\end{equation}
配分函数等价于体系的自由能。在温度 $T$、外磁场 $B$ 作为变量的情况下,自由能是体系的特性函数,故可以
通过自由能计算出各物理量,包括内能、热容、磁化强度等。因此求解 Ising 模型实际上就是求出体系的配分函
数(自由能)。

一维 Ising 模型的严格解可利用转移矩阵的手段求出。热力学极限下(自旋数目 $N\to\infty$),自由能为
\begin{equation}
  F = - \frac{\ln Z_N}{\beta}
    = - N J - \frac{N}{\beta}
              \ln\qty[\cosh(\beta\mu B) + \sqrt{\ee^{-4\beta J} + \sinh^2(\beta\mu B)}].
\end{equation}
磁化强度为
\begin{equation}
  \frac{\bar{M}}{N\mu} = \frac{\sinh{\beta\mu B}}{\sqrt{\ee^{-4\beta J} + \sinh^2(\beta\mu B)}}.
\end{equation}
显然,只要 $\beta<\infty$,当 $B\to 0$ 时,$\bar{M}\to 0$;而当 $\beta\to 0$ 时,$\bar{M}\to N\mu$。
这说明一维 Ising 模型的临界温度 $\Tc=0$,事实上不存在有限温度下的相变 \cite{linzonghan,pathria}。

二维 Ising 模型的严格解比较复杂。对于正方晶格,可利用对偶变换求出临界温度
\cite{pathria}
\begin{equation}
  \label{eq:ising-Tc}
  \frac{\Tc}{J} = \frac{2}{\ln(1+\sqrt{2})} \approx \num{2.2692}.
\end{equation}
外场为零时,可以求得(单位自旋的)自由能为
\cite{pathria,onsager1944crystal}
\begin{equation}
  \frac{\beta F}{N}
  = - \frac{\ln Z_N}{N}
  = - \frac{\ln 2}{2} - \ln\cosh 2\beta J
    - \frac{1}{2\pp} \int_0^\pp \ln\qty[1+\sqrt{1-\kappa^2\sin^2\theta}] \dd{\theta},
\end{equation}
其中
\begin{equation}
  \kappa = \frac{2\sinh 2\beta J}{\cosh^2 2\beta J}.
\end{equation}
进而可以求得热容 \cite{pathria}:
\begin{align}
  \label{eq:ising-specific-heat}
  \frac{C}{N}
  & \simeq -\frac{2}{\pp} \ln^2 \qty(1+\sqrt{2}) \ln\abs{1-\frac{T}{\Tc}} + \const \notag \\
  & \simeq \num{-0.4945} \ln\abs{1-\frac{T}{\Tc}} + \const.
\end{align}
可以看到热容在临界温度处呈现出对数发散行为。

磁化强度的计算结果由杨振宁首次给出 \cite{pathria,yang1952spontaneous}:
\begin{equation}
  \label{eq:ising-magnetization}
  \frac{\bar{M}}{N\mu} =
  \begin{cases}
    \qty(1 - \sinh^{-4} 2\beta J)^{1/8}, & T<\Tc; \\
    0, & T>\Tc.
  \end{cases}
\end{equation}

\section{临界行为与重整化群}

\subsection{临界指数}

一般而言,在接近临界点处,热力学量 $f$ 可按约化温度 $t=T/\Tc-1$ 展开:
\begin{equation}
  f(t) = A t^\lambda \qty\big(1 + B t^y + \cdots),
\end{equation}
式中 $y>0$。定义临界指数 $\lambda$ 为
\begin{equation}
  \lambda = \lim_{t \to 0} \frac{\ln f(t)}{\ln t}.
\end{equation}
它刻画了热力学量 $f$ 的临界行为:
\begin{equation}
  \lim_{t \to 0} f(t) =
  \begin{cases}
    0,      & t > 0; \\
    \const, & t = 0; \\
    \infty, & t < 0.
  \end{cases}
\end{equation}
注意当 $f(t) \sim \ln t$ 时,同样有
\begin{equation}
  \lambda =    \lim_{t \to 0} \frac{\ln f(t)}{\ln t}
          \sim \lim_{t \to 0} \frac{\ln \ln t}{\ln t} = 0.
\end{equation}

取 $f$ 为热容 $C$,则定义了临界指数 $\alpha$。因此对于 Ising 模型,根据
式~\eqref{eq:ising-specific-heat},可有
\begin{equation}
  \alpha = 0.
\end{equation}
取 $f$ 为序参量,即磁化强度 $\bar{M}$,则定义了临界指数 $\beta$。根据
式~\eqref{eq:ising-magnetization},有
\begin{equation}
  \beta = \frac{1}{8}.
\end{equation}

临界指数的存在具有普适性。对于不同的系统,如铁磁系统、van der Waals 气液相变、超导体系等,均可以
得到完全相同的临界指数。这种普适性的根源是关联长度 $\xi$ 在临界点处的发散性为。在临界点处,
$\xi\to\infty$,其他的特征长度则保持不变,因此热力学量的奇异性仅依赖于 $\xi$ 的奇异性,而与体系的
具体特征无关 \cite{linzonghan}。

%以铁磁系统为例,设单位自旋的自由能 $f$ 是约化温度、约化磁场
%\begin{equation}
%  t = \frac{T}{\Tc} - 1 \qc
%  b = \frac{\mu B}{\Tc}
%\end{equation}
%的函数。将其分解为奇异部分 $f_{\symup{s}}$ 和非奇异部分 $f_{\symup{r}}$:
%\begin{equation}
%  f(t,\,b) = f_{\symup{s}}(t,\,b) + f_{\symup{r}}(t,\,b).
%\end{equation}
%在临界点附近,只有奇异部分 $f_{\symup{s}}$ 需要考虑。
%
%标度变换可以视为对晶格的某种“缩放”。经过一次标度变换后,设体系的晶格常数由 $a$ 变成了 $a'=La$。
%相应地,有
%\begin{equation}
%  \label{eq:transform-relation-t-b}
%  t' = L^x t \qc b' = L^y b.
%\end{equation}
%体系总的自由能(配分函数)在标度变换下保持不变,但自旋数目发生了变化:$N \to N'=L^{-d} N$,$d$
%为体系的维度。因而有
%\begin{equation}
%  N f_{\symup{s}} (t,\,b) = N' f_{\symup{s}} (t',\,b'),
%\end{equation}
%即
%\begin{equation}
%  f_{\symup{s}} (t,\,b) = L^{-d} f_{\symup{s}} (L^x t, \, L^y b).
%\end{equation}
%取 $L^x t=1$,则有
%\begin{equation}
%  f_{\symup{s}} (t,\,b) = t^{d/x} f_{\symup{s}} (1, \, L^y b).
%\end{equation}
%考虑到热容 $C$ 正比于自由能对温度的二阶导数,可得
%\begin{equation}
%  C \sim \pdv[2]{f_{\symup{s}}}{t} \sim t^{d/x-2},
%\end{equation}
%即临界指数
%\begin{equation}
%  \alpha = 2 - \frac{d}{x}.
%\end{equation}
%类似地,可以计算出其他临界指数满足的条件:
%\begin{equation}
%  \beta  = \frac{d-y}{x}  \qc
%  \gamma = \frac{2y-d}{x} \qc
%  \delta = \frac{y}{d-y}.
%\end{equation}
%消去 $x$ 、$y$,即得到标度律
%\begin{equation}
%  \label{eq:scaling-relation-i-ii}
%  \alpha + 2\beta + \gamma = 2 \qc
%  \gamma = \beta (\delta - 1).
%\end{equation}
%
%另一方面,关联长度满足变换关系 $\xi\to\xi'=L^{-1}\xi$。代入式~\eqref{eq:transform-relation-t-b},得
%\begin{equation}
%  \frac{\xi'}{\xi} = \qty(\frac{t'}{t})^{-\nu} = L^{x\nu} = L^{-1}
%  \implies \nu = \frac{1}{x}.
%\end{equation}
%因此得到另一条标度律
%\begin{equation}
%  \label{eq:scaling-relation-iii}
%  \nu d = 2 - \alpha.
%\end{equation}
%
%如果定义关联函数
%\begin{equation}
%  g(\bm{r}_{ij}) = \overline{(\sigma_i-\bar{\sigma_i})(\sigma_j-\bar{\sigma_j})},
%\end{equation}
%则有另一个临界指数 $\eta$ 与之关联:
%\begin{equation}
%  g(\bm{r}) \sim r^{2-d-\eta}.
%\end{equation}
%并同样有标度律
%\begin{equation}
%  \label{eq:scaling-relation-iv}
%  \eta = 2 + d - 2y \implies \gamma = (2-\eta) \nu.
%\end{equation}
%
%这样,利用四条标度律 \eqref{eq:scaling-relation-i-ii}、\eqref{eq:scaling-relation-iii} 和
%\eqref{eq:scaling-relation-iv}~式,

利用自由能(配分函数)在标度变换下的形式不变性,可以得到临界指数所满足的标度律
\cite{pathria,linzonghan}
\begin{equation}
  \alpha + 2\beta + \gamma = 2 \qc
  \gamma = \beta (\delta - 1)  \qc
  \nu d = 2 - \alpha           \qc
  \gamma = (2-\eta) \nu.
\end{equation}
利用它们可以计算得到 Ising 模型的其他几个临界指数 \cite{pathria,linzonghan}:
\begin{equation}
  \gamma = \frac{7}{4} \qc
  \delta = 15          \qc
  \nu    = 1           \qc
  \eta   = \frac{1}{4}.
\end{equation}

\subsection{重整化群}

前面已经指出,在相变临界点处关联长度 $\xi$ 会出现发散。因此,$\xi$ 在任意尺度变换下均是不变的,
即对应变换的不动点。根据这一观点,我们可以直接计算出临界指数而无需严格求解配分函数,这称为
重整化群方法。

一般而言,体系的配分函数总可以写成
\begin{equation}
  Z_N = \sum_\q{\sigma_i} \ee^{-\beta H(\q{\sigma_i})},
\end{equation}
其中,体系的 Hamilton 量为
\begin{equation}
    H(\q{\sigma_i})
  = H(\q{\sigma_i},\,\bm{K})
  = - \sum_i K^{(1)}_i \sigma_i
    - \sum_{\nearest{ij}} K^{(2)}_{ij} \sigma_i\sigma_j
    - \sum_{\nearest{ijk}} K^{(3)}_{ijk} B_i\sigma_i\sigma_j\sigma_k - \cdots
\end{equation}
耦合参数 $\bm{K}$ 可以表征体系中的多种相互作用 \cite{pathria,surukeng}。

接下来对体系进行一次标度变换,即粗粒近似。此时,多个自旋被统一处理,视作一个 Kadanoff 集团。
自旋数目(即体系自由度)$N$ 和关联长度 $\xi$ 分别变为
\begin{equation}
  \label{eq:N-xi-renormalization}
  N' = L^{-d} N \qc \xi' = L^{-1} \xi.
\end{equation}
其中 $L>1$ 为缩放系数,$d$ 则为体系的维度。经过一次粗粒近似后,体系的配分函数数值上应当保持不变,
但形式上需要变为对 Kadanoff 集团 $\q{\sigma_i'}$ 的求和 \cite{pathria,exact}:
\begin{equation}
  \label{eq:Z-H-renormalization}
  Z_{N'}
  = \sum_\q{\sigma'_i} \ee^{-\beta H^{\text{RG}}(\q{\sigma_i'},\,\bm{K}')}
  = \sum_\q{\sigma'_i} \sum_\q{\sigma_i}
    \ee^{-\beta \qty\big[H(\q{\sigma_i},\,\bm{K})-\opT_\lambda(\q{\sigma_i},\,\q{\sigma_i'})]}.
\end{equation}
式中的 $\opT_\lambda(\q{\sigma_i},\,\q{\sigma_i'})$ 刻画了从单个自旋 $\q{\sigma_i}$ 到自旋集团
$\q{\sigma_i'}$ 的变换,而 $\bm{K}'$ 为变换后的耦合参数,它满足
\begin{equation}
  \label{eq:K-renormalization}
  \bm{K}' = \trR{L}(\bm{K}).
\end{equation}
耦合参数 $\bm{K}$ 作为向量可以张成一个向量空间 $\spaceV$,因而我们可以把 $\trR{L}$ 理解为 $\spaceV$
上的一个算子(未必要求线性)。在不动点 $\bm{K}^*$ 附近做展开,可以得到 $\trR{L}$ 的线性近似
$\trA[*]{L}$。计算矩阵 $\trA[*]{L}$ 的本征值,就可以直接给出临界指数的具体数值,而无需再对
配分函数进行严格求解 \cite{pathria,surukeng}。

%粗粒近似可以重复进行,即
%\begin{equation}
%  \bm{K} \to \bm{K}' = \trR{L}(\bm{K}) \to \cdots \to \bm{K}^{(n)} = \trR[n]{L}(\bm{K}).
%\end{equation}
%可见 $\trR[k]{L}$ 连同恒等变换构成一个半群(不存在逆变换)。根据重整化群理论的精神,重整化变换
%$\trR{L}$ 存在不动点 $\bm{K}^*$,满足
%\begin{equation}
%  \trR{L}(\bm{K}^*) = \bm{K}^*.
%\end{equation}
%在不动点附近分别展开 $\bm{K}$ 和 $\bm{K}'$,得到
%\begin{equation}
%  \bm{K}  = \bm{K}^* + \delta{\bm{K}} \qc
%  \bm{K}' = \bm{K}^* + \delta{\bm{K}'}.
%\end{equation}
%代入 \eqref{eq:K-renormalization}~式,有
%\begin{align}
%  &\mathrel{\phantom{\implies}}
%    \bm{K}^* + \delta{\bm{K}'}
%  = \trR{L}(\bm{K}^* + \delta{\bm{K}})
%  = \trR{L}(\bm{K}^*) + \eval{\pdv{\trR{L}}{\bm{K}}}_{\bm{K}^*} \delta{\bm{K}} \notag \\
%  &\implies
%    \delta{\bm{K}'}
%  = \eval{\pdv{\trR{L}}{\bm{K}}}_{\bm{K}^*} \delta{\bm{K}}
%  = \trA[*]{L} \delta{\bm{K}}.
%\end{align}
%式中,$\trA[*]{L}$ 是 $\trR{L}$ 在 $\bm{K}^*$ 处的线性近似。设 $\trA[*]{L}$ 的本征值与
%本征向量分别为 $\lambda_i$ 和 $\bm{\phi}_i$,则可将 $\delta{\bm{K}}$ 和 $\delta{\bm{K}'}$ 展开:
%\begin{equation}
%  \delta{\bm{K}}  = \sum_i u_i  \bm{\phi}_i \qc
%  \delta{\bm{K}'} = \sum_i u'_i \bm{\phi}_i.
%\end{equation}
%两边作用 $\trA[*]{L}$,可得到系数满足的关系
%\begin{equation}
%  u_i' = \lambda_i u_i.
%\end{equation}
%同理,作用 $n$ 次 $\trA[*]{L}$,即 $\qty\big(\trA[*]{L})^n$,可有
%\begin{equation}
%  u_i^{(n)} = \lambda_i^n u_i^{(0)}.
%\end{equation}
%显然,若本征值 $\lambda_i>1$,则随着 $n$ 的增加,$u_i^{(n)}$ 的权重越来越大,因此决定了体系的
%临界行为,对应的参数 $u_i$ 称为相关变量;反之,若本征值 $\lambda_i<1$,则 $u_i^{(n)}$ 的权重随着
%$n$ 的增加越来越小,故对应的 $u_i$ 称为无关变量。$\lambda_i=1$ 对应的 $u_i^{(n)}$ 会保持定常行为,
%能够带来介于以上二者之间的影响,如幂律行为中的对数项。
%
%对于 Ising 模型,可以给出两个相关变量,它们分别与约化温度和约化磁场关联:
%\begin{equation}
%  u_1 \propto t \qc u_2 \propto b.
%\end{equation}
%根据式~\eqref{eq:N-xi-renormalization},我们有
%\begin{equation}
%  \xi' = L^{-1} \xi \implies \xi^{(n)} = L^{-n} \xi^{(0)}.
%\end{equation}
%关联长度 $\xi$ 是 $u_i$ 的函数,因此
%\begin{equation}
%  \xi^{(n)} = \xi\qty\big(\q{u_i^{(n)}}) = \xi\qty\big(\q{\lambda_i^n u_i^{(0)}}) \qc
%  \xi^{(0)} = \xi\qty\big(\q{u_i^{(0)}}).
%\end{equation}
%根据临界指数 $\nu$ 的定义,有
%\begin{equation}
%  \xi \propto t^{-\nu} \propto u_1^{-\nu}.
%\end{equation}
%联立以上三式,我们有
%\begin{equation}
%  \xi^{(n)} \propto \lambda_1^{-n\nu} u_1^{-\nu} \qc
%  \xi^{(0)} \propto u_1^{-\nu}
%  \implies \frac{\xi^{(n)}}{\xi^{(0)}} = \lambda_1^{-n\nu} = L^{-n}
%  \implies \nu = \frac{\ln L}{\ln \lambda_1}.
%\end{equation}
%这样,我们就利用重整化群方法计算得到了临界指数 $\nu$。一旦给定了具体的重整化变换 $\trR{L}$,
%就立刻可以给出 $\nu$ 的数值,而无需对配分函数进行严格求解。
%
%\section{Ising 模型与 \AdSCFT{}}

\section{Ising 模型的 Monte Carlo 模拟}

\subsection{Metropolis 算法}

在 \eqref{eq:ising-probability}~式中,我们给出了构型 $\q{\sigma_i}$ 在平衡态中出现的概率。严格
计算这一概率,需要对所有可能的构型进行求和。设自旋数目为 $N$,则构型数正比于 $N!$,可见直接对它们
进行求和是不现实的。为了回避过于巨大的计算开销,Monte Carlo 算法给出了另外一种方案,即以按照一定
概率的抽样,代替对全部构型的平均。

考虑自旋构型 $\q{\sigma_i}$ 随时间的演化。设在 $t$ 时刻,构型 $\q{\sigma_i}$ 对应的概率为
$P\qty\big(\q{\sigma_i},\,t)$,则 \cite{pathria,landau2005guide}
\begin{align}
  P\qty\big(\q{\sigma_i},\,t+1) = P\qty\big(\q{\sigma_i},\,t)
  &+ \sum_{\q{\sigma'_i}} P\qty\big(\q{\sigma'_i},\,t) W\qty\big(\q{\sigma'_i}\to\q{\sigma_i})
     \notag \\
  &- \sum_{\q{\sigma'_i}} P\qty\big(\q{\sigma_i},\,t)  W\qty\big(\q{\sigma_i}\to\q{\sigma'_i}).
\end{align}
式中,$W\qty\big(\q{\sigma_i}\to\q{\sigma'_i})$ 表示从 $\q{\sigma_i}$ 到 $\q{\sigma'_i}$ 的
转移概率,需要满足细致平衡条件 \cite{pathria,landau2005guide}:
\begin{equation}
  \label{eq:detailed-balance}
    P_{\text{eq}}(\q{\sigma_i})  W\qty\big(\q{\sigma_i}\to\q{\sigma'_i})
  = P_{\text{eq}}(\q{\sigma'_i}) W\qty\big(\q{\sigma'_i}\to\q{\sigma_i}).
\end{equation}
平衡态下,构型 $\q{\sigma_i}$ 对应的概率可由式~\eqref{eq:ising-probability} 获得,它正比于
$\ee^{-\beta H(\q{\sigma_i})}$,因而有
\begin{equation}
    \frac{W\qty\big(\q{\sigma_i}\to\q{\sigma'_i})}{W\qty\big(\q{\sigma'_i}\to\q{\sigma_i})}
  = \frac{\ee^{-\beta H(\q{\sigma'_i})}}{\ee^{-\beta H(\q{\sigma_i})}}
  = \ee^{-\beta \incr{E}}.
\end{equation}
其中的 $\incr{E}$ 表示两构型间的能量差。考虑到概率的有效取值范围为 $[0,\,1]$,我们有
\begin{equation}
  \label{eq:metropolis-probability}
  W\qty\big(\q{\sigma_i}\to\q{\sigma'_i}) =
  \begin{cases}
    1, & \incr{E} \leqslant 0; \\
    \ee^{-\beta \incr{E}}, & \incr{E} > 0. \\
  \end{cases}
\end{equation}

Metropolis 算法是 Monte Carlo 模拟的一种具体实现,它给出了选取下一时刻自旋构型的方案。
具体步骤可以表述如下 \cite{pathria,landau2005guide,binder2010monte}:

\begin{enumerate}
  \item 初始化所有自旋,可以随机取为 0 或 1。但如果初始时就令自旋(近似)满足一定的概率分布,
    可以使得算法更为高效。
  \item 在自旋构型 $\q{\sigma_i}$ 的基础上,随机生成一个与之相差不大的构型 $\q{\sigma'_i}$。
    通常做法是翻转某一个自旋。
  \item 计算能量差 $\incr{E}$,按照式~\eqref{eq:metropolis-probability} 获得接受概率 $P$。生成一个
    区间 $(0,\,1)$ 内的随机数 $\xi$,若 $\xi<P$,则接受构型 $\q{\sigma'_i}$(即翻转相应自旋);
    否则不进行操作。
  \item 遍历整个点阵,这成为一个 Monte Carlo 步(Monte Carlo sweep, MCS)。为满足细致平衡条件
    \eqref{eq:detailed-balance}~式,原则上需在点阵中随机选取自旋,并计算接受概率/决定是否翻转;
    但出于效率的考虑,也常常直接按次序遍历整个点阵。
  \item 重复进行 2--4 步,测量各物理量的,并计算它们的统计平均值及标准误差。
\end{enumerate}

\subsection{模拟结果}

在后文中,机器学习所用到的 Ising 模型训练数据集均利用上面介绍的 Metropolis 算法生成。为了文章的
完整性,这里我们也顺带给出 Ising 模型的 Monte Carlo 模拟结果。

\begin{figure}[htb]
  \begin{subfigure}[b]{0.47\textwidth}
    \hfill
    \imageinput{ising-energy}
    \phantomcaption{}
  \end{subfigure}
  \begin{subfigure}[b]{0.47\textwidth}
    \hfill
    \imageinput{ising-magnet}
    \phantomcaption{}
  \end{subfigure}
  \\[3ex]
  \begin{subfigure}[b]{0.47\textwidth}
    \hfill
    \imageinput{ising-cv}
    \phantomcaption{}
  \end{subfigure}
  \begin{subfigure}[b]{0.47\textwidth}
    \hfill
    \imageinput{ising-cv-exact}
    \phantomcaption{}
    \label{fig:ising-cv-exact}
  \end{subfigure}
  \caption{Ising 模型各物理量随温度的变化曲线。为了更清晰地反映出相变的临界情况,我们在临界点附近
    增大了采样密度。图中还用灰线标出了临界温度 $\Tc$ 的位置。图中的误差棒代表正负一个标准差。
    (a)~能量;(b)~自发磁化;(c)~热容,注意到在较小的点阵中存在巨大的涨落 \protect\footnotemark{};
    (d)~热容的精确计算结果,额外给出了 $4 \times 4$、$8 \times 8$ 和 $256 \times 256$ 点阵中的情形}
  \label{fig:ising-observables}
\end{figure}

\footnotetext{不确定度(标准差)在 $T/J=\num{1.8}$ 和 \num{2.8} 处的突变来自于采样的不均匀。为了提高临界点附近的精度,该区间之内温度采样间隔为 \num{0.02},区间之外则为 \num{0.1}。}

以下结果中,Ising 点阵大小分别为 $16 \times 16$、$32 \times 32$、$64 \times 64$ 和
$128 \times 128$,并采用周期性边界条件。选取温度区间 $T/J \in [1,\,3.6]$,外磁场为零。进行
\num{900} MCS 之后,认为体系达到平衡态。之后再进行 \num{100} MCS,并在其中取系综平均计算各物理量。
每个温度下,重复模拟 50 次,统计各物理量的平均值及标准差(作为不确定度)。模拟结果见
图~\ref{fig:ising-observables}。

根据模拟结果,我们发现随着体系尺度的增加,临界行为逐渐接近 \ref{sec:ising-physics}~节中给出的
严格计算结果,如自发磁矩的幂律发散、热容的对数发散等。

为了与模拟结果进行对照,我们还给出了有限体系热容的精确结果,见图~\ref{fig:ising-cv-exact}。
可以看到,随着尺度的增加,热容的极大值逐渐增大,极值点出现的位置也向 $\Tc$ 靠近。事实上,将极值点
位置 $T_*$ 和热容极大值 $C_*$ 关于体系尺度 $L$ 分别进行拟合,可以得到如下关系 \cite{pathria}:
\begin{equation}
  T_* - \Tc \propto L^{-1} \qc
  C_*       \propto \ln L.
\end{equation}
结果见图~\ref{fig:ising-fit}。

\begin{figure}[htb]
  \begin{subfigure}[b]{0.48\textwidth}
    \hfill
    \imageinput{ising-cv-fit-i}
    \phantomcaption{}
    \label{fig:ising-cv-fit-i}
  \end{subfigure}
  \begin{subfigure}[b]{0.48\textwidth}
    \hfill
    \imageinput{ising-cv-fit-ii}
    \phantomcaption{}
    \label{fig:ising-cv-fit-ii}
  \end{subfigure}
  \caption{利用图~\ref{fig:ising-cv-exact} 中热容的精确计算结果关于体系尺度拟合。
    (a)~临界温度随点阵尺度的变化,$x$、$y$ 轴均取对数坐标;
    (b)~热容极大值随点阵尺度的变化,只有 $x$ 轴取对数坐标}
  \label{fig:ising-fit}
\end{figure}
