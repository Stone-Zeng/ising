\chapter{机器学习与重整化群}

\section{限制 Boltzmann 机(RBM)}

限制 Boltzmann 机(restricted Boltzmann machine, RBM)是一类生成型随机神经网络(generative
stochastic neural network),它可以学习到输入数据的概率分布情况 \cite{hinton2006reducing}。
顾名思义,RBM 是 Boltzmann 机加上一定限制所得到的网络。一般的 Boltzmann 机是一种基于能量的模型
(energy-based model)\cite{zhouzhihua,rbmonline}。对于这种模型,我们可以赋予它一个能量函数。
与物理学中类似,体系的稳定状态将在能量最低时达到,这一稳定状态即对应最优参数。

Boltzmann 机的结构如图~\ref{fig:boltzmann-machine} 所示。它共有两层,与数据直接相连的称为可见层
(visible layer),另一层称为隐藏层(hidden layer)。可见层用于处理输入输出,隐藏层则反应了数据的
内在结构。无论是可见层还是隐藏层,所有单元均全部连接在一起。Boltzmann 机的能量函数由下式给出
\cite{zhouzhihua}:
\begin{equation}
  E(\bm{s}) = E(\q{s_i}) = - \sum_{i<j} W_{ij} s_i s_j - \sum_i \theta_i s_i.
\end{equation}
式中,$s_i$ 取为布尔型,即 $s_i\in\q{0,\,1}$。$W_{ij}$ 是单元 $s_i$ 与 $s_j$ 之间的连接权重,
$\theta_i$ 则表示单元 $s_i$ 处的偏差。

\begin{figure}[htb]
  \centering
  \begin{subfigure}[b]{0.45\textwidth}
    \centering
    \imageinput{boltzmann-machine.pdf}
    \caption{}
    \label{fig:boltzmann-machine}
  \end{subfigure}
  \begin{subfigure}[b]{0.45\textwidth}
    \centering
    \imageinput{rbm.pdf}
    \vspace{0.4cm}
    \caption{}
    \label{fig:rbm}
  \end{subfigure}
  \caption{Boltzmann 机和限制 Boltzmann 机的结构示意图。其中红色圆圈表示可见层,蓝色方块表示隐藏层。
    (a)~Boltzmann 机,所有单元之间均有权重矩阵连接;
    (b)~限制 Boltzmann 机,只有可见层与隐藏层之间才连接有权重矩阵}
\end{figure}

与 Ising 模型类似[式~\eqref{eq:ising-probability}],状态 $\bm{s}=\q{s_i}$ 出现的概率由 Boltzmann
分布给出 \cite{rbmonline}:
\begin{equation}
  \label{eq:boltzmann-machine-probability}
  P(\bm{s}) = \frac{\ee^{-E(\bm{s})}}{Z},
\end{equation}
其中配分函数 $Z$ 的定义为
\begin{equation}
  Z = \sum_{\bm{s}} \ee^{-E(\bm{s})}.
\end{equation}
Boltzmann 机的训练过程就是最大化训练样本所对应的概率。

由于 Boltzmann 机是全连接的,它的训练效率并不高。我们可以引入限制条件,仅保留可见层与隐藏层之间的
连接。由此便得到了限制 Boltzmann 机(如图~\ref{fig:rbm}),它的能量函数为
\cite{exact,rbmonline,chen2018equivalence}
\begin{equation}
  \label{eq:rbm-energy}
  E(\bm{v},\,\bm{h})
  = - \bm{v}^\trans \bm{W} \bm{h} - \bm{b}^\trans \bm{v} - \bm{c}^\trans \bm{h}
  = - \sum_{i,\,j} W_{ij} v_i h_j - \sum_i b_i v_i - \sum_j c_j h_j.
\end{equation}
式中,$v_i$ 和 $h_j$ 分别代表可见层与隐藏层中的单元,$W_{ij}$ 为两层之间的连接权重,$b_i$ 和 $c_j$
分别是可见层与隐藏层中单元对应的偏差。

RBM 中,可见层 $\bm{v}$ 与隐藏层 $\bm{h}$ 是彼此分离的,可以视为两个随机变量,因而
式~\eqref{eq:boltzmann-machine-probability} 即成为 $\bm{v}$ 与 $\bm{h}$ 的联合概率分布:
\begin{equation}
  P(\bm{v},\,\bm{h})
  = \frac{\ee^{-E(\bm{v},\,\bm{h})}}{Z}
  = \frac{\ee^{-E(\bm{v},\,\bm{h})}}{\sum_{\bm{v},\,\bm{h}}\ee^{-E(\bm{v},\,\bm{h})}}.
\end{equation}
对所有的 $\bm{h}$ 进行求和,我们就得到了某一可见层 $\bm{v}$ 对应的边缘概率分布:
\begin{equation}
  \label{eq:rbm-probability-v}
  P(\bm{v}) = \sum_{\bm{h}} P(\bm{v},\,\bm{h})
            = \frac{1}{Z} \sum_{\bm{h}} \ee^{-E(\bm{v},\,\bm{h})}
            \equiv \frac{1}{Z} \ee^{-H^{\text{RBM}}(\bm{v})};
\end{equation}
同理,对所有的 $\bm{v}$ 进行求和,可得到隐藏层 $\bm{h}$ 对应的边缘概率分布:
\begin{equation}
  \label{eq:rbm-probability-h}
  P(\bm{h}) = \sum_{\bm{h}} P(\bm{v},\,\bm{h})
            = \frac{1}{Z} \sum_{\bm{v}} \ee^{-E(\bm{v},\,\bm{h})}
            \equiv \frac{1}{Z} \ee^{-H^{\text{RBM}}(\bm{h})}.
\end{equation}
这里定义的 $H^{\text{RBM}}(\bm{v})$ 和 $H^{\text{RBM}}(\bm{h})$ 称为变分 Hamilton 量
(variational Hamiltonian)\cite{exact}。

\section{\texorpdfstring{CD\raisebox{0.13ex}{-}\textit{k}}{CD-k} 训练算法}
\label{sec:CDk-algorithm}

RBM 常采用对比散度算法(contrastive divergence, \CDk)进行训练,它是一种计算极大似然
(maximum likelihood)梯度值的近似方法
\cite{zhouzhihua,rbmonline,lyy1994rbm,hinton2012practical,hinton2002training}。

\subsection{损失函数}

对于一般的基于能量模型,其损失函数由下式给出 \cite{rbmonline}:
\begin{equation}
  \label{eq:ebm-loss-function}
  L(\bm{\theta})
  = - \likeL(\bm{\theta},\,\domD)
  = - \frac{1}{\abs{\domD}} \sum_{\bm{x}^{(i)}\in\domD} \ln P\qty\big(\bm{x}^{(i)}).
\end{equation}
式中的 $\domD$ 表示数据集,$\likeL$ 称为对数似然(log-likelihood)。根据梯度下降法的精神,我们需要
求出损失函数关于参数的导数(即梯度):
\begin{equation}
  \pdv{L(\bm{\theta})}{\bm{\theta}}
  = \frac{1}{\abs{\domD}} \sum_{\bm{x}^{(i)}\in\domD}
    - \pdv{\bm{\theta}} \ln P\qty\big(\bm{x}^{(i)}).
\end{equation}
实际上只需计算求和号里面的内容 $-\partial{\ln P\qty\big(\bm{x}^{(i)})}\,/\,\partial{\bm{\theta}}$。

RBM 所采用的训练数据集只包含了可见层的内容,即
\begin{equation}
  - \pdv{\bm{\theta}} \ln P\qty\big(\bm{x}^{(i)}) \equiv
  - \pdv{\bm{\theta}} \ln P\qty\big(\bm{v}^{(i)}).
\end{equation}
仿照统计物理的处理方法,引入自由能
\footnote{在统计物理中,自由能 $F=-T\ln Z$,而配分函数需要对所有可能的状态求和。但在限制
  Boltzmann 机中,只对隐藏层求和,所以 $F$ 是 $\bm{v}$ 的函数。} \cite{rbmonline}
\begin{equation}
  F(\bm{v}) = - \ln \sum_{\bm{h}} \ee^{-E(\bm{v},\,\bm{h})}.
\end{equation}
代入式~\eqref{eq:rbm-probability-v},可有
\begin{equation}
  \ln P\qty\big(\bm{v}^{(i)}) = \frac{\ee^{-F(\bm{v}^{(i)})}}{Z} \qc
  Z = \sum_{\bm{v}} \ee^{-F(\bm{v})}.
\end{equation}
于是梯度可以表示为
\begin{align}
  \label{eq:rmb-gradient}
     - \pdv{\bm{\theta}} \ln P\qty\big(\bm{v}^{(i)})
  &= - \pdv{\bm{\theta}} \qty\Big[-F\qty\big(\bm{v}^{(i)}) - \ln Z]
     \vphantom{\sum_x} \notag \\
  &= \pdv{F\qty\big(\bm{v}^{(i)})}{\bm{\theta}} + \frac{1}{Z} \pdv{Z}{\bm{\theta}}
   = \pdv{F\qty\big(\bm{v}^{(i)})}{\bm{\theta}}
     - \frac{1}{Z} \sum_{\bm{v}} \ee^{-F(\bm{v})} \pdv{F(\bm{v})}{\bm{\theta}} \notag \\
  &= \pdv{F\qty\big(\bm{v}^{(i)})}{\bm{\theta}}
     - \sum_{\bm{v}} P(\bm{v}) \pdv{F(\bm{v})}{\bm{\theta}}.
\end{align}
其中的参数 $\bm{\theta}$ 包含连接权重 $\bm{W}$、可见层偏差 $\bm{b}$ 与隐藏层偏差 $\bm{c}$。

式~\eqref{eq:rmb-gradient} 共包含两项,分别称为正相(positive phase)和负相(negative phase)。
显然,负相的计算比较困难,这也是 \CDk{} 算法着力需要解决的问题 \cite{rbmonline}。

\subsection{Gibbs 采样}

计算负相的困难之处在于需要对所有可能的可见层 $\bm{v}$ 分布进行求和,而这一分布对应的状态数量极其
巨大。与我们在处理 Ising 模型的解决方法类似,这里同样可以用一组样本的分布代替整体分布,即 Monte
Carlo 方法 \cite{rbmonline}:
\begin{equation}
  \sum_{\bm{v}} P(\bm{v}) \pdv{F(\bm{v})}{\bm{\theta}}
  \approx \frac{1}{\abs{\domN}} \sum_{\bm{v}\in\domN} \pdv{F(\bm{v})}{\bm{\theta}}.
\end{equation}
式中的 $\domN$ 表示一组依照概率 $P$ 进行的采样。

由于可见层、隐藏层的层内均没有连接,因此可以在层之间来回“跳转”,并将其视作一条 Markov 链,
如图~\ref{fig:gibbs-sampling} 所示。“跳转”的具体操作可以借由条件概率获得:设给定可见层分布
$\bm{v}$,则隐藏层分布为
\begin{equation}
  \label{eq:hidden-layer-dist}
  \bm{h} \sim P(\bm{h}\big|\bm{v});
\end{equation}
由此可计算下一组可见层分布
\begin{equation}
  \label{eq:visible-layer-dist}
  \bm{v}' \sim P(\bm{v}\big|\bm{h}).
\end{equation}
式中,符号 $\bm{x}\sim P$ 表示按照概率 $P$ 生成 $\bm{x}$。考虑到 $\bm{v}$ 与 $\bm{h}$ 均为布尔型,
我们可以近似用概率(的取整结果)直接代替 $\bm{v}$ 和 $\bm{h}$。

\begin{figure}[htb]
  \centering
  \imageinput{gibbs-sampling.pdf}
  \caption{Gibbs 采样的示意图。上标代表进行的采样次数}
  \label{fig:gibbs-sampling}
\end{figure}

以上步骤可以持续进行,这称为 Gibbs 采样(Gibbs sampling),采样次数即为 \CDk{} 算法中的 $k$。
当 $k\to\infty$ 时,可认为 $\bm{v}^{(k)}$、$\bm{h}^{(k+1)}$
\footnote{这里的上标 $(k)$ 和 $(k+1)$ 是采样次数,而样本指标 $(i)$ 不同。}
已经服从所要求的分布 $P(\bm{v})$。显然,$k\to\infty$ 在计算上是不可取的。实践中,往往选择 $k<15$
以提高计算速度。甚至在大多数情况下,取 $k=1$ 即可满足要求。此时,\eqref{eq:rmb-gradient}~式中的
梯度近似可以表示为 \cite{rbmonline,hinton2012practical,hinton2002training}
\begin{equation}
  \label{eq:rmb-gradient-approx}
  - \pdv{\bm{\theta}} \ln P\qty\big(\bm{v}^{(i)})
  \approx \pdv{F\qty\big(\bm{v}^{(i)})}{\bm{\theta}}
        - \pdv{F\qty\big({\bm{v}'}^{(i)})}{\bm{\theta}}.
\end{equation}
其中的 ${\bm{v}'}^{(i)}$ 表示经过 Gibbs 采样生成的可见层单元。

\subsection{梯度的计算}

首先来计算式~\eqref{eq:hidden-layer-dist}、\eqref{eq:visible-layer-dist} 中出现的条件概率
$P(\bm{h}|\bm{v})$。利用乘法公式,有 \cite{zhouzhihua,rbmonline}
\begin{align}
  \label{eq:conditional-probability}
  P(\bm{h}|\bm{v})
  &= \frac{P(\bm{v},\,\bm{h})}{P(\bm{v})}
   = \frac{\ee^{-E(\bm{v},\,\bm{h})}}%
          {\sum_{\tilde{\bm{h}}} \ee^{-E(\bm{v},\,\tilde{\bm{h}})}} \notag \\
  &= \frac{\exp\qty\big(\bm{v}^\trans\bm{W}\bm{h} + \bm{b}^\trans\bm{v} + \bm{c}^\trans\bm{h})}%
          {\sum_{\tilde{\bm{h}}} \exp\qty\big(\bm{v}^\trans\bm{W}\tilde{\bm{h}}
           + \bm{b}^\trans\bm{v} + \bm{c}^\trans\tilde{\bm{h}})}
   = \frac{\exp\qty\big[\qty\big(\bm{v}^\trans\bm{W} + \bm{c}^\trans)\bm{h}\,]}%
          {\sum_{\tilde{\bm{h}}} \exp\qty\big[\qty\big(\bm{v}^\trans\bm{W} + \bm{c}^\trans)
                                              \tilde{\bm{h}}\,]} \notag \\
  &= \frac{\exp\qty\big[\,\sum_j h_j \qty\big(c_j + \sum_i W_{ij}v_i)]}%
          {\sum_{\q{\tilde{h}_i}} \exp\qty\big[\,\sum_j \tilde{h}_j
                                               \qty\big(c_j + \sum_i W_{ij}v_i)]} \notag \\
  &= \prod_j \frac{\exp\qty\big[h_j \qty\big(c_j + \sum_i W_{ij}v_i)]}%
                  {\sum_{\q{\tilde{h}_i}} \exp\qty\big[\tilde{h}_j
                                               \qty\big(c_j + \sum_i W_{ij}v_i)]}
   = \prod_j P(h_j|\bm{v}).
\end{align}
考虑到 $h_j$ 只能取 0 或 1,又有 \cite{rbmonline,hinton2012practical,andrzejewski2009training}
\begin{align}
  P(h_j|\bm{v})
   = P(h_j=1|\bm{v})
  &= \frac{\exp 1 \cdot \qty\big(c_j + \sum_i W_{ij}v_i)}%
          {\exp 0 \cdot \qty\big(c_j + \sum_i W_{ij}v_i)
           + 1 \cdot \qty\big(c_j + \sum_i W_{ij}v_i)} \notag \\
  &= \frac{\exp\qty\big(c_j + \sum_i W_{ij}v_i)}{1 + \exp\qty\big(c_j + \sum_i W_{ij}v_i)}
   = \sigma\qty\big(c_j + \sum_i W_{ij}v_i).
\end{align}
同理,
\begin{equation}
  P(\bm{v}|\bm{h}) = \prod_i P(v_i=1|\bm{h}) = \prod_i \sigma\qty\big(b_i + \sum_j W_{ij}h_j).
\end{equation}
利用概率近似代替生成的 $\bm{h}$ 和 $\bm{v}'$,我们有
\begin{equation}
  \label{eq:h-v-approx}
  h_j  \approx \sigma\qty\big(c_j + \sum_i W_{ij}v_i) \qc
  v'_i \approx \sigma\qty\big(b_i + \sum_j W_{ij}h_j).
\end{equation}

接下来计算式~\eqref{eq:rmb-gradient-approx} 中自由能的偏导数。代入 Boltzmann 机的能量函数
\eqref{eq:rbm-energy}~式,可以把自由能写成
\begin{align}
  F(\bm{v})
  &= - \ln \sum_{\q{h_i}}
       \exp\qty\Big(\sum_{i,\,j} W_{ij} v_i h_j + \sum_i b_i v_i + \sum_j c_j h_j) \notag \\
  &= - \sum_i b_i v_i
     - \ln \sum_{\q{h_i}} \exp\qty\bigg[\sum_j h_j \qty\Big(c_j + \sum_i W_{ij} v_i)] \notag \\
  &= - \sum_i b_i v_i
     - \ln \prod_j \sum_{h_j} \exp\qty\bigg[h_j \qty\Big(c_j + \sum_i W_{ij} v_i)] \notag \\
  &= - \sum_i b_i v_i
     - \sum_j \ln \sum_{h_j} \ee^{h_j \qty\big(c_j + \sum_i W_{ij} v_i)}.
\end{align}
利用 $h_j$ 的布尔特性,可以将上式化简:
\begin{align}
  F(\bm{v})
  &= - \sum_i b_i v_i - \sum_j \ln
       \qty\bigg[  \ee^{0 \cdot \qty\big(c_j + \sum_i W_{ij} v_i)}
                 + \ee^{1 \cdot \qty\big(c_j + \sum_i W_{ij} v_i)}] \notag \\
  &= - \sum_i b_i v_i - \sum_j \ln \qty\Big(1 + \ee^{c_j + \sum_i W_{ij} v_i}).
\end{align}
此时可以计算出自由能关于各参量的偏导数 \cite{rbmonline}:
\begin{align}
  \pdv{F(\bm{v})}{W_{ij}} &= - \pdv{W_{ij}} \ln \qty\Big(1+\ee^{c_j+\sum_iW_{ij}v_i})
                           = - v_i \cdot \frac{\ee^{c_j+W_{ij}v_i}}{1+\ee^{c_j+W_{ij}v_i}}
                           = - v_i \cdot \sigma\qty\Big(c_j+\sum_iW_{ij}v_i); \\
  \pdv{F(\bm{v})}{b_i}    &= - v_i; \\
  \pdv{F(\bm{v})}{c_j}    &= - \pdv{c_j} \ln \qty\Big(1+\ee^{c_j+\sum_iW_{ij}v_i})
                           = - \frac{\ee^{c_j+\sum_iW_{ij}v_i}}{1+\ee^{c_j+\sum_iW_{ij}v_i}}
                           = - \sigma\qty\Big(c_j+\sum_iW_{ij}v_i).
\end{align}
代入式~\eqref{eq:h-v-approx},并改写成矩阵形式:
\begin{equation}
  \pdv{F(\bm{v})}{\bm{W}} = - \bm{v} \bm{h}^\trans \qc
  \pdv{F(\bm{v})}{\bm{b}} = - \bm{v} \qc
  \pdv{F(\bm{v})}{\bm{c}} = - \bm{h}.
\end{equation}
此即式~\eqref{eq:rmb-gradient-approx} 中的正相部分。

通过 Gibbs 采样,负相部分的形式与正相部分完全一致,只要把 $\bm{v}$、$\bm{h}$ 换成 $\bm{v}'$、
$\bm{h}'$,因此有
\begin{equation}
  \pdv{F(\bm{v}')}{\bm{W}} = - \bm{v}' {\bm{h}'}^\trans \qc
  \pdv{F(\bm{v}')}{\bm{b}} = - \bm{v}' \qc
  \pdv{F(\bm{v}')}{\bm{c}} = - \bm{h}'.
\end{equation}
至此,我们便得到了梯度下降法所需要的参数更新公式 \cite{zhouzhihua,hinton2012practical}:
\begin{equation}
  \left\{
  \begin{aligned}
    \bm{W} \! &\gets \bm{W} \! - \alpha \qty\big(\bm{v} \bm{h}^\trans
                                                 - \bm{v}' {\bm{h}'}^\trans), \\
    \bm{b} \, &\gets \, \bm{b} - \alpha \qty\big(\bm{v} - \bm{v}'), \\
    \bm{c} \, &\gets \, \bm{c} - \alpha \qty\big(\bm{h} - \bm{h}').
  \end{aligned}
  \right.
\end{equation}
式中的 $\alpha$ 为学习率。

\section{RBM 的应用:手写数字}
\label{sec:rbm-mnist}

本节我们将利用 MNIST 数据集训练 RBM,并基于已有数据生成新的数据,即让计算机学会“书写”数字
\cite{rbmonline,lyy1994rbm}。

MNIST 数据集 \num{60000} 张灰度图片组成,每张图片表示一个手写数字,其大小被统一为 $28 \times 28$
像素,并保证数字处于居中位置 \cite{mnist}。RBM 的训练需要使用布尔型数据,我们也据此将图片进行了
二值化处理,如图~\ref{fig:mnist-image} 所示。

\begin{figure}[htb]
  \centering
  \begin{subfigure}[b]{0.35\textwidth}
    \centering
    \imageinput[width=2cm]{mnist-image.pdf}
    \caption{}
  \end{subfigure}
  \begin{subfigure}[b]{0.35\textwidth}
    \centering
    \imageinput[width=2cm]{mnist-image-binary.pdf}
    \caption{}
  \end{subfigure}
  \caption{MNIST 数据集中的首张图片,对应数字 5。
    (a)~原始图像,灰度值从白色到黑色分别对应 0--255;
    (b)~二值化后的图像,只包含值 0(白色)和 1(黑色)}
  \label{fig:mnist-image}
\end{figure}

具体的训练参数如下:

\begin{enumerate}
  \item 可见层单元数量:784($=28 \times 28$),即把图片压平为一维向量;
  \item 隐藏层单元数量:100;
  \item 批次(batch)大小:64,表示每次同时计算的样本数量;
  \item 训练轮次:20,一个轮次表示将整个数据集训练一遍;
  \item \CDk{} 中的参数 $k=30$;
  \item 学习率 $\alpha=0.1$。
\end{enumerate}

在训练过程中,我们需要监测损失函数的变化。从图~\ref{fig:learning-curve} 中可以看到,随着步数的
增加,损失函数迅速下降,并很快趋于平稳。这说明 RBM 的训练情况良好。

\begin{figure}[htb]
  \centering
  \imageinput{learning-curve.pdf}
  \caption{学习曲线。这里的步数(steps)等于批次数与训练轮次的乘积,注意取了对数坐标}
  \label{fig:learning-curve}
\end{figure}

\begin{figure}[htb]
  \centering
  \begin{subfigure}[b]{0.3\textwidth}
    \centering
    \imageinput[width=3cm]{mnist-weight-epoch-1.pdf}
    \caption{}
    \label{fig:mnist-weight-epoch-a}
  \end{subfigure}
  \begin{subfigure}[b]{0.3\textwidth}
    \centering
    \imageinput[width=3cm]{mnist-weight-epoch-6.pdf}
    \caption{}
    \label{fig:mnist-weight-epoch-b}
  \end{subfigure}
  \begin{subfigure}[b]{0.3\textwidth}
    \centering
    \imageinput[width=3cm]{mnist-weight-epoch-20.pdf}
    \caption{}
    \label{fig:mnist-weight-epoch-c}
  \end{subfigure}
  \caption{训练过程中的权重矩阵(部分)。权重矩阵 $\bm{W}$ 的形状为 $100 \times 784$,这里将其
    变形为 100 个 $28 \times 28$ 的子矩阵,以便更加清晰地呈现出从可见层到隐藏层的映照关系。
    (a)~第 1 个训练轮次;(b)~第 6 个训练轮次;(c)~第 20 个训练轮次}
  \label{fig:mnist-weight-epoch}
\end{figure}

与此同时,我们还需要监测权重矩阵的变化情况 \cite{lyy1994rbm}。初始时,权重矩阵 $\bm{W}$ 中的所有
元素均取为 0 到 1之间的随机数。经过一个轮次的训练后,$\bm{W}$ 中出现了一些明暗变化(图~%
\ref{fig:mnist-weight-epoch-a});到第 6 个轮次,可以明显地看到 $\bm{W}$ 中的展现出了笔画的特征,
但仍有较大噪音(图~\ref{fig:mnist-weight-epoch-b});到第 20 轮即训练结束时,各子权重矩阵均显著
出现了明暗对比,具有清晰的笔画线条,并且噪音较小(图~\ref{fig:mnist-weight-epoch-c}
和图~\ref{fig:mnist-weight})。注意到图~\ref{fig:mnist-weight} 中仍有一些子权重矩阵的训练效果不
理想,出现了较大的涨落。但可以预料,随着训练轮次的继续增加,涨落还会逐渐减小。

\begin{figure}[!htb]
  \centering
  \imageinput[width=10cm]{mnist-weight.pdf}
  \caption{最终生成的权重矩阵}
  \label{fig:mnist-weight}
\end{figure}

\begin{figure}[!htb]
  \centering
  \imageinput[width=12cm]{mnist-samples.pdf}
  \caption{由训练完成的 RBM 生成的数字图像。第一行为原始图像,之后每一行增加 \num{500} 次 Gibbs
    采样,到最后一行共进行了 \num{4000} 次采样}
  \label{fig:mnist-samples}
\end{figure}

最后,我们根据训练得到的数据生成新的手写数字。首先从 MNIST 数据集中随机选取一些样本图像作为可见层,
利用 Gibbs 采样生成隐藏层,再以同样方式返回到可见层,以此类推进行迭代,结果见图~%
\ref{fig:mnist-samples}。经过若干次采样之后,RBM 确实给出了不同于原始图像的分布,即学会了“书写”
数字。对于笔画比较简单的数字,如 1、7 等,学习效果较好;但对于稍复杂的数字,如 3、6 等,训练后的
RBM 给出了错误的结果;另外有些数字,如 5 和 9,RBM 只给出了模糊的阴影,并没有很好地复现原始图像。
这说明 RBM 训练算法仍有可以提高的余地。

\section{RBM 与重整化群的对应关系}

RBM 的思想与重整化群是十分类似的。在 RBM 中,我们把可见层中的信息编码到了隐藏层中;而在重整化群
方法中,我们把自旋进行粗粒近似,得到重整化后的 Hamilton 量。如果把 RBM 中的可见层理解为原始自旋
点阵,而把隐藏层理解为粗粒近似下的 Kadanoff 集团,我们便可以得到 RBM 与重整化群的对应关系。
在式~\eqref{eq:Z-H-renormalization} 中,我们给出了重整化前后 Hamilton 量的对应关系 \cite{exact}:
\begin{equation}
    \ee^{-H^{\text{RG}}(\bm{h})}
  = \sum_{\bm{v}} \ee^{\opT(\bm{v},\,\bm{h})-H(\bm{v})}.
\end{equation}
这里我们已经把自旋 $\sigma_i'$、$\sigma_i'$ 用 RBM 的语言 $\bm{v}$、$\bm{h}$ 进行表述,
并把 $\beta$ 归入了 $H$。接下来我们令
\begin{equation}
  \opT(\bm{v},\,\bm{h}) = H(\bm{v}) - E(\bm{v},\,\bm{h}).
\end{equation}
代入 \eqref{eq:rbm-probability-h}~式,可得
\begin{align}
  &\mathrel{\phantom{\implies}}
    \frac{1}{Z} \ee^{-H^{\text{RG}}(\bm{h})}
  = \frac{1}{Z} \sum_{\bm{v}} \ee^{-E(\bm{v},\,\bm{h})}
  = P(\bm{h})
  = \frac{1}{Z} \ee^{-H^{\text{RBM}}(\bm{h})} \notag \\
  &\implies H^{\text{RG}}(\bm{h}) = H^{\text{RBM}}(\bm{h}).
\end{align}
由此就建立了 RBM 与重整化群的严格映照。

利用式~\eqref{eq:rbm-probability-v}、\eqref{eq:conditional-probability} 进行计算,可以证明
\cite{exact}
\begin{equation}
  \ee^{\opT(\bm{v},\,\bm{h})} = P(\bm{h}|\bm{v}) \ee^{H(\bm{v})-H^{\text{RBM}}(\bm{v})}.
\end{equation}
当重整化变换 $\opT(\bm{v},\,\bm{h})$ 可以使得自旋集团严格重现原始自旋的自由能时,我们有
\begin{equation}
  \sum_{\bm{h}} \ee^{\opT(\bm{v},\,\bm{h})} = 1;
\end{equation}
另一方面,给定 $\bm{v}$ 下的条件概率 $P(\bm{h}|\bm{v})$ 满足
\begin{equation}
  \sum_{\bm{h}} P(\bm{h}|\bm{v}) = 1.
\end{equation}
这样可以得到
\begin{equation}
  H^{\text{RBM}}(\bm{v}) = H(\bm{v}),
\end{equation}
这说明经由 RBM 获得的概率分布严格等价于原始数据的分布。

\section{利用 RBM 研究 Ising 模型}

鉴于 RBM 与重整化群之间存在一一对应的关系,利用 Ising 模型点阵数据训练 RBM 之后,我们得到的结果
就相当于进行一次粗粒近似的重整化自旋集团。

为了与 MNIST 数据集保持一致,我们采用由 Metropolis 算法生成的 $28 \times 28$ 点阵。采样温度为
$\Tc/J+\num{0.03}$,外磁场为零。我们一共生成了 \num{60000} 组点阵数据,其中前 \num{59000} 组作为
训练集,其余作为测试集。RBM 的训练参数与 \ref{sec:rbm-mnist}~节中相同,除了训练轮次改为 50,
以得到更好的训练结果。待训练结束后,给出权重矩阵(图~\ref{fig:ising-weight})以及由 RBM 重建
而成的 Ising 模型点阵(图~\ref{fig:ising-samples})。

\begin{figure}[htb]
  \centering
  \imageinput[width=10cm]{ising-weight.pdf}
  \caption{训练结束时的权重矩阵}
  \label{fig:ising-weight}
\end{figure}

\begin{figure}[htb]
  \centering
  \imageinput[width=12cm]{ising-samples.pdf}
  \caption{由训练完成的 RBM 重建的 Ising 模型点阵。第一行为原始数据,之后每一行增加 \num{500}
    次 Gibbs 采样,到最后一行共进行了 \num{4000} 次采样}
  \label{fig:ising-samples}
\end{figure}

与 MNIST 数据集上的训练结果类似,权重矩阵给出了数据集的一些特征信息。一些子权重矩阵中出现了明暗相间
的条纹(如 6 行 8 列、6 行 10 列、8 行 1 列等),这种权重可以识别出不同相的边界;更多的子矩阵中
则出现了暗点(如 1 行 2 列、3 行 7 列、7 行 5 列等),它们反映了自旋的局域性信息,即在一个自旋集团
中,只有中心位置的原始自旋会对粗粒近似下的有效自旋产生贡献,而其余位置的原始自旋则贡献很小。由 RBM
所重建出的 Ising 模型点阵大致刻画了原始点阵的性质。对于明显具有标度变换不变性的结构(如第 6 列),
重建点阵能够保持住原有的“花斑”图样。但与 MNIST 类似,RBM 的学习效果并不十分理想。例如第 4 列,重建
点阵中存在明显的斑块,而这在原始数据中是不存在的。

\begin{figure}[htb]
  \centering
  \imageinput[width=12cm]{ising-hidden.pdf}
  \caption{RBM 所给出的隐藏层。这是在图~\ref{fig:ising-samples} (除第一行)的基础上,额外进行一次
    由可见层 $\bm{v}$ 到隐藏层 $\bm{h}$ 的 Gibbs 采样所得到的结果}
  \label{fig:ising-hidden}
\end{figure}

我们还给出了由 RBM 生成的隐藏层(图~\ref{fig:ising-hidden})。事实上,该图并不能反映出实质性的
物理信息。根据重整化群的思想,粗粒近似需要不断重复进行直至不动点的产生。因此我们可以推断,单层 RBM
所揭示的物理图像并不完善,而只有 RBM 组合成多层神经网络[称为深度信念网络(deep belief network,
DBN)]\cite{exact},才可以比较全面地展现出与重整化群的严格对应关系。

\section{卷积层}

\subsection{卷积}

在训练 RBM 的过程中,原始数据中的二维矩阵(MNIST 中的二维图像,或二维 Ising 模型的自旋构型)都被
压平成了一维向量。因此,原始数据中有关局域性(locality)的信息便会被全部丢失,没有被机器学习
利用起来。

为了克服这一问题,我们可以引入卷积层(convolution layer)。数学上,一维体系下的卷积可以定义如下
\cite{wiki:convolution}:
\begin{equation}
  \label{eq:convolution-continuous}
  (f \ast g) (t) = \int_{-\infty}^{+\infty} f(\tau) \, g(t-\tau) \dd{\tau}.
\end{equation}
对于离散情况,有
\begin{equation}
  (f \ast g) [n] = \sum_{k=-\infty}^{+\infty} f[k] \, g[k-n] \qc n,\,k \in \Z.
\end{equation}

卷积可以理解为一种“加权平均”。将卷积核 $f$ 与相同长度的“信号” $g$ 的片段逐项相乘,并依次放置在
相应位置,即完成了 $f$ 与 $g$ 的卷积。

将以上定义推广至高维,可得
\begin{equation}
  (f \ast g) [\bm{n}] = \sum_{\bm{k}} f[\bm{k}] \, g[\bm{k}-\bm{n}] \qc
  n=\mqty[n_1 \\ n_2 \\ \vdots \\ n_N], \, k=\mqty[k_1 \\ k_2 \\ \vdots \\ k_N] \in \Z^N.
\end{equation}
对于二维图像处理的问题,可以把卷积核 $f$ 理解为“滤镜”(filter),其过程如图~\ref{fig:convolution}
所示。事实上,如果取卷积核为 Gauss 函数
\begin{equation}
  f(\bm{x}) = \frac{1}{2\pp\sigma^2} \exp(-\frac{\norm{\bm{x}}^2}{2\sigma^2}),
\end{equation}
与待处理的图像进行卷积时,每次做“加权平均”,由于 Gauss 函数离中心较远处收敛迅速,所以中心像素
所占权重最大;而随着到中心的距离增加,各像素贡献也逐渐减小。整体来看,每个像素均用其周围一部分像素
的加权平均来代替,因而最终的结果就是对原始数据的 Gauss 模糊。

\begin{figure}[htb]
  \centering
  \imageinput{convolution.pdf}
  \caption{二维卷积示意图 \cite{liamconvolution}。随着卷积核(“滤镜”)在原始图像上移动,最终得到
    结果的就是以各像素为中心的局部加权平均}
  \label{fig:convolution}
\end{figure}

\subsection{小波变换}

类似于 Fourier 变换,小波变换(wavelet transform)也是一种函数变换。Fourier 变换将函数从时域转化为
频域(物理上等价于从坐标空间变换到动量空间),即用具有不同频率的基函数(三角函数)将原始函数展开。
与之类似,小波变换用一组母小波(mother wavelet)来展开原始函数。不同于三角函数,母小波具有收敛
迅速的特点,因此利用母小波的平移,在给出频域特征的同时还可以保留一定的时域特征 \cite{wiki:wavelet}。

在 $d$ 维空间中,利用母小波 $\psi$ 的缩放和平移可以张成一组基 \cite{wiki:wavelet,padic}:
\begin{equation}
  \label{eq:daughter-wavelet}
  \psi_{\bm{a},\,s}(\bm{x}) = \frac{1}{s^{d/2}} \psi\qty(\frac{\bm{x}-\bm{a}}{s}),
\end{equation}
其中的 $s$ 和 $\bm{a}$ 分别表征了 $\psi$ 的缩放和平移。这组基称为子小波(daughter wavelet)。
任意函数可以利用子小波展开:
\begin{equation}
  W_f(\bm{a},\,s) = \int \dd[d]{x} f(\bm{x}) \psi^\dagger_{\bm{a},\,s}(\bm{x}).
\end{equation}
这里 $\psi^\dagger$ 是 $\psi$ 的对偶小波。其逆变换为
\footnote{由于数学上的某些限制,逆变换未必存在。}
\begin{equation}
  f(\bm{x}) = \frac{1}{C_\psi} \int_0^\infty \frac{\dd{s}}{s^{d+1}}
              \int \dd[d]{a} W_f(\bm{a},\,s) \psi_{\bm{a},\,s}(\bm{x}).
\end{equation}
式中的 $C_\psi$ 为一系数。

与 \eqref{eq:convolution-continuous}~式进行对照,可以看到由于平移的存在,小波变换也可以理解为
一种卷积,它同样包括了卷积核(小波基)在原始函数上平移之后求和(积分)的操作。

\subsection{\AdSCFT{} 与小波变换}

在 \AdSCFT{} 中,区块(bulk)中可规范化的局域场 $\phi^i(\bm{x},z)$ 可以利用边界算符
(boundary operator)进行重建 \cite{padic}:
\begin{align}
  &\mathrel{\phantom{=}} \phi^i(\bm{x},z) \notag \\
  &= \int \dd[d]{y} K_i(\bm{x},z|\bm{y})  \opO^i(\bm{y}) \notag \\
  &+ \sum_{j,k} \frac{\lambda^i_{jk}}{N}
       \int \dd[d]{x'} \dd{z'} G_i(\bm{x},z|\bm{x}'\!,z')
       \int \dd[d]{y_1} K_j(\bm{x}'\!,z'|\bm{y}_1)  \opO^j(\bm{y}_1)
       \int \dd[d]{y_2} K_k(\bm{x}'\!,z'|\bm{y}_2)  \opO^k(\bm{y}_2) \notag \\
  &+ \bigO\qty(\frac{\lambda^2}{N^2}) \int \cdots
\end{align}
其中的 $\opO^i$ 是与区块中场 $\phi^i$ 对偶的边界算符:
\begin{equation}
  \lim_{z \to 0} \phi^i(\bm{x},z) = z^{-\Delta_i} \opO^i(\bm{x}),
\end{equation}
$\Delta_i$ 是 $\opO^i$ 的共形维数(conformal dimension);$K_i(\bm{x},z|\bm{y})$ 是 $\opO^i$
的边界—区块核(boundary--bulk kernel),称为模糊函数(smearing function),其定义为
\begin{equation}
  \label{eq:boundary-bulk-kernel}
  K_i(\bm{x},z|\bm{y})
  = \qty[\frac{z}{z^2-\norm{\bm{x}-\bm{y}}^2}]^{d-\Delta_i} \,
    \Theta\qty\big(z^2-\norm{\bm{x}-\bm{y}}^2);
\end{equation}
其中的 $\Theta$ 表示阶跃函数;$G_i(\bm{x},z|\bm{x}'\!,z')$ 是区块—区块核。
边界—区块核与区块—区块核有如下关系:
\begin{equation}
  \lim_{z' \to 0} {z'}^{\Delta_i-d} G_i(\bm{x},z|\bm{x}'\!,z') \sim K_i(\bm{x},z|\bm{x}').
\end{equation}
以上 $G^i$、$K_i$ 均可理解为传播子(propagator),或 Green 函数。

注意到式~\eqref{eq:boundary-bulk-kernel} 中 $K_i$ 的形式与小波变换非常类似 \cite{padic}。
我们取母小波为
\begin{equation}
  \psi_\Delta(\bm{x}) = \qty(\frac{1}{1-\norm{\bm{x}}^2})^{d-\Delta} \,
                        \Theta\qty\big(1-\norm{\bm{x}}^2)
\end{equation}
根据式~\eqref{eq:daughter-wavelet} 生成子小波,有
\begin{equation}
  \psi_{\bm{a},\,s}(\bm{x})
  = \frac{1}{z^{d/2}} \psi\qty(\frac{\bm{y}-\bm{x}}{z})
  = \frac{1}{z^{d/2}} \qty(\frac{1}{1-\norm{\frac{\bm{y}-\bm{x}}{z}}^2})^{d-\Delta}
    \Theta\qty(1-\norm{\frac{\bm{y}-\bm{x}}{z}}^2).
\end{equation}
这样即可得到
\begin{equation}
  K(\bm{x},z|\bm{y}) = z^{\Delta-d/2} \psi_{\bm{x},\,z}(\bm{y}).
\end{equation}

场 $\phi^i(\bm{x},z)$ 同样可由边界算符 $\opO_\Delta$ 的小波变换表示:
\begin{equation}
  \phi^i(\bm{x},z) = z^{\Delta-d/2} W_{\opO}(\bm{x},z).
\end{equation}
此处,Poincaré 坐标中的 $z$ 由缩放参数决定。

\section{卷积限制 Boltzmann 机(CRBM)}

利用 RBM 处理 Ising 模型,只能给出与重整化群相对应的定性关系,而不能给出具有较为深刻物理背景的
定量结果。考虑到 Ising 模型作为二维共形场论中最简单的最小模型(minimal model),我们可以从
\AdSCFT{} 的角度出发做一点讨论。

前面已经指出,\AdSCFT{} 提供了传播子 $K_i$ 与小波变换的对应关系,而小波变换则有可以被理解为某种
形式的卷积。因此我们认为,利用卷积网络处理 Ising 模型,则机器所学习到的卷积核就可以给出传播子
$K_i$ 的定量结果。

\subsection{网络构建}

在图~\ref{fig:rbm} 中,我们给出了 RBM 的网络构建。与图~\ref{fig:neural-net} 比较,可以看出二者
十分相似。事实上,RBM 正是基于线性网络的能量模型,它的核心是在线性映照
$\bm{x}\to\bm{W}\bm{x}+\bm{b}$ 基础上给出的能量函数 \eqref{eq:rbm-energy}~式。

\begin{figure}[htb]
  \centering
  \imageinput{crbm.pdf}
  \caption{卷积限制 Boltzmann 机示意图。绿色菱形表示卷积核,它们作用在可见层(红色圆圈)上,
    并生成隐藏层(蓝色方块)。与图~\ref{fig:rbm} 不同,这里单元之间的连线并不表示权重。不同深浅的
    连线展现了卷积核与可见层不同位置的对应。需要注意,可见层、隐藏层与卷积核实际上均是二维矩阵,
    它们相比该示意图额外还有一个垂直纸面的维度}
  \label{fig:crbm}
\end{figure}

下面我们需要把线性网络改变成卷积网络,称为卷积 RBM(CRBM),如图~\ref{fig:crbm} 所示
\cite{lee2009convolutional,lee2011unsupervised}。线性网络中,可见层与隐藏层之间有如下关系:
\begin{equation}
  \bm{h}+\bm{c} \sim \bm{W} (\bm{v}+\bm{b}).
\end{equation}
写出分量形式,为
\begin{equation}
  h_j+c_j \sim \sum_i W_{ij} (v_i+b_i).
\end{equation}
这里用“$\sim$”表示其间还可以增加激活函数,如 $\sigma$ 函数等。在卷积网络中,可见层与隐藏层
则通过下式相连接:
\begin{equation}
  h_{\bm{j}} + c_{\bm{j}} \sim \sum_{\bm{i}} W_{\bm{i}} (v_{\bm{i}+\bm{j}} + b_{\bm{i}+\bm{j}}).
\end{equation}
注意这里的指标 $\bm{i}$、$\bm{j}$ 均有两个维度,而不再是一个整数,且求和需要对两个维度分别进行。
类比 \eqref{eq:rbm-energy}~式,我们可以写出 CRBM 的能量函数
\cite{lee2009convolutional,lee2011unsupervised}:
\begin{equation}
  E(\bm{v},\,\bm{h})
  = - \sum_{\bm{i},\,\bm{j}} W_{\bm{i}} v_{\bm{i}+\bm{j}} h_{\bm{j}}
    - \sum_{\bm{i}} b_{\bm{i}} v_{\bm{i}} - \sum_{\bm{j}} c_{\bm{j}} h_{\bm{j}}.
\end{equation}
除了第一项,CRBM 与 RBM 的能量函数在形式上几乎完全一致。但由于参数空间增加了一个维度,这种改变
实际上是非平庸的。

\subsection{训练方法}

CRBM 的损失函数仍采用 \eqref{eq:ebm-loss-function}~式的形式,因此与 RBM 相同,也可以使用 \CDk{}
算法训练。仿照 \ref{sec:CDk-algorithm}~节的推导过程,我们可以计算得到 CRBM 的参数更新公式
\cite{lee2009convolutional,lee2011unsupervised}:
\begin{equation}
  \label{eq:crbm-parameters-update}
  \left\{
  \begin{aligned}
    W_{\bm{i}}    &\gets W_{\bm{i}} - \alpha
                         \qty\Big(  \sum_{\bm{j}} v_{\bm{i}+\bm{j}} h_{\bm{j}}
                                  - \sum_{\bm{j}} v'_{\bm{i}+\bm{j}} h'_{\bm{j}}), \\
    b_{\bm{i}} \! &\gets \! b_{\bm{i}} - \alpha \qty\big(v_{\bm{i}} - v'_{\bm{i}}),   \\
    c_{\bm{j}} \! &\gets \! c_{\bm{j}} - \alpha \qty\big(h_{\bm{j}} - h'_{\bm{j}}).
  \end{aligned}
  \right.
\end{equation}
式中的 $\alpha$ 为学习率。利用梯度下降法,即可据此进行训练。

由于参数空间提高到了二维,训练参数的数目大大增加。根据 \ref{subsec:regularization}~小节中的讨论,
为了缓解过拟合问题,我们有必要在损失函数中引入正则项。一种有效的手段是强制参数稀疏化,它只需要修改
式~\eqref{eq:crbm-parameters-update} 中参数 $\bm{b}$ 的更新规则
\cite{lee2009convolutional,lee2011unsupervised}:
\begin{equation}
  \incr{b_{\bm{i}}} = \incr{b_{\bm{i}}}^0 + \incr{b}^{\text{sparsity}},
\end{equation}
其中 $\incr{b_{\bm{i}}}^0$ 仍按式~\eqref{eq:crbm-parameters-update} 选取,而稀疏化项则为
\begin{equation}
  \incr{b}^{\text{sparsity}}
  \propto p - \frac{1}{N_{\text{hidden}}} \sum_{\bm{j}} P\qty\big(h_{\bm{j}}=1|\bm{v}).
\end{equation}
