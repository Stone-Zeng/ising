\chapter{机器学习与重整化群}

\section{限制 Boltzmann 机(RBM)}

限制 Boltzmann 机(restricted Boltzmann machine, RBM)是一类生成型随机神经网络( generative
stochastic artificial neural network ),它可以学习到输入数据的概率分布情况。顾名思义,RBM 是
Boltzmann 机加上一定限制所得到的网络。一般的 Boltzmann 机是一种基于能量的模型(energy-based
model)。对于这种模型,我们可以赋予它一个能量函数。与物理学中类似,体系的稳定状态将在能量最低时
达到,这一稳定状态即对应最优参数。

Boltzmann 机的结构如图~\ref{fig:boltzmann-machine} 所示。它共有两层,与数据直接相连的称为可见层
(visible layer),另一层称为隐藏层(hidden layer)。可见层用于处理输入输出,隐藏层则反应了数据的
内在结构。无论是可见层还是隐藏层,其中的所有单元均全部连接在一起。Boltzmann 机的能量函数由下式给出:
\begin{equation}
  E(\bm{s}) = E(\q{s_i}) = - \sum_{i<j} W_{ij} s_i s_j - \sum_i \theta_i s_i.
\end{equation}
式中,$s_i$ 取为布尔型,即 $s_i\in\q{0,\,1}$。$W_{ij}$ 是单元 $s_i$ 与 $s_j$ 之间的连接权重,
$\theta_i$ 则表示单元 $s_i$ 处的偏差。

\begin{figure}[htb]
  \centering
  \begin{subfigure}[b]{0.45\textwidth}
    \centering
    \tikzinput{boltzmann-machine}
    \caption{Boltzmann 机}
    \label{fig:boltzmann-machine}
  \end{subfigure}
  \begin{subfigure}[b]{0.45\textwidth}
    \centering
    \tikzinput{rbm}
    \vspace{0.4cm}
    \caption{限制 Boltzmann 机(RBM)}
    \label{fig:rbm}
  \end{subfigure}
  \caption{Boltzmann 机和限制 Boltzmann 机的结构示意图。其中红色圆圈表示可见层,蓝色方块表示隐藏层。
  Boltzmann 机的所有单元之间均有权重矩阵连接;而在限制 Boltzmann 机中,只有可见层与隐藏层之间才连接
  有权重矩阵}
\end{figure}

与 Ising 模型类似[式~\eqref{eq:ising-probability}],状态 $\bm{s}=\q{s_i}$ 出现的概率由 Boltzmann
分布给出:
\begin{equation}
  \label{eq:boltzmann-machine-probability}
  P(\bm{s}) = \frac{\ee^{-E(\bm{s})}}{Z},
\end{equation}
其中配分函数 $Z$ 的定义为
\begin{equation}
  Z = \sum_{\bm{s}} \ee^{-E(\bm{s})}.
\end{equation}
Boltzmann 机的训练过程就是最大化训练样本所对应的概率。

由于 Boltzmann 机是全连接的,它的训练效率并不高。我们可以引入限制条件,仅保留可见层与隐藏层之间的
连接。由此便得到了限制 Boltzmann 机(如图~\ref{fig:rbm}),它的能量函数为
\begin{equation}
  \label{eq:rbm-energy}
  E(\bm{v},\,\bm{h})
  = - \bm{v}^\trans \bm{W} \bm{h} - \bm{b}^\trans \bm{v} - \bm{c}^\trans \bm{h}
  = - \sum_{i,\,j} W_{ij} v_i h_j - \sum_i b_i v_i - \sum_j c_j h_j.
\end{equation}
式中,$v_i$ 和 $h_j$ 分别代表可见层与隐藏层中的单元,$W_{ij}$ 为两层之间的连接权重,$b_i$ 和 $c_j$
分别是可见层与隐藏层中单元对应的偏差。

RBM 中,可见层 $\bm{v}$ 与隐藏层 $\bm{h}$ 是彼此分离的,可以视为两个随机变量,因而
式~\eqref{eq:boltzmann-machine-probability} 即成为 $\bm{v}$ 与 $\bm{h}$ 的联合概率分布:
\begin{equation}
  P(\bm{v},\,\bm{h})
  = \frac{\ee^{-E(\bm{v},\,\bm{h})}}{Z}
  = \frac{\ee^{-E(\bm{v},\,\bm{h})}}{\sum_{\bm{v},\,\bm{h}}\ee^{-E(\bm{v},\,\bm{h})}}.
\end{equation}
对所有的 $\bm{h}$ 进行求和,我们就得到了某一可见层 $\bm{v}$ 对应的边缘概率分布:
\begin{equation}
  \label{eq:rbm-probability}
  P(\bm{v}) = \frac{1}{Z} \sum_{\bm{h}} P(\bm{v},\,\bm{h})
            = \sum_{\bm{h}} \ee^{-E(\bm{v},\,\bm{h})}.
\end{equation}

\section{CD-\textit{k} 训练算法}

RBM 常采用对比散度算法(contrastive divergence, \CDk)进行训练,它是一种计算极大似然梯度值的
近似方法。

\subsection{损失函数}

对于一般的基于能量模型,其损失函数由下式给出:
\begin{equation}
  L(\bm{\theta})
  = - \likeL(\bm{\theta},\,\domD)
  = - \frac{1}{\abs{\domD}} \sum_{\bm{x}^{(i)}\in\domD} \ln P\qty\big(\bm{x}^{(i)}).
\end{equation}
式中的 $\domD$ 表示数据集,$\likeL$ 称为对数似然(log-likelihood)。根据梯度下降法的精神,我们需要
求出损失函数关于参数的导数(即梯度):
\begin{equation}
  \pdv{L(\bm{\theta})}{\bm{\theta}}
  = \frac{1}{\abs{\domD}} \sum_{\bm{x}^{(i)}\in\domD}
    - \pdv{\bm{\theta}} \ln P\qty\big(\bm{x}^{(i)}).
\end{equation}
实际上只需计算求和号里面的内容 $-\partial{\ln P\qty\big(\bm{x}^{(i)})}\,/\,\partial{\bm{\theta}}$。

RBM 所采用的训练数据集只包含了可见层的内容,即
\begin{equation}
  - \pdv{\bm{\theta}} \ln P\qty\big(\bm{x}^{(i)}) \equiv
  - \pdv{\bm{\theta}} \ln P\qty\big(\bm{v}^{(i)}).
\end{equation}
仿照统计物理的处理方法,引入自由能
\footnote{在统计物理中,自由能 $F=-T\ln Z$,而配分函数需要对所有可能的状态求和。但在限制
  Boltzmann 机中,只对隐藏层求和,所以 $F$ 是 $\bm{v}$ 的函数。}
\begin{equation}
  F(\bm{v}) = - \ln \sum_{\bm{h}} \ee^{-E(\bm{v},\,\bm{h})}.
\end{equation}
代入式~\eqref{eq:rbm-probability},可有
\begin{equation}
  \ln P\qty\big(\bm{v}^{(i)}) = \frac{\ee^{-F(\bm{v}^{(i)})}}{Z} \qc
  Z = \sum_{\bm{v}} \ee^{-F(\bm{v})}.
\end{equation}
于是梯度可以表示为
\begin{align}
  \label{eq:rmb-gradient}
     - \pdv{\bm{\theta}} \ln P\qty\big(\bm{v}^{(i)})
  &= - \pdv{\bm{\theta}} \qty\Big[-F\qty\big(\bm{v}^{(i)}) - \ln Z]
     \vphantom{\sum_x} \notag \\
  &= \pdv{F\qty\big(\bm{v}^{(i)})}{\bm{\theta}} + \frac{1}{Z} \pdv{Z}{\bm{\theta}}
   = \pdv{F\qty\big(\bm{v}^{(i)})}{\bm{\theta}}
     - \frac{1}{Z} \sum_{\bm{v}} \ee^{-F(\bm{v})} \pdv{F(\bm{v})}{\bm{\theta}} \notag \\
  &= \pdv{F\qty\big(\bm{v}^{(i)})}{\bm{\theta}}
     - \sum_{\bm{v}} P\qty\big(\bm{v}) \pdv{F(\bm{v})}{\bm{\theta}}.
\end{align}
其中的参数 $\bm{\theta}$ 包含连接权重 $\bm{W}$、可见层偏差 $\bm{b}$ 与隐藏层偏差 $\bm{c}$。

式~\eqref{eq:rmb-gradient} 共包含两项,分别称为正相(positive phase)和负相(negative phase)。
显然,负相的计算比较困难,这也是 \CDk{} 算法着力需要解决的问题。

\subsection{Gibbs 采样}

计算负相的困难之处在于需要对所有可能的可见层 $\bm{v}$ 分布进行求和,而这一分布对应的状态数量极其
巨大。与我们在处理 Ising 模型的解决方法类似,这里同样可以用一组样本的分布代替整体分布,即 Monte
Carlo 方法:
\begin{equation}
  \sum_{\bm{v}} P(\bm{v}) \pdv{F(\bm{v})}{\bm{\theta}}
  \approx \frac{1}{\abs{\domN}} \sum_{\bm{v}\in\domN} \pdv{F(\bm{v})}{\bm{\theta}}.
\end{equation}
式中的 $\domN$ 表示一组依照概率 $P$ 进行的采样。

由于可见层、隐藏层的层内均没有连接,因此可以在层之间来回“跳转”,并将其视作一条 Markov 链,
如图~\ref{fig:gibbs-sampling} 所示。“跳转”的具体操作可以借由条件概率获得:设给定可见层分布
$\bm{v}$,则隐藏层分布为
\begin{equation}
  \label{eq:hidden-layer-dist}
  \bm{h} \sim P\qty\big(\bm{h}\big|\bm{v});
\end{equation}
由此可计算下一组可见层分布
\begin{equation}
  \label{eq:visible-layer-dist}
  \bm{v}' \sim P\qty\big(\bm{v}\big|\bm{h}).
\end{equation}
式中,符号 $\bm{x}\sim P$ 表示按照概率 $P$ 生成 $\bm{x}$。考虑到 $\bm{v}$ 与 $\bm{h}$ 均为布尔型,
我们可以近似用概率直接代替 $\bm{v}$ 和 $\bm{h}$。

\begin{figure}[htb]
  \centering
  \tikzinput{gibbs-sampling}
  \caption{Gibbs 采样的示意图。上标代表进行的采样次数}
  \label{fig:gibbs-sampling}
\end{figure}

以上步骤可以持续进行,这称为 Gibbs 采样(Gibbs sampling),采样次数即为 \CDk{} 算法中的 $k$。
当 $k\to\infty$ 时,可认为 $\bm{v}^{(k)}$、$\bm{h}^{(k+1)}$
\footnote{这里的上标 $(k)$ 和 $(k+1)$ 是采样次数,而样本指标 $(i)$ 不同。}
已经服从所要求的分布 $P(\bm{v})$。显然,$k\to\infty$ 在计算上是不可取的。实践中,往往选择 $k<15$
以提高计算速度。甚至在大多数情况下,取 $k=1$ 即可满足要求。此时,\eqref{eq:rmb-gradient}~式中的
梯度近似可以表示为
\begin{equation}
  \label{eq:rmb-gradient-approx}
  - \pdv{\bm{\theta}} \ln P\qty\big(\bm{v}^{(i)})
  \approx \pdv{F\qty\big(\bm{v}^{(i)})}{\bm{\theta}}
        - \pdv{F\qty\big({\bm{v}'}^{(i)})}{\bm{\theta}}.
\end{equation}
其中的 ${\bm{v}'}^{(i)}$ 表示经过 Gibbs 采样生成的可见层单元。

\subsection{梯度的计算}

首先来计算式~\eqref{eq:hidden-layer-dist}、\eqref{eq:visible-layer-dist} 中出现的条件概率
$P(\bm{h}|\bm{v})$。利用乘法公式,有
\begin{align}
  P(\bm{h}|\bm{v})
  &= \frac{P(\bm{v},\,\bm{h})}{P(\bm{v})}
   = \frac{\ee^{-E(\bm{v},\,\bm{h})}}%
          {\sum_{\tilde{\bm{h}}} \ee^{-E(\bm{v},\,\tilde{\bm{h}})}} \notag \\
  &= \frac{\exp\qty\big(\bm{v}^\trans\bm{W}\bm{h} + \bm{b}^\trans\bm{v} + \bm{c}^\trans\bm{h})}%
          {\sum_{\tilde{\bm{h}}} \exp\qty\big(\bm{v}^\trans\bm{W}\tilde{\bm{h}}
           + \bm{b}^\trans\bm{v} + \bm{c}^\trans\tilde{\bm{h}})}
   = \frac{\exp\qty\big[\qty\big(\bm{v}^\trans\bm{W} + \bm{c}^\trans)\bm{h}\,]}%
          {\sum_{\tilde{\bm{h}}} \exp\qty\big[\qty\big(\bm{v}^\trans\bm{W} + \bm{c}^\trans)
                                              \tilde{\bm{h}}\,]} \notag \\
  &= \frac{\exp\qty\big[\,\sum_j h_j \qty\big(c_j + \sum_i W_{ij}v_i)]}%
          {\sum_{\q{\tilde{h}_i}} \exp\qty\big[\,\sum_j \tilde{h}_j
                                               \qty\big(c_j + \sum_i W_{ij}v_i)]} \notag \\
  &= \prod_j \frac{\exp\qty\big[h_j \qty\big(c_j + \sum_i W_{ij}v_i)]}%
                  {\sum_{\q{\tilde{h}_i}} \exp\qty\big[\tilde{h}_j
                                               \qty\big(c_j + \sum_i W_{ij}v_i)]}
   = \prod_j P(h_j|\bm{v}).
\end{align}
考虑到 $h_j$ 只能取 0 或 1,又有
\begin{align}
  P(h_j|\bm{v})
   = P(h_j=1|\bm{v})
  &= \frac{\exp 1 \cdot \qty\big(c_j + \sum_i W_{ij}v_i)}%
          {\exp 0 \cdot \qty\big(c_j + \sum_i W_{ij}v_i)
           + 1 \cdot \qty\big(c_j + \sum_i W_{ij}v_i)} \notag \\
  &= \frac{\exp\qty\big(c_j + \sum_i W_{ij}v_i)}{1 + \exp\qty\big(c_j + \sum_i W_{ij}v_i)}
   = \sigma\qty\big(c_j + \sum_i W_{ij}v_i).
\end{align}
同理,
\begin{equation}
  P(\bm{v}|\bm{h}) = \prod_i P(v_i=1|\bm{h}) = \prod_i \sigma\qty\big(b_i + \sum_j W_{ij}h_j).
\end{equation}
利用概率近似代替生成的 $\bm{h}$ 和 $\bm{v}'$,我们有
\begin{equation}
  \label{eq:h-v-approx}
  h_j  \approx \sigma\qty\big(c_j + \sum_i W_{ij}v_i) \qc
  v'_i \approx \sigma\qty\big(b_i + \sum_j W_{ij}h_j).
\end{equation}

接下来计算式~\eqref{eq:rmb-gradient-approx} 中自由能的偏导数。代入 Boltzmann 机的能量函数
\eqref{eq:rbm-energy}~式,可以把自由能写成
\begin{align}
  F(\bm{v})
  &= - \ln \sum_{\q{h_i}}
       \exp\qty\Big(\sum_{i,\,j} W_{ij} v_i h_j + \sum_i b_i v_i + \sum_j c_j h_j) \notag \\
  &= - \sum_i b_i v_i
     - \ln \sum_{\q{h_i}} \exp\qty\bigg[\sum_j h_j \qty\Big(c_j + \sum_i W_{ij} v_i)] \notag \\
  &= - \sum_i b_i v_i
     - \ln \prod_j \sum_{h_j} \exp\qty\bigg[h_j \qty\Big(c_j + \sum_i W_{ij} v_i)] \notag \\
  &= - \sum_i b_i v_i
     - \sum_j \ln \sum_{h_j} \ee^{h_j \qty\big(c_j + \sum_i W_{ij} v_i)}.
\end{align}
利用 $h_j$ 的布尔特性,可以将上式化简:
\begin{align}
  F(\bm{v})
  &= - \sum_i b_i v_i - \sum_j \ln
       \qty\bigg[  \ee^{0 \cdot \qty\big(c_j + \sum_i W_{ij} v_i)}
                 + \ee^{1 \cdot \qty\big(c_j + \sum_i W_{ij} v_i)}] \notag \\
  &= - \sum_i b_i v_i - \sum_j \ln \qty\Big(1 + \ee^{c_j + \sum_i W_{ij} v_i}).
\end{align}
此时可以计算出自由能关于各参量的偏导数:
\begin{align}
  %TODO: subequation
  \pdv{F(\bm{v})}{W_{ij}} &= - \pdv{W_{ij}} \ln \qty\Big(1+\ee^{c_j+\sum_iW_{ij}v_i})
                           = - v_i \cdot \frac{\ee^{c_j+W_{ij}v_i}}{1+\ee^{c_j+W_{ij}v_i}}
                           = - v_i \cdot \sigma\qty\Big(c_j+\sum_iW_{ij}v_i); \\
  \pdv{F(\bm{v})}{b_i}    &= - v_i; \\
  \pdv{F(\bm{v})}{c_j}    &= - \pdv{c_j} \ln \qty\Big(1+\ee^{c_j+\sum_iW_{ij}v_i})
                           = - \frac{\ee^{c_j+\sum_iW_{ij}v_i}}{1+\ee^{c_j+\sum_iW_{ij}v_i}}
                           = - \sigma\qty\Big(c_j+\sum_iW_{ij}v_i).
\end{align}
代入式~\eqref{eq:h-v-approx},并改写成矩阵形式:
\begin{equation}
  \pdv{F(\bm{v})}{\bm{W}} = - \bm{v} \bm{h}^\trans \qc
  \pdv{F(\bm{v})}{\bm{b}} = - \bm{v} \qc
  \pdv{F(\bm{v})}{\bm{c}} = - \bm{h}.
\end{equation}
此即式~\eqref{eq:rmb-gradient-approx} 中的正相部分。

通过 Gibbs 采样,负相部分的形式与正相部分完全一致,只需把 $\bm{v}$、$\bm{h}$ 换成 $\bm{v}'$、
$\bm{h}'$,因此
\begin{equation}
  \pdv{F(\bm{v}')}{\bm{W}} = - \bm{v}' {\bm{h}'}^\trans \qc
  \pdv{F(\bm{v}')}{\bm{b}} = - \bm{v}' \qc
  \pdv{F(\bm{v}')}{\bm{c}} = - \bm{h}'.
\end{equation}
至此,我们便得到了梯度下降法所需要的参数更新公式:
\begin{equation}
  \left\{
  \begin{aligned}
    \bm{W} \! &\gets \bm{W} \! - \alpha \qty\big(\bm{v} \bm{h}^\trans
                                                 - \bm{v}' {\bm{h}'}^\trans) \qc \\
    \bm{b} \, &\gets \, \bm{b} - \alpha \qty\big(\bm{v} - \bm{v}') \qc \\
    \bm{c} \, &\gets \, \bm{c} - \alpha \qty\big(\bm{h} - \bm{h}').
  \end{aligned}
  \right.
\end{equation}
式中的 $\alpha$ 为学习率。

\section{RBM 的应用:手写数字}

本节我们将利用 MNIST 数据集训练 RBM,并基于已有数据生成新的数据,即让计算机学会“书写”数字。

MNIST 数据集 \num{60000} 张灰度图片组成,每张图片表示一个数字,其大小被统一为 $28 \times 28$ 像素,
并保证数字处于居中位置。RBM 的训练需要使用布尔型数据,我们也据此将图片进行了二值化处理,
如图~\ref{fig:mnist} 所示。

\begin{figure}[htb]
  \centering
  \begin{subfigure}[b]{0.35\textwidth}
    \centering
    \tikzinput{mnist-i}
    \caption{原始图像,灰度值包括 0--255}
  \end{subfigure}
  \begin{subfigure}[b]{0.35\textwidth}
    \centering
    \tikzinput{mnist-ii}
    \caption{二值化后的图像,只包含值 0 和 1}
  \end{subfigure}
  \caption{MNIST 数据集中的首张图片,这是一个数字 5。图中白色区域始终代表 0 值}
  \label{fig:mnist}
\end{figure}

具体的训练数据如下:

\begin{enumerate}
  \item 可见层单元数量:784($=28 \times 28$),即把图片压平为一维向量;
  \item 隐藏层单元数量:100;
  \item 批次(batch)大小:64,表示每次同时计算的样本数量;
  \item 训练轮次(epoch):20,一个轮次表示将整个数据集训练一遍;
  \item \CDk{} 中的参数 $k=1$;
  \item 学习率 $\alpha=0.1$。
\end{enumerate}

\section{RBM 与重整化群的对应关系}

\section{利用 RBM 研究 Ising 模型}
\section{卷积限制 Boltzmann 机(CRBM)}
\section{卷积层、小波变换与 \AdSCFT{}}
\section{利用 CRBM 研究 Ising 模型}
