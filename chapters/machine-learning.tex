\chapter{机器学习简介}

\section{机器学习有关思想方法}

机器学习是当今计算机科学的重要分支。它可以利用计算机的强大计算能力,从大量数据中提取特征,并构建
相应的模型。生成的模型还可以用来进行预测。

考虑最简单的回归模型。假设我们给定了一些数据点 $(\bm{x}^{(i)}, \, y^{(i)})$,其中 $\bm{x}^{(i)}$
称为特征(feature),$y^{(i)}\in\R$ 称为标签(label)。训练样本的数量为 $m$,即
$i=1,\,2,\,\ldots,\,m$。特征的数量为 $n$,但按照惯例,取 $x_0=0$,故有 $\bm{x}^{(i)}\in\R^{n+1}$。
一般而言,特征反映了样本的各种性质(如图片中的像素点),而标签则是需要需要预测的内容(如图片对应的
物体类别)。为了学习样本数据,我们需要给出一个目标函数 $h_\bm{\theta}: \R^{n+1}\to\R$,使得
$h_\bm{\theta}(\hat{\bm{x}})=\hat{y}$。这里的 $\bm{\theta}$ 称为参数(parameter),它包含了模型的
全部训练信息。

显然,如果要求学习的效果越好,就要求预测值 $\hat{y}$ 与真实值 $y$ 的差距越小。因此我们需要定义
损失函数(loss function) $L$,并使 $L$ 关于参数 $\bm{\theta}$ 取最小值。如果数据规模较小、模型比较
简单,$L$ 的最小值可以解析求得。例如线性回归模型,即
\begin{equation}
  h_\bm{\theta} (\bm{x})
  = \theta_0 + \theta_1 x_1 + \theta_2 x_2 + \cdots + \theta_n x_n
  = \sum_{j=0}^n \theta_j x_j,
\end{equation}
我们取损失函数为
\begin{equation}
  \label{eq:loss-function-linear}
  L(\bm{\theta})
  = \frac{1}{2} \sum_{i=1}^m \qty\Big[y^{(i)} - h_\bm{\theta}\qty\big(\bm{x}^{(i)})]^2
  = \frac{1}{2} \norm{\bm{y} - \bm{X}\bm{\theta}}^2 \in \R.
\end{equation}
式中 $\bm{X}\in\R^{m\times(n+1)}$,且有 $X_{ij} = x^{(i)}_j$。利用最小二乘估计,可以求得最优解
\begin{equation}
  \hat{\bm{\theta}} = \qty\big(\bm{X}^\trans\bm{X})^{-1} \bm{X}^\trans \bm{y}.
\end{equation}

然而,当数据规模很大时,可以发现矩阵 $\bm{X}^\trans\bm{X}$ 的阶数也会很大,导致求其逆阵的开销无法
接受。因此可以改换思路,利用梯度下降法逼近损失函数的最小值。对于一般的多元函数,梯度给出了其在某一点
处变化率最大的方向。因而沿着梯度方向不断改变 $\bm{\theta}$ 的值,就可以逐步趋近最小值。我们可以令
\begin{equation}
  \bm{\theta} \coloneq \bm{\theta} - \alpha\nabla\bm{\theta},
\end{equation}
并不断循环,直至收敛。式中 $\alpha$ 称为学习率,用来调节每次下降的“步幅”。

在回归模型中,我们对每一组数据都给出了对应的标签。这称之为监督学习(supervised learning)。与之
相对应,另外还有无监督学习(unsupervised learning),指的是不需要为数据提供标签。无监督学习要求机器
从数据本身获取更加深刻的信息,因此相比监督学习更为困难。

除了回归问题(物理学中一般称为拟合),监督学习还包括分类等。无监督学习主要包括聚类、降维等。

一般来说,机器学习的思路可以概括如下:

\begin{enumerate}
  \item 根据数据的具体类型、特性构建相应的模型;
  \item 给出模型的损失函数;
  \item 在数据集上进行训练,找出能使损失函数取最小值的参数;
  \item 验证模型及参数的有效性,进行参数微调。
  \item 在新的数据集上测试训练好的模型。
\end{enumerate}

\section{Ising 模型相变研究(一):线性模型}

在给定的温度下,我们可以利用 Monte Carlo 算法生成一系列 Ising 模型的自旋构型( configuration )。
不妨把构型作为前文所述的特征 $\bm{x}^{(i)}$,把温度作为标签 $y^{(i)}$。对大量数据样本进行训练后,
给定某一自旋构型,即可以利用机器学习的手段获知其所对应的温度。对于二维正方晶格的 Ising 模型,
其相变临界点是已知的,由此还可以知道该自旋构型处于有序态(铁磁)或是无序态(顺磁)。

\subsection{训练模型}

下面我们利用线性分类模型分析由 Monte Carlo 算法生成的 Ising 自旋构型。Ising 自旋构型是一个
$N\times N$ 的矩阵,每一个矩阵元仅可取 1 或 0。这称为布尔型数据(binary)。为了计算方便,接下来把
这一矩阵压平为一位向量,即 $\bm{x}^{(i)}\in\R^{N^2}$。对于标签 $y^{(i)}$,我们同样进行二值化,即有
\begin{equation}
  \begin{cases}
    0, & T <         \Tc; \\
    1, & T \geqslant \Tc.
  \end{cases}
\end{equation}

如图~\ref{fig:linear-layer},二值化的线性分类模型可以用下式表示:
\begin{equation}
  y = \sigma(\bm{W}^\trans \bm{x} + b).
\end{equation}
式中,$\bm{W}\in\R^{N^2}$ 称为权重(weight),$b\in\R$ 称为偏差(bias)。注意到前文的
$\bm{\theta}$ 相当于这里 $\bm{W}$ 和 $b$ 的组合,并有 $b=\theta_0$。$\sigma$ 称为激活函数
(activation function),其作用是将线性模型给出的结果映照到 $\q{0,\,1}$ 上。为了计算的方便,常把
激活函数取为逻辑函数(logistics sigmoid function),它满足
\begin{equation}
  \sigma(z) = \frac{1}{1+\ee^{-z}}.
\end{equation}
注意到 $\sigma$ 的值域为连续区间 $(0,\,1)$,我们把这一结果结果理解为 $y$ 属于类别 1 的概率
$P(y=1)$。

\begin{figure}[htb]
  \centering
  \imageinput{linear-layer}
  \caption{线性分类模型示意图}
  \label{fig:linear-layer}
\end{figure}

\subsection{训练结果}

\section{神经网络}

\subsection{网络结构与损失函数}

对于线性可分的问题(即可以用数据空间中的单一平面划分不同类别),简单的线性模型就能够取得较好的
训练结果。但实际问题往往要复杂得多,且单纯增加参数的数量并不能显著改善结果。上一节中我们所研究的
Ising 模型相的分类问题,就显示出了线性模型的局限性。

一种比较好的解决方案是将多层线性模型组合起来。在每一个线性层之后,都额外作用激活函数。激活函数
具有非线性效应,因此多层线性模型的组合并不等同于参数更多的单个线性层。

在如图~\ref{fig:neural-net} 这样的网络中,每一个节点都可以类比为生物神经网络中的神经元(neuron),
因此这种网络称为人工神经网络(artificial neural network)。神经元能够从“树突”(dendrite)接受上一层
的信息,经过简单处理后,又从“轴突”(axon)将信息传递给下一层。一个线性层神经元和一个激活函数神经元
$\sigma$ 组成了一个感知机(perceptron)。与线性模型类似,每个感知机均有两组参数:连接权重 $\bm{W}$
和偏差 $b$。神经网络的训练,就是对每一个感知机找到合适的 $\bm{W}$ 与 $b$,使得损失函数尽可能小。

\begin{figure}[htb]
  \centering
  \imageinput{neural-net}
  \caption{具有两层的神经网络。注意实际训练中所用的网络,其隐藏层单元数目未必大于训练数据单元数目}
  \label{fig:neural-net}
\end{figure}

类比式~\eqref{eq:loss-function-linear},神经网络的损失函数同样可以用均方差表示。设数据样本的特征为
$\bm{x}^{(i)}$,标签为 $y^{(i)}$,则损失函数
\begin{equation}
  L = \frac{1}{2} \sum_{i=1}^m \qty\big[y^{(i)} - \hat{y}^{(i)}]^2,
\end{equation}
其中 $\hat{y}^{(i)}$ 是神经网络对特征 $\bm{x}^{(i)}$ 给出的预测值。损失函数的另一种选择是交叉熵
(cross-entropy)。交叉熵刻画了两个概率分布之间的相关性。对于离散概率分布 $f$ 和 $g$,其定义为
\begin{equation}
  H(p,\,q) = - \sum_x p(x) \ln q(x).
\end{equation}
对于布尔型数据,$x$ 只能取 0 或 1,因此可以给出损失函数
\begin{equation}
  L = - \frac{1}{m} \sum_{i=1}^m
        \qty\Big[y^{(i)}\ln\hat{y}^{(i)} - \qty\big(1-y^{(i)})\ln\qty\big(1-y^{(i)})]^2.
\end{equation}

神经网络可以利用反向传播算法(back-propagation)训练,它同样是一种梯度下降策略。具体来说,首先
利用初始化后的权重和偏差给出预测值,之后自顶向下(对应到图~\ref{fig:neural-net} 中为从右向左)逐层
计算梯度,并据此更新权重和偏差。以此类推,重复训练直至达到稳定。

常用的机器学习框架,如 TensorFlow、PyTorch 等都会提供反向传播算法的高效实现,此处不再赘述。

\subsection{正则化}
\label{subsec:regularization}

神经网络中参数数目极其巨大,但实际问题的刻画往往并不需要如此多的参数。因此,神经网络的训练常会出现
过拟合(overfitting)的问题,使得数据样本中的冗余信息也被神经网络所习得。过拟合的后果往往是模型在
训练数据集上表现良好,但缺乏泛化能力,一旦在其他数据集上测试时效果很差。

使用大规模数据集在一定程度上可以缓解过拟合。但在实际情况下,大规模的数据样本不易获得。我们的解决
手段是在损失函数中引入正则化(regularization)项。一种常用正则化方案是权重衰减(weight decay),
或 $L_2$ 正则化。它在损失函数中引入额外一项
\begin{equation}
  \frac{\lambda}{2m} \sum_k \norm{\bm{\theta}_k}^2,
\end{equation}
式中 $\lambda>0$ 称为正则化参数,求和对所有参数进行,而 $L_2$ 范数 $\norm{\cdot}$ 定义为
\begin{equation}
  \norm{\bm{\theta}} = \sqrt{\sum_i \theta_i^2}.
\end{equation}
正则化项的存在可以理解为一种“惩罚”机制。当参数中存在大量非零值,则正则化项会较大,进而使得损失函数
也随之增大。因此,正则化可以使得参数中的绝大部分变为零(成为稀疏矩阵),降低参数自由度,从而缓解
过拟合问题。

\section{Ising 模型相变研究(二):多层神经网络}
