\chapter{结论}

本文在经典 Ising 模型的基础上,引入了现代机器学习手段对其进行了一些研究,主要包含两个方面:

一、
利用监督学习对 Ising 模型的有序相和无序相进行分类,这里采用的主要是线性分类模型,以及由多个线性层
组合而成的神经网络。我们发现,最简单的线性分类模型就一个可以给出比较准确的定性结果;而利用神经网络,
还可以计算得到一些定量结果,如临界温度 $\Tc$。

二、
探讨 RBM 等基于能量的模型与重整化群的联系。RBM 是生成型神经网络的代表,以 MNIST 手写数字数据集为例,
我们展示了 RBM 的学习能力。类似地,将其用来学习 Ising 模型的点阵数据,给出的训练结果也与粗粒近似
的思想相吻合。鉴于 RBM 无法给出定量结果,也就难以开展进一步的研究,我们还引入了卷积层,试图从理论
上建立起与 \AdSCFT{} 之间的对应关系。

相比传统解析计算与数值模拟技术,机器学习无疑打开了一扇新的窗口。现在已经发现,机器学习与重整化群、
张量网络、量子信息及量子计算均有着密切联系。如何利用机器学习探索新的物理图像,仍需要我们进一步的
研究。
