Machine learning is undoubtedly one of the most rapidly developing fields in computer science.
With the deepening of research, machine learning has attracted more and more physicists' attention.
In addition to employing the huge power of machines for data processing, it is more essential and
significant to explore the underlying physical background in machine learning.

As the most fundamental model in statistical physics, Ising model has rather rich physical
connotation. The existence of exact solution of 2D Ising model can provide accurate testing
standards for various studies. Ising model plays an important role in conformal field theory,
which can be associated with \AdSCFT{}, another hot research fields in theoretical physics.
Furthermore, wide-used numerical simulation techniques, such as Monte Carlo algorithm, help people
acquire large amount of data from Ising model in a short time. Therefore, it is possible to obtain
the corresponding physical results via machine learning.

In this paper, we will take Ising model as the main research object, and introduce a variety of
machine learning methods based on it, so as to explore the relevant physical images.

In chapter 1, we will briefly summarize the statistical physics of Ising model and discuss its
critical phenomenon based on renormalization group. The idea of Monte Carlo algorithm and some
simulation results will be given here.

Chapter 2 is used for introducing the basic ideas and methods of machine learning. We will use
linear regression and multi-layer artificial neural network to study the data of Ising model spin
lattice, in order to demonstrate their ability of classifying each phase.

Chapter 3 is the focus of this paper. First, we will show the network structure of the restricted
Boltzmann machine (RBM), derive the contrastive divergence algorithm, and train an RBM on the MNIST
handwritten digit database. Second, we will explain the exact mapping between RBM and
renormalization group, and discuss the training results on Ising model data. Finally, the
convolution layer will be introduced, and its connection between wavelet transform and \AdSCFT{}
will be considered. We will show the main idea of applying convolution layer to RBM to obtain
quantitative results as well.

The main conclusions of this paper are given in chapter 4.
