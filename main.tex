\PassOptionsToPackage{log-declarations=false}{xparse}
%\PassOptionsToPackage{showframe}{geometry}

\documentclass[oneside]{fduthesis}
\usepackage{physics,siunitx,enumitem,subcaption}

\fdusetup{
  style={
    font=none,
    cjk-font=founder,
    logo={fudan-name.pdf, fudan-emblem-new.pdf},
    footnote-style=xits,
    hyperlink-color=material,
    bib-backend=bibtex,
    bib-resource={reference.bib}
  },
  info={
    title={机器学习与 Ising 模型},
    department={物理系},
    major={物理学},
    author={曾祥东},
    student-id={14307110142},
    supervisor={孔令欣},
    supervisor-title={研究员},
    affiliation={复旦大学物理系},
    keywords={机器学习, Ising 模型, 神经网络, RBM, 卷积, \AdSCFT{}},
    keywords*={Machine learning, Ising model, Neural network, RBM, Convolution, \AdSCFT{}}
  }
}

% Fonts
\setmainfont{XITS}[%
  UprightFeatures={SmallCapsFont=*},
  BoldFeatures={SmallCapsFont=* Bold},
  ItalicFeatures={SmallCapsFont=* Italic},
  BoldItalicFeatures={SmallCapsFont=* Bold Italic}]
\setsansfont{TeX Gyre Heros}
\setmonofont{TeX Gyre Cursor}[Ligatures=CommonOff]
\setmathfont{XITS Math}[math-style=ISO, bold-style=ISO, StylisticSet=1]
\ExplSyntaxOn
\__fdu_set_math_font:nn
  { XITS~ Math } { math-style = ISO, bold-style = ISO, version = script }
\cs_set:Npn \mathscript #1 { \mbox { \mathversion { script } #1 } }
\ExplSyntaxOff
% CJK fonts
%\setCJKmainfont{FZGFShuSong}[%
%  BoldFont=FZXiaoBiaoSong-B05,ItalicFont=FZKai-Z03, BoldItalicFont=FZKai-Z03]
%\setCJKmainfont{Source Han Serif SC}[ItalicFont=FZKai-Z03, BoldItalicFont=FZKai-Z03]
%\setCJKsansfont{Source Han Sans SC Medium}[%
%  BoldFont=Source Han Sans SC Bold, ItalicFont=*, BoldItalicFont=*]
%\setCJKmonofont{FZFangSong-Z02}
%\setCJKfamilyfont{kai}{FZKai-Z03}
% https://tex.stackexchange.com/a/11916/136923
% Will break spaces between CJK characters.
%\def\CJKmovesymbol#1{\raise-0.08ex\hbox{#1}}
%\let\CJKsymbol\CJKmovesymbol
%\let\CJKpunctsymbol\CJKmovesymbol

\setlist[enumerate]{itemsep=0pt, parsep=\parskip, topsep=0pt}
\setlist[itemize]{itemsep=0pt, parsep=\parskip, topsep=0pt}

% Constants
\def\ee{{\symup{e}}}
\def\ii{{\symup{i}}}
\def\pp{{\symup{\pi}}}
\def\kB{k_{\symup{B}}}
\def\Tc{T_{\symup{c}}}
% Operators
\def\incr{{\symup{\Delta}}}
\def\trans{{\symup{T}}}
\def\opT{{\symbfup{T}}}
\def\opT{{\symbfup{T}}}
\def\opO{{\symcal{O}}}
\def\bigO{\mathscript{$\symscr{O}$}}
\def\likeL{{\symcal{L}}}
\newcommand\trR[2][]{\mathscript{$\symbfscr{R}^{\,#1}_{#2}$}}
\newcommand\trA[2][]{\mathscript{$\symbfscr{A}^{\,#1}_{#2}$}}
% Set
\def\R{{\symbb{R}}}
\def\Z{{\symbb{Z}}}
\def\domD{{\symcal{D}}}
\def\domN{{\symcal{N}}}
\def\spaceV{\mathscript{$\symscr{V}$}}
% Decorations
\def\bm#1{{\symbf{#1}}}
\def\nearest#1{\langle#1\rangle}
\def\q#1{{\{#1\}}}
% Text
\def\const{\text{const}}
% Math miscellaneous
\allowdisplaybreaks

\newcommand\imageinput[2][]{\includegraphics[#1]{figures/#2}}
\def\code#1{\texttt{#1}}

\def\CDk{CD\raisebox{0.13ex}{-}$k$}
\def\AdSCFT{AdS/CFT}

\begin{document}

\frontmatter
\tableofcontents
\begin{abstract}
  机器学习在计算机科学界无疑是发展最为迅速的领域之一。随着研究的深入,机器学习在物理学中的应用也逐渐
为人们所重视。然而,除了利用机器的强大计算能力进行数据处理,深入探讨机器学习中隐含的物理背景,
显然更为本质和重要。

作为统计物理中最简单的模型之一,Ising 模型具有十分丰富的内涵。二维严格解的存在,则为各种研究提供了
准确的检验标准。Ising 模型在共形场论中的重要地位,又可以与 \AdSCFT{} 相联系,从而得到更深层次的
物理。另一方面,成熟的数值模拟技术(如 Monte Carlo 等),为提供 Ising 模型的大量数据做了保障。据此,
在大规模训练数据集的基础上,运用机器学习手段获得相应的物理结果,也就成为了可能。

本文我们将以 Ising 模型作为主要研究对象,在其基础上引入多种机器学习手段,进而探讨有关的物理图像。

第 1 章里,我们将简要概述 Ising 模型的统计物理,并基于重整化群讨论其临界现象。我们还将简单介绍
Monte Carlo 算法的原理及有关模拟结果。

在第 2 章,我们将介绍机器学习的基本思想方法。作为例子,我们分别利用线性回归模型和多层人工神经网络
对 Ising 模型数据进行学习,展示其对各相进行分类的能力。

第 3 章是本文的重点。首先我们将给出限制 Boltzmann 机的网络结构,推导对比散度算法,并在 MNIST
手写数字数据集上对其进行训练。接着,我们将探讨限制 Boltzmann 机与重整化群的对应关系,并对 Ising
模型数据上的训练结果予以讨论。最后引入卷积层,考虑卷积与小波变换和 \AdSCFT{} 之间的联系,从理论上
分析利用卷积限制 Boltzmann 机获取定量结果的可能性。

第 4 章中将给出本文的主要结论。

\end{abstract}
\begin{abstract*}
  Machine learning is undoubtedly one of the most rapidly developing fields in computer science.
With the deepening of research, machine learning has attracted more and more physicists' attention.
In addition to employing the huge power of machines for data processing, it is more essential and
significant to explore the underlying physical background in machine learning.

As the most fundamental model in statistical physics, Ising model has rather rich physical
connotation. The existence of exact solution of 2D Ising model can provide accurate testing
standards for various studies. Ising model plays an important role in conformal field theory,
which can be associated with \AdSCFT{}, another hot research fields in theoretical physics.
Furthermore, wide-used numerical simulation techniques, such as Monte Carlo algorithm, help people
acquire large amount of data from Ising model in a short time. Therefore, it is possible to obtain
the corresponding physical results via machine learning.

In this paper, we will take Ising model as the main research object, and introduce a variety of
machine learning methods based on it, so as to explore the relevant physical images.

In chapter 1, we will briefly summarize the statistical physics of Ising model and discuss its
critical phenomenon based on renormalization group. The idea of Monte Carlo algorithm and some
simulation results will be given here.

Chapter 2 is used for introducing the basic ideas and methods of machine learning. We will use
linear regression and multi-layer artificial neural network to study the data of Ising model spin
lattice, in order to demonstrate their ability of classifying each phase.

Chapter 3 is the focus of this paper. First, we will show the network structure of the restricted
Boltzmann machine (RBM), derive the contrastive divergence algorithm, and train an RBM on the MNIST
handwritten digit database. Second, we will explain the exact mapping between RBM and
renormalization group, and discuss the training results on Ising model data. Finally, the
convolution layer will be introduced, and its connection between wavelet transform and \AdSCFT{}
will be considered. We will show the main idea of applying convolution layer to RBM to obtain
quantitative results as well.

The main conclusions of this paper will be given in chapter 4.

\end{abstract*}

\mainmatter
\chapter{Ising 模型}

\section{Ising 模型的统计物理}
\label{sec:ising-physics}

Ising 模型是统计物理中描述铁磁系统的一种简单模型。Ising 模型由一系列自旋排列而成的点阵构成,其中的每
个自旋只有两种取向,即 $+1$ 或 $-1$。自旋与自旋之间的相互作用可以导致相变的出现,最典型的例子是二维
正方晶格的 Ising 模型,其严格解由 Onsager 给出。

Ising 模型的 Hamilton 量由下式给出:
\begin{equation}
  H(\q{\sigma_i}) = -\sum_{\nearest{ij}} J_{ij}\sigma_i\sigma_j - \mu \sum_{i} B_i\sigma_i.
\end{equation}
式中,“$\nearest{ij}$” 表示相邻的自旋。$J_{ij}$ 是耦合常数,通常情况下与自旋位置无关,即
$J_{ij}=J$,因而可以放在求和号外部。$J>0$ 时,代表体系处于铁磁态;$J<0$ 时,则代表处于反铁磁态。
$B_i$ 是外磁场,对于匀强磁场,有 $B_i=B$,即同样与位置无关。$\mu$ 是对应的磁矩。由于体系关于
$\sigma_i$ 的正负是对称的,方便起见不妨认为 $B \geqslant 0$。此时,我们有
\begin{equation}
  H(\q{\sigma_i}) = -J \sum_{\nearest{ij}} \sigma_i\sigma_j - \mu B \sum_{i} \sigma_i.
\end{equation}

我们可以为某一特定的自旋构型引入概率。温度 $T$ 时,构型 $\q{\sigma_i}$ 在平衡态中出现的概率由
Boltzmann 分布给出:
\begin{equation}
  \label{eq:ising-probability}
  P(\q{\sigma_i}) = \frac{\ee^{-\beta H(\q{\sigma_i})}}{Z_N}.
\end{equation}
其中的 $\beta=1/T$。而 $Z_N$ 称为配分函数,它使得概率 $P$ 满足归一化条件
\begin{equation}
  \sum_\q{\sigma_i} P(\q{\sigma_i}) = 1
  \implies Z_N = \sum_\q{\sigma_i} \ee^{-\beta H(\q{\sigma_i})}.
\end{equation}
配分函数等价于体系的自由能。在温度 $T$、外磁场 $B$ 作为变量的情况下,自由能是体系的特性函数,故可以
通过自由能计算出各物理量,包括内能、热容、磁化强度等。因此求解 Ising 模型实际上就是求出体系的配分函
数(自由能)。

一维 Ising 模型的严格解可利用转移矩阵的手段求出。热力学极限下(自旋数目 $N\to\infty$),自由能为
\begin{equation}
  F = - \frac{\ln Z_N}{\beta}
    = - N J - \frac{N}{\beta}
              \ln\qty[\cosh(\beta\mu B) + \sqrt{\ee^{-4\beta J} + \sinh^2(\beta\mu B)}].
\end{equation}
磁化强度为
\begin{equation}
  \frac{\bar{M}}{N\mu} = \frac{\sinh{\beta\mu B}}{\sqrt{\ee^{-4\beta J} + \sinh^2(\beta\mu B)}}.
\end{equation}
显然,只要 $\beta<\infty$,当 $B\to 0$ 时,$\bar{M}\to 0$;而当 $\beta\to 0$ 时,$\bar{M}\to N\mu$。
这说明一维 Ising 模型的临界温度 $\Tc=0$,事实上不存在有限温度下的相变。

二维 Ising 模型的严格解比较复杂。对于正方晶格,可利用对偶变换求出临界温度
\begin{equation}
  \frac{\Tc}{J} = \frac{2}{\ln(1+\sqrt{2})} \approx \num{2.2692}.
\end{equation}
外场为零时,可以求得(单位自旋的)自由能为
\begin{equation}
  \frac{\beta F}{N}
  = - \frac{\ln Z_N}{N}
  = - \frac{\ln 2}{2} - \ln\cosh 2\beta J
    - \frac{1}{2\pp} \int_0^\pp \ln\qty[1+\sqrt{1-\kappa^2\sin^2\theta}] \dd{\theta},
\end{equation}
其中
\begin{equation}
  \kappa = \frac{2\sinh 2\beta J}{\cosh^2 2\beta J}.
\end{equation}
进而可以求得热容:
\begin{align}
  \label{eq:ising-specific-heat}
  \frac{C}{N}
  & \simeq -\frac{2}{\pp} \ln^2 \qty(1+\sqrt{2}) \ln\abs{1-\frac{T}{\Tc}} + \const \notag \\
  & \simeq \num{-0.4945} \ln\abs{1-\frac{T}{\Tc}} + \const.
\end{align}
可以看到热容在临界温度处呈现出对数发散行为。

磁化强度的计算结果由杨振宁首次给出:
\begin{equation}
  \label{eq:ising-magnetization}
  \frac{\bar{M}}{N\mu} =
  \begin{cases}
    \qty(1 - \sinh^{-4} 2\beta J)^{1/8}, & T<\Tc; \\
    0, & T>\Tc.
  \end{cases}
\end{equation}

\section{临界指数与标度变换}

一般而言,在接近临界点处,热力学量 $f$ 可按约化温度 $t=T/\Tc-1$ 展开:
\begin{equation}
  f(t) = A t^\lambda \qty\big(1 + B t^y + \cdots),
\end{equation}
式中 $y>0$。定义临界指数 $\lambda$ 为
\begin{equation}
  \lambda = \lim_{t \to 0} \frac{\ln f(t)}{\ln t}.
\end{equation}
它刻画了热力学量 $f$ 的临界行为:
\begin{equation}
  \lim_{t \to 0} f(t) =
  \begin{cases}
    0,      & t > 0; \\
    \const, & t = 0; \\
    \infty, & t < 0.
  \end{cases}
\end{equation}
注意当 $f(t) \sim \ln t$ 时,同样有
\begin{equation}
  \lambda =    \lim_{t \to 0} \frac{\ln f(t)}{\ln t}
          \sim \lim_{t \to 0} \frac{\ln \ln t}{\ln t} = 0.
\end{equation}

取 $f$ 为热容 $C$,则定义了临界指数 $\alpha$。因此对于 Ising 模型,根据
式~\eqref{eq:ising-specific-heat},可有
\begin{equation}
  \alpha = 0.
\end{equation}
取 $f$ 为序参量,即磁化强度 $\bar{M}$,则定义了临界指数 $\beta$。根据
式~\eqref{eq:ising-magnetization},有
\begin{equation}
  \beta = \frac{1}{8}.
\end{equation}

临界指数的存在具有普适性。对于不同的系统,如铁磁系统、van der Waals 气液相变、超导体系等,均可以
得到完全相同的临界指数。这种普适性的根源是关联长度 $\xi$ 在临界点处的发散性为。在临界点处,
$\xi\to\infty$,其他的特征长度则保持不变,因此热力学量的奇异性仅依赖于 $\xi$ 的奇异性,而与体系的
具体特征无关。

以铁磁系统为例,设单位自旋的自由能 $f$ 是约化温度、约化磁场
\begin{equation}
  t = \frac{T}{\Tc} - 1 \qc
  b = \frac{\mu B}{\Tc}
\end{equation}
的函数。将其分解为奇异部分 $f_{\symup{s}}$ 和非奇异部分 $f_{\symup{r}}$:
\begin{equation}
  f(t,\,b) = f_{\symup{s}}(t,\,b) + f_{\symup{r}}(t,\,b).
\end{equation}
在临界点附近,只有奇异部分 $f_{\symup{s}}$ 需要考虑。

标度变换可以视为对晶格的某种“缩放”。经过一次标度变换后,设体系的晶格常数由 $a$ 变成了 $a'=La$。
相应地,有
\begin{equation}
  \label{eq:transform-relation-t-b}
  t' = L^x t \qc b' = L^y b.
\end{equation}
体系总的自由能(配分函数)在标度变换下保持不变,但自旋数目发生了变化:$N \to N'=L^{-d} N$,$d$
为体系的维度。因而有
\begin{equation}
  N f_{\symup{s}} (t,\,b) = N' f_{\symup{s}} (t',\,b'),
\end{equation}
即
\begin{equation}
  f_{\symup{s}} (t,\,b) = L^{-d} f_{\symup{s}} (L^x t, \, L^y b).
\end{equation}
取 $L^x t=1$,则有
\begin{equation}
  f_{\symup{s}} (t,\,b) = t^{d/x} f_{\symup{s}} (1, \, L^y b).
\end{equation}
考虑到热容 $C$ 正比于自由能对温度的二阶导数,可得
\begin{equation}
  C \sim \pdv[2]{f_{\symup{s}}}{t} \sim t^{d/x-2},
\end{equation}
即临界指数
\begin{equation}
  \alpha = 2 - \frac{d}{x}.
\end{equation}
类似地,可以计算出其他临界指数满足的条件:
\begin{equation}
  \beta  = \frac{d-y}{x}  \qc
  \gamma = \frac{2y-d}{x} \qc
  \delta = \frac{y}{d-y}.
\end{equation}
消去 $x$ 、$y$,即得到标度律
\begin{equation}
  \label{eq:scaling-relation-i-ii}
  \alpha + 2\beta + \gamma = 2 \qc
  \gamma = \beta (\delta - 1).
\end{equation}

另一方面,关联长度满足变换关系 $\xi\to\xi'=L^{-1}\xi$。代入式~\eqref{eq:transform-relation-t-b},得
\begin{equation}
  \frac{\xi'}{\xi} = \qty(\frac{t'}{t})^{-\nu} = L^{x\nu} = L^{-1}
  \implies \nu = \frac{1}{x}.
\end{equation}
因此得到另一条标度律
\begin{equation}
  \label{eq:scaling-relation-iii}
  \nu d = 2 - \alpha.
\end{equation}

如果定义关联函数
\begin{equation}
  g(\bm{r}_{ij}) = \overline{(\sigma_i-\bar{\sigma_i})(\sigma_j-\bar{\sigma_j})},
\end{equation}
则有另一个临界指数 $\eta$ 与之关联:
\begin{equation}
  g(\bm{r}) \sim r^{2-d-\eta}.
\end{equation}
并同样有标度律
\begin{equation}
  \label{eq:scaling-relation-iv}
  \eta = 2 + d - 2y \implies \gamma = (2-\eta) \nu.
\end{equation}

这样,利用四条标度律 \eqref{eq:scaling-relation-i-ii}、\eqref{eq:scaling-relation-iii} 和
\eqref{eq:scaling-relation-iv}~式,就可以计算得到 Ising 模型的其他几个临界指数:
\begin{equation}
  \gamma = \frac{7}{4} \qc
  \delta = 15          \qc
  \nu    = 1           \qc
  \eta   = \frac{1}{4}.
\end{equation}

\section{重整化群}

前面已经指出,在相变临界点处关联长度 $\xi$ 会出现发散。因此,$\xi$ 在任意尺度变换下均是不变的,
即对应变换的不动点。根据这一观点,我们可以直接计算出临界指数而无需严格求解配分函数,这称为
重整化群方法。

一般而言,体系的配分函数总可以写成
\begin{equation}
  Z_N = \sum_\q{\sigma_i} \ee^{-\beta H(\q{\sigma_i})},
\end{equation}
其中,体系的 Hamilton 量为
\begin{equation}
    H(\q{\sigma_i})
  = H(\q{\sigma_i},\,\bm{K})
  = - \sum_i K^{(1)}_i \sigma_i
    - \sum_{\nearest{ij}} K^{(2)}_{ij} \sigma_i\sigma_j
    - \sum_{\nearest{ijk}} K^{(3)}_{ijk} B_i\sigma_i\sigma_j\sigma_k - \cdots
\end{equation}
耦合参数 $\bm{K}$ 可以表征体系中的多种相互作用。

接下来对体系进行一次标度变换,即粗粒近似。此时,多个自旋被统一处理,视作一个 Kadanoff 集团。
自旋数目(即体系自由度)$N$ 和关联长度 $\xi$ 分别变为
\begin{equation}
  \label{eq:N-xi-renormalization}
  N' = L^{-d} N \qc \xi' = L^{-1} \xi.
\end{equation}
其中 $L>1$ 为缩放系数,$d$ 则为体系的维度。经过一次粗粒近似后,体系的配分函数数值上应当保持不变,
但形式上需要变为对 Kadanoff 集团 $\q{\sigma_i'}$ 的求和:
\begin{equation}
  Z_{N'} = \sum_\q{\sigma'_i} \ee^{-\beta H(\q{\sigma_i'},\,\bm{K}')},
\end{equation}
式中的 $\bm{K}'$ 为变换后的耦合参数,它满足
\begin{equation}
  \label{eq:K-renormalization}
  \bm{K}' = \trR{L}(\bm{K}).
\end{equation}
考虑到 $\bm{K}$ 作为向量可以张成一个向量空间 $\spaceV$,我们可以把 $\trR{L}$ 理解为 $\spaceV$
上的一个算子(未必要求线性)。

粗粒近似可以重复进行,即
\begin{equation}
  \bm{K} \to \bm{K}' = \trR{L}(\bm{K}) \to \cdots \to \bm{K}^{(n)} = \trR[n]{L}(\bm{K}).
\end{equation}
可见 $\trR[k]{L}$ 连同恒等变换构成一个半群(不存在逆变换)。根据重整化群理论的精神,重整化变换
$\trR{L}$ 存在不动点 $\bm{K}^*$,满足
\begin{equation}
  \trR{L}(\bm{K}^*) = \bm{K}^*.
\end{equation}
在不动点附近分别展开 $\bm{K}$ 和 $\bm{K}'$,得到
\begin{equation}
  \bm{K}  = \bm{K}^* + \delta{\bm{K}} \qc
  \bm{K}' = \bm{K}^* + \delta{\bm{K}'}.
\end{equation}
代入 \eqref{eq:K-renormalization}~式,有
\begin{align}
  &\mathrel{\phantom{\implies}}
    \bm{K}^* + \delta{\bm{K}'}
  = \trR{L}(\bm{K}^* + \delta{\bm{K}})
  = \trR{L}(\bm{K}^*) + \eval{\pdv{\trR{L}}{\bm{K}}}_{\bm{K}^*} \delta{\bm{K}} \notag \\
  &\implies
    \delta{\bm{K}'}
  = \eval{\pdv{\trR{L}}{\bm{K}}}_{\bm{K}^*} \delta{\bm{K}}
  = \trA[*]{L} \delta{\bm{K}}.
\end{align}
式中,$\trA[*]{L}$ 是 $\trR{L}$ 在 $\bm{K}^*$ 处的线性近似。设 $\trA[*]{L}$ 的本征值与
本征向量分别为 $\lambda_i$ 和 $\bm{\phi}_i$,则可将 $\delta{\bm{K}}$ 和 $\delta{\bm{K}'}$ 展开:
\begin{equation}
  \delta{\bm{K}}  = \sum_i u_i  \bm{\phi}_i \qc
  \delta{\bm{K}'} = \sum_i u'_i \bm{\phi}_i.
\end{equation}
两边作用 $\trA[*]{L}$,可得到系数满足的关系
\begin{equation}
  u_i' = \lambda_i u_i.
\end{equation}
同理,作用 $n$ 次 $\trA[*]{L}$,即 $\qty\big(\trA[*]{L})^n$,可有
\begin{equation}
  u_i^{(n)} = \lambda_i^n u_i^{(0)}.
\end{equation}
显然,若本征值 $\lambda_i>1$,则随着 $n$ 的增加,$u_i^{(n)}$ 的权重越来越大,因此决定了体系的
临界行为,对应的参数 $u_i$ 称为相关变量;反之,若本征值 $\lambda_i<1$,则 $u_i^{(n)}$ 的权重随着
$n$ 的增加越来越小,故对应的 $u_i$ 称为无关变量。$\lambda_i=1$ 对应的 $u_i^{(n)}$ 会保持定常行为,
能够带来介于以上二者之间的影响,如幂律行为中的对数项。

对于 Ising 模型,可以给出两个相关变量,它们分别与约化温度和约化磁场关联:
\begin{equation}
  u_1 \propto t \qc u_2 \propto b.
\end{equation}
根据式~\eqref{eq:N-xi-renormalization},我们有
\begin{equation}
  \xi' = L^{-1} \xi \implies \xi^{(n)} = L^{-n} \xi^{(0)}.
\end{equation}
关联长度 $\xi$ 是 $u_i$ 的函数,因此
\begin{equation}
  \xi^{(n)} = \xi\qty\big(\q{u_i^{(n)}}) = \xi\qty\big(\q{\lambda_i^n u_i^{(0)}}) \qc
  \xi^{(0)} = \xi\qty\big(\q{u_i^{(0)}}).
\end{equation}
根据临界指数 $\nu$ 的定义,有
\begin{equation}
  \xi \propto t^{-\nu} \propto u_1^{-\nu}.
\end{equation}
联立以上三式,我们有
\begin{equation}
  \xi^{(n)} \propto \lambda_1^{-n\nu} u_1^{-\nu} \qc
  \xi^{(0)} \propto u_1^{-\nu}
  \implies \frac{\xi^{(n)}}{\xi^{(0)}} = \lambda_1^{-n\nu} = L^{-n}
  \implies \nu = \frac{\ln L}{\ln \lambda_1}.
\end{equation}
这样,我们就利用重整化群方法计算得到了临界指数 $\nu$。一旦给定了具体的重整化变换 $\trR{L}$,
就立刻可以给出 $\nu$ 的数值,而无需对配分函数进行严格求解。

\section{Ising 模型与 \AdSCFT{}}

\section{Ising 模型的 Monte Carlo 模拟}

\subsection{Metropolis 算法}

在 \eqref{eq:ising-probability}~式中,我们给出了构型 $\q{\sigma_i}$ 在平衡态中出现的概率。严格
计算这一概率,需要对所有可能的构型进行求和。设自旋数目为 $N$,则构型数正比于 $N!$,可见直接对它们
进行求和是不现实的。为了回避过于巨大的计算开销,Monte Carlo 算法给出了另外一种方案,即以按照一定
概率的抽样,代替对全部构型的平均。

考虑自旋构型 $\q{\sigma_i}$ 随时间的演化。设在 $t$ 时刻,构型 $\q{\sigma_i}$ 对应的概率为
$P\qty\big(\q{\sigma_i},\,t)$,则
\begin{align}
  P\qty\big(\q{\sigma_i},\,t+1)
  &= P\qty\big(\q{\sigma_i},\,t) \notag \\
  &+ \sum_{\q{\sigma'_i}} P\qty\big(\q{\sigma'_i},\,t) W\qty\big(\q{\sigma'_i}\to\q{\sigma_i})
   - \sum_{\q{\sigma'_i}} P\qty\big(\q{\sigma_i},\,t)  W\qty\big(\q{\sigma_i}\to\q{\sigma'_i}).
\end{align}
式中,$W\qty\big(\q{\sigma_i}\to\q{\sigma'_i})$ 表示从 $\q{\sigma_i}$ 到 $\q{\sigma'_i}$ 的
转移概率,需要满足细致平衡条件:
\begin{equation}
  \label{eq:detailed-balance}
    P_{\text{eq}}(\q{\sigma_i})  W\qty\big(\q{\sigma_i}\to\q{\sigma'_i})
  = P_{\text{eq}}(\q{\sigma'_i}) W\qty\big(\q{\sigma'_i}\to\q{\sigma_i}).
\end{equation}
平衡态下,构型 $\q{\sigma_i}$ 对应的概率可由式~\eqref{eq:ising-probability} 获得,它正比于
$\ee^{-\beta H(\q{\sigma_i})}$,因而有
\begin{equation}
    \frac{W\qty\big(\q{\sigma_i}\to\q{\sigma'_i})}{W\qty\big(\q{\sigma'_i}\to\q{\sigma_i})}
  = \frac{\ee^{-\beta H(\q{\sigma'_i})}}{\ee^{-\beta H(\q{\sigma_i})}}
  = \ee^{-\beta \incr{E}}.
\end{equation}
其中的 $\incr{E}$ 表示两构型间的能量差。考虑到概率的有效取值范围为 $[0,\,1]$,我们有
\begin{equation}
  \label{eq:metropolis-probability}
  W\qty\big(\q{\sigma_i}\to\q{\sigma'_i}) =
  \begin{cases}
    1, & \incr{E} \leqslant 0; \\
    \ee^{-\beta \incr{E}}, & \incr{E} > 0. \\
  \end{cases}
\end{equation}

Metropolis 算法是 Monte Carlo 模拟的一种具体实现,它给出了选取下一时刻自旋构型的方案。
具体步骤可以表述如下:

\begin{enumerate}
  \item 初始化所有自旋,可以随机取为 0 或 1。但如果初始时就令自旋(近似)满足一定的概率分布,
    可以使得算法更为高效。
  \item 在自旋构型 $\q{\sigma_i}$ 的基础上,随机生成一个与之相差不大的构型 $\q{\sigma'_i}$。
    通常做法是翻转某一个自旋。
  \item 计算能量差 $\incr{E}$,按照式~\eqref{eq:metropolis-probability} 获得接受概率 $P$。生成一个
    区间 $(0,\,1)$ 内的随机数 $\xi$,若 $\xi<P$,则接受构型 $\q{\sigma'_i}$(即翻转相应自旋);
    否则不进行操作。
  \item 遍历整个点阵,这成为一个 Monte Carlo 步(Monte Carlo sweep, MCS)。为满足细致平衡条件
    \eqref{eq:detailed-balance}~式,原则上需在点阵中随机选取自旋,并计算接受概率/决定是否翻转;
    但出于效率的考虑,也常常直接按次序遍历整个点阵。
  \item 重复进行 2--4 步,测量各物理量的,并计算它们的统计平均值及标准误差。
\end{enumerate}

\subsection{模拟结果}

在后文中,机器学习所用到的 Ising 模型训练数据集均利用上面介绍的 Metropolis 算法生成。为了文章的
完整性,这里我们也顺带给出 Ising 模型的 Monte Carlo 模拟结果。

以下结果中,Ising 点阵大小分别为 $16 \times 16$、$32 \times 32$、$64 \times 64$ 和
$128 \times 128$,并采用周期性边界条件。选取温度区间 $T/J \in [1,\,3.6]$,外磁场为零。进行
\num{900} MCS 之后,认为体系达到平衡态。之后再进行 \num{100} MCS,并在其中取系综平均计算各物理量。
每个温度下,重复模拟 50 次,统计各物理量的平均值及标准差(作为不确定度)。模拟结果见
图~\ref{fig:ising-observables}。

\begin{figure}[htb]
  \begin{subfigure}[b]{0.47\textwidth}
    \hfill
    \imageinput{ising-energy}
    \phantomcaption{}
  \end{subfigure}
  \begin{subfigure}[b]{0.47\textwidth}
    \hfill
    \imageinput{ising-magnet}
    \phantomcaption{}
  \end{subfigure}
  \\[3ex]
  \begin{subfigure}[b]{0.47\textwidth}
    \hfill
    \imageinput{ising-cv}
    \phantomcaption{}
  \end{subfigure}
  \begin{subfigure}[b]{0.47\textwidth}
    \hfill
    \imageinput{ising-cv-exact}
    \phantomcaption{}
    \label{fig:ising-cv-exact}
  \end{subfigure}
  \caption{Ising 模型各物理量随温度的变化曲线。为了更清晰地反映出相变的临界情况,我们在临界点附近
    增大了采样密度。图中还用灰线标出了临界温度 $\Tc$ 的位置。图中的误差棒代表正负一个标准差。
    (a)~能量;(b)~自发磁化;(c)~热容,注意到在较小的点阵中存在巨大的涨落 \protect\footnotemark{};
    (d)~热容的精确计算结果,额外给出了 $4 \times 4$、$8 \times 8$ 和 $256 \times 256$ 点阵中的情形}
  \label{fig:ising-observables}
\end{figure}

\footnotetext{不确定度(标准差)在 $T/J=\num{1.8}$ 和 \num{2.8} 处的突变来自于采样的不均匀。为了提高临界点附近的精度,该区间之内温度采样间隔为 \num{0.02},区间之外则为 \num{0.1}。}

根据模拟结果,我们发现随着体系尺度的增加,临界行为逐渐接近 \ref{sec:ising-physics}~节中给出的
严格计算结果,如自发磁矩的幂律发散、热容的对数发散等。

为了与模拟结果进行对照,我们还给出了有限体系热容的精确结果,见图~\ref{fig:ising-cv-exact}。
可以看到,随着尺度的增加,热容的极大值逐渐增大,极值点出现的位置也向 $\Tc$ 靠近。事实上,将极值点
位置 $T_*$ 和热容极大值 $C_*$ 关于体系尺度 $L$ 分别进行拟合,可以得到如下关系:
\begin{equation}
  T_* - \Tc \propto L^{-1} \qc
  C_*       \propto \ln L.
\end{equation}
结果见图~\ref{fig:ising-fit}。

\begin{figure}
  \begin{subfigure}[b]{0.47\textwidth}
    \hfill
    \imageinput{ising-cv-fit-i}
  \end{subfigure}
  \begin{subfigure}[b]{0.47\textwidth}
    \hfill
    \imageinput{ising-cv-fit-ii}
  \end{subfigure}
  \caption{利用图~\ref{fig:ising-cv-exact} 中热容的精确计算结果关于体系尺度拟合。
    (a)~临界温度随点阵尺度的变化,$x$、$y$ 轴均取对数坐标;
    (b)~热容极大值随点阵尺度的变化,只有 $x$ 轴取对数坐标}
  \label{fig:ising-fit}
\end{figure}

\chapter{机器学习简介}

\section{机器学习有关思想方法}

机器学习是当今计算机科学的重要分支。它可以利用计算机的强大计算能力,从大量数据中提取特征,并构建
相应的模型。生成的模型还可以用来进行预测。

考虑最简单的回归模型。假设我们给定了一些数据点 $(\bm{x}^{(i)}, \, y^{(i)})$,其中 $\bm{x}^{(i)}$
称为特征(feature),$y^{(i)}\in\R$ 称为标签(label)。训练样本的数量为 $m$,即
$i=1,\,2,\,\ldots,\,m$。特征的数量为 $n$,但按照惯例,取 $x_0=0$,故有 $\bm{x}^{(i)}\in\R^{n+1}$。
一般而言,特征反映了样本的各种性质(如图片中的像素点),而标签则是需要需要预测的内容(如图片对应的
物体类别)。为了学习样本数据,我们需要给出一个目标函数 $h_\bm{\theta}: \R^{n+1}\to\R$,使得
$h_\bm{\theta}(\hat{\bm{x}})=\hat{y}$。这里的 $\bm{\theta}$ 称为参数(parameter),它包含了模型的
全部训练信息。

显然,如果要求学习的效果越好,就要求预测值 $\hat{y}$ 与真实值 $y$ 的差距越小。因此我们需要定义
损失函数(loss function) $L$,并使 $L$ 关于参数 $\bm{\theta}$ 取最小值。如果数据规模较小、模型比较
简单,$L$ 的最小值可以解析求得。例如线性回归模型,即
\begin{equation}
  h_\bm{\theta} (\bm{x})
  = \theta_0 + \theta_1 x_1 + \theta_2 x_2 + \cdots + \theta_n x_n
  = \sum_{j=0}^n \theta_j x_j,
\end{equation}
我们取损失函数为
\begin{equation}
  L(\bm{\theta})
  = \frac{1}{2} \sum_{i=1}^m \qty\Big[y^{(i)} - h_\bm{\theta}\qty\big(\bm{x}^{(i)})]^2
  = \frac{1}{2} \norm{\bm{y} - \bm{X}\bm{\theta}}^2 \in \R.
\end{equation}
式中 $\bm{X}\in\R^{m\times(n+1)}$,且有 $X_{ij} = x^{(i)}_j$。利用最小二乘估计,可以求得最优解
\begin{equation}
  \hat{\bm{\theta}} = \qty\big(\bm{X}^\trans\bm{X})^{-1} \bm{X}^\trans \bm{y}.
\end{equation}

然而,当数据规模很大时,可以发现矩阵 $\bm{X}^\trans\bm{X}$ 的阶数也会很大,导致求其逆阵的开销无法
接受。因此可以改换思路,利用梯度下降法逼近损失函数的最小值。对于一般的多元函数,梯度给出了其在某一点
处变化率最大的方向。因而沿着梯度方向不断改变 $\bm{\theta}$ 的值,就可以逐步趋近最小值。我们可以令
\begin{equation}
  \bm{\theta} \coloneq \bm{\theta} - \alpha\nabla\bm{\theta},
\end{equation}
并不断循环,直至收敛。式中 $\alpha$ 称为学习率,用来调节每次下降的“步幅”。

在回归模型中,我们对每一组数据都给出了对应的标签。这称之为监督学习(supervised learning)。与之
相对应,另外还有无监督学习(unsupervised learning),指的是不需要为数据提供标签。无监督学习要求机器
从数据本身获取更加深刻的信息,因此相比监督学习更为困难。

除了回归问题(物理学中一般称为拟合),监督学习还包括分类等。无监督学习主要包括聚类、降维等。

一般来说,机器学习的思路可以概括如下:

\begin{enumerate}
  \item 根据数据的具体类型、特性构建相应的模型;
  \item 给出模型的损失函数;
  \item 在数据集上进行训练,找出能使损失函数取最小值的参数;
  \item 验证模型及参数的有效性,进行参数微调。
  \item 在新的数据集上测试训练好的模型。
\end{enumerate}

\section{对 Ising 模型相变的研究:线性模型}

在给定的温度下,我们可以利用 Monte Carlo 算法生成一系列 Ising 模型的自旋构型( configuration )。
不妨把构型作为前文所述的特征 $\bm{x}^{(i)}$,把温度作为标签 $y^{(i)}$。对大量数据样本进行训练后,
给定某一自旋构型,即可以利用机器学习的手段获知其所对应的温度。对于二维正方晶格的 Ising 模型,
其相变临界点是已知的,由此还可以知道该自旋构型处于有序态(铁磁)或是无序态(顺磁)。

\subsection{训练模型}

下面我们利用线性分类模型分析由 Monte Carlo 算法生成的 Ising 自旋构型。Ising 自旋构型是一个
$N\times N$ 的矩阵,每一个矩阵元仅可取 1 或 0。这称为布尔型数据(binary)。为了计算方便,接下来把
这一矩阵压平为一位向量,即 $\bm{x}^{(i)}\in\R^{N^2}$。对于标签 $y^{(i)}$,我们同样进行二值化,即有
\begin{equation}
  \begin{cases}
    0, & T <         \Tc; \\
    1, & T \geqslant \Tc.
  \end{cases}
\end{equation}

二值化的线性分类模型可以用下式表示:
\begin{equation}
  y = \sigma(\bm{W}^\trans \bm{x} + b).
\end{equation}
式中,$\bm{W}\in\R^{N^2}$ 称为权重(weight),$b\in\R$ 称为偏差(bias)。注意到前文的
$\bm{\theta}$ 相当于这里 $\bm{W}$ 和 $b$ 的组合,并有 $b=\theta_0$。$\sigma$ 称为激活函数
(activation function),其作用是将线性模型给出的结果映照到 $\q{0,\,1}$ 上。为了计算的方便,常把
激活函数取为逻辑函数(logistics sigmoid function),它满足
\begin{equation}
  \sigma(z) = \frac{1}{1+\ee^{-z}}.
\end{equation}
注意到 $\sigma$ 的值域为连续区间 $(0,\,1)$,我们把这一结果结果理解为 $y$ 属于 1 类别的概率
$P(y=1)$。

我们可以用网络图来描述这一模型,见图。


\chapter{机器学习与重整化群}

\section{限制 Boltzmann 机(RBM)}

限制 Boltzmann 机(restricted Boltzmann machine, RBM)是一类生成型随机神经网络(generative
stochastic neural network),它可以学习到输入数据的概率分布情况 \cite{hinton2006reducing}。
顾名思义,RBM 是 Boltzmann 机加上一定限制所得到的网络。一般的 Boltzmann 机是一种基于能量的模型
(energy-based model)\cite{zhouzhihua,rbmonline}。对于这种模型,我们可以赋予它一个能量函数。
与物理学中类似,体系的稳定状态将在能量最低时达到,这一稳定状态即对应最优参数。

Boltzmann 机的结构如图~\ref{fig:boltzmann-machine} 所示。它共有两层,与数据直接相连的称为可见层
(visible layer),另一层称为隐藏层(hidden layer)。可见层用于处理输入输出,隐藏层则反应了数据的
内在结构。无论是可见层还是隐藏层,所有单元均全部连接在一起。Boltzmann 机的能量函数由下式给出
\cite{zhouzhihua}:
\begin{equation}
  E(\bm{s}) = E(\q{s_i}) = - \sum_{i<j} W_{ij} s_i s_j - \sum_i \theta_i s_i.
\end{equation}
式中,$s_i$ 取为布尔型,即 $s_i\in\q{0,\,1}$。$W_{ij}$ 是单元 $s_i$ 与 $s_j$ 之间的连接权重,
$\theta_i$ 则表示单元 $s_i$ 处的偏差。

\begin{figure}[htb]
  \centering
  \begin{subfigure}[b]{0.45\textwidth}
    \centering
    \imageinput{boltzmann-machine.pdf}
    \caption{}
    \label{fig:boltzmann-machine}
  \end{subfigure}
  \begin{subfigure}[b]{0.45\textwidth}
    \centering
    \imageinput{rbm.pdf}
    \vspace{0.4cm}
    \caption{}
    \label{fig:rbm}
  \end{subfigure}
  \caption{Boltzmann 机和限制 Boltzmann 机的结构示意图。其中红色圆圈表示可见层,蓝色方块表示隐藏层。
    (a)~Boltzmann 机,所有单元之间均有权重矩阵连接;
    (b)~限制 Boltzmann 机,只有可见层与隐藏层之间才连接有权重矩阵}
\end{figure}

与 Ising 模型类似[式~\eqref{eq:ising-probability}],状态 $\bm{s}=\q{s_i}$ 出现的概率由 Boltzmann
分布给出 \cite{rbmonline}:
\begin{equation}
  \label{eq:boltzmann-machine-probability}
  P(\bm{s}) = \frac{\ee^{-E(\bm{s})}}{Z},
\end{equation}
其中配分函数 $Z$ 的定义为
\begin{equation}
  Z = \sum_{\bm{s}} \ee^{-E(\bm{s})}.
\end{equation}
Boltzmann 机的训练过程就是最大化训练样本所对应的概率。

由于 Boltzmann 机是全连接的,它的训练效率并不高。我们可以引入限制条件,仅保留可见层与隐藏层之间的
连接。由此便得到了限制 Boltzmann 机(如图~\ref{fig:rbm}),它的能量函数为
\cite{exact,rbmonline,chen2018equivalence}
\begin{equation}
  \label{eq:rbm-energy}
  E(\bm{v},\,\bm{h})
  = - \bm{v}^\trans \bm{W} \bm{h} - \bm{b}^\trans \bm{v} - \bm{c}^\trans \bm{h}
  = - \sum_{i,\,j} W_{ij} v_i h_j - \sum_i b_i v_i - \sum_j c_j h_j.
\end{equation}
式中,$v_i$ 和 $h_j$ 分别代表可见层与隐藏层中的单元,$W_{ij}$ 为两层之间的连接权重,$b_i$ 和 $c_j$
分别是可见层与隐藏层中单元对应的偏差。

RBM 中,可见层 $\bm{v}$ 与隐藏层 $\bm{h}$ 是彼此分离的,可以视为两个随机变量,因而
式~\eqref{eq:boltzmann-machine-probability} 即成为 $\bm{v}$ 与 $\bm{h}$ 的联合概率分布:
\begin{equation}
  P(\bm{v},\,\bm{h})
  = \frac{\ee^{-E(\bm{v},\,\bm{h})}}{Z}
  = \frac{\ee^{-E(\bm{v},\,\bm{h})}}{\sum_{\bm{v},\,\bm{h}}\ee^{-E(\bm{v},\,\bm{h})}}.
\end{equation}
对所有的 $\bm{h}$ 进行求和,我们就得到了某一可见层 $\bm{v}$ 对应的边缘概率分布:
\begin{equation}
  \label{eq:rbm-probability-v}
  P(\bm{v}) = \sum_{\bm{h}} P(\bm{v},\,\bm{h})
            = \frac{1}{Z} \sum_{\bm{h}} \ee^{-E(\bm{v},\,\bm{h})}
            \equiv \frac{1}{Z} \ee^{-H^{\text{RBM}}(\bm{v})};
\end{equation}
同理,对所有的 $\bm{v}$ 进行求和,可得到隐藏层 $\bm{h}$ 对应的边缘概率分布:
\begin{equation}
  \label{eq:rbm-probability-h}
  P(\bm{h}) = \sum_{\bm{h}} P(\bm{v},\,\bm{h})
            = \frac{1}{Z} \sum_{\bm{v}} \ee^{-E(\bm{v},\,\bm{h})}
            \equiv \frac{1}{Z} \ee^{-H^{\text{RBM}}(\bm{h})}.
\end{equation}
这里定义的 $H^{\text{RBM}}(\bm{v})$ 和 $H^{\text{RBM}}(\bm{h})$ 称为变分 Hamilton 量
(variational Hamiltonian)\cite{exact}。

\section{\texorpdfstring{CD\raisebox{0.13ex}{-}\textit{k}}{CD-k} 训练算法}
\label{sec:CDk-algorithm}

RBM 常采用对比散度算法(contrastive divergence, \CDk)进行训练,它是一种计算极大似然
(maximum likelihood)梯度值的近似方法
\cite{zhouzhihua,rbmonline,lyy1994rbm,hinton2012practical,hinton2002training}。

\subsection{损失函数}

对于一般的基于能量模型,其损失函数由下式给出 \cite{rbmonline}:
\begin{equation}
  \label{eq:ebm-loss-function}
  L(\bm{\theta})
  = - \likeL(\bm{\theta},\,\domD)
  = - \frac{1}{\abs{\domD}} \sum_{\bm{x}^{(i)}\in\domD} \ln P\qty\big(\bm{x}^{(i)}).
\end{equation}
式中的 $\domD$ 表示数据集,$\likeL$ 称为对数似然(log-likelihood)。根据梯度下降法的精神,我们需要
求出损失函数关于参数的导数(即梯度):
\begin{equation}
  \pdv{L(\bm{\theta})}{\bm{\theta}}
  = \frac{1}{\abs{\domD}} \sum_{\bm{x}^{(i)}\in\domD}
    - \pdv{\bm{\theta}} \ln P\qty\big(\bm{x}^{(i)}).
\end{equation}
实际上只需计算求和号里面的内容 $-\partial{\ln P\qty\big(\bm{x}^{(i)})}\,/\,\partial{\bm{\theta}}$。

RBM 所采用的训练数据集只包含了可见层的内容,即
\begin{equation}
  - \pdv{\bm{\theta}} \ln P\qty\big(\bm{x}^{(i)}) \equiv
  - \pdv{\bm{\theta}} \ln P\qty\big(\bm{v}^{(i)}).
\end{equation}
仿照统计物理的处理方法,引入自由能
\footnote{在统计物理中,自由能 $F=-T\ln Z$,而配分函数需要对所有可能的状态求和。但在限制
  Boltzmann 机中,只对隐藏层求和,所以 $F$ 是 $\bm{v}$ 的函数。} \cite{rbmonline}
\begin{equation}
  F(\bm{v}) = - \ln \sum_{\bm{h}} \ee^{-E(\bm{v},\,\bm{h})}.
\end{equation}
代入式~\eqref{eq:rbm-probability-v},可有
\begin{equation}
  \ln P\qty\big(\bm{v}^{(i)}) = \frac{\ee^{-F(\bm{v}^{(i)})}}{Z} \qc
  Z = \sum_{\bm{v}} \ee^{-F(\bm{v})}.
\end{equation}
于是梯度可以表示为
\begin{align}
  \label{eq:rmb-gradient}
     - \pdv{\bm{\theta}} \ln P\qty\big(\bm{v}^{(i)})
  &= - \pdv{\bm{\theta}} \qty\Big[-F\qty\big(\bm{v}^{(i)}) - \ln Z]
     \vphantom{\sum_x} \notag \\
  &= \pdv{F\qty\big(\bm{v}^{(i)})}{\bm{\theta}} + \frac{1}{Z} \pdv{Z}{\bm{\theta}}
   = \pdv{F\qty\big(\bm{v}^{(i)})}{\bm{\theta}}
     - \frac{1}{Z} \sum_{\bm{v}} \ee^{-F(\bm{v})} \pdv{F(\bm{v})}{\bm{\theta}} \notag \\
  &= \pdv{F\qty\big(\bm{v}^{(i)})}{\bm{\theta}}
     - \sum_{\bm{v}} P(\bm{v}) \pdv{F(\bm{v})}{\bm{\theta}}.
\end{align}
其中的参数 $\bm{\theta}$ 包含连接权重 $\bm{W}$、可见层偏差 $\bm{b}$ 与隐藏层偏差 $\bm{c}$。

式~\eqref{eq:rmb-gradient} 共包含两项,分别称为正相(positive phase)和负相(negative phase)。
显然,负相的计算比较困难,这也是 \CDk{} 算法着力需要解决的问题 \cite{rbmonline}。

\subsection{Gibbs 采样}

计算负相的困难之处在于需要对所有可能的可见层 $\bm{v}$ 分布进行求和,而这一分布对应的状态数量极其
巨大。与我们在处理 Ising 模型的解决方法类似,这里同样可以用一组样本的分布代替整体分布,即 Monte
Carlo 方法 \cite{rbmonline}:
\begin{equation}
  \sum_{\bm{v}} P(\bm{v}) \pdv{F(\bm{v})}{\bm{\theta}}
  \approx \frac{1}{\abs{\domN}} \sum_{\bm{v}\in\domN} \pdv{F(\bm{v})}{\bm{\theta}}.
\end{equation}
式中的 $\domN$ 表示一组依照概率 $P$ 进行的采样。

由于可见层、隐藏层的层内均没有连接,因此可以在层之间来回“跳转”,并将其视作一条 Markov 链,
如图~\ref{fig:gibbs-sampling} 所示。“跳转”的具体操作可以借由条件概率获得:设给定可见层分布
$\bm{v}$,则隐藏层分布为
\begin{equation}
  \label{eq:hidden-layer-dist}
  \bm{h} \sim P(\bm{h}\big|\bm{v});
\end{equation}
由此可计算下一组可见层分布
\begin{equation}
  \label{eq:visible-layer-dist}
  \bm{v}' \sim P(\bm{v}\big|\bm{h}).
\end{equation}
式中,符号 $\bm{x}\sim P$ 表示按照概率 $P$ 生成 $\bm{x}$。考虑到 $\bm{v}$ 与 $\bm{h}$ 均为布尔型,
我们可以近似用概率(的取整结果)直接代替 $\bm{v}$ 和 $\bm{h}$。

\begin{figure}[htb]
  \centering
  \imageinput{gibbs-sampling.pdf}
  \caption{Gibbs 采样的示意图。上标代表进行的采样次数}
  \label{fig:gibbs-sampling}
\end{figure}

以上步骤可以持续进行,这称为 Gibbs 采样(Gibbs sampling),采样次数即为 \CDk{} 算法中的 $k$。
当 $k\to\infty$ 时,可认为 $\bm{v}^{(k)}$、$\bm{h}^{(k+1)}$
\footnote{这里的上标 $(k)$ 和 $(k+1)$ 是采样次数,而样本指标 $(i)$ 不同。}
已经服从所要求的分布 $P(\bm{v})$。显然,$k\to\infty$ 在计算上是不可取的。实践中,往往选择 $k<15$
以提高计算速度。甚至在大多数情况下,取 $k=1$ 即可满足要求。此时,\eqref{eq:rmb-gradient}~式中的
梯度近似可以表示为 \cite{rbmonline,hinton2012practical,hinton2002training}
\begin{equation}
  \label{eq:rmb-gradient-approx}
  - \pdv{\bm{\theta}} \ln P\qty\big(\bm{v}^{(i)})
  \approx \pdv{F\qty\big(\bm{v}^{(i)})}{\bm{\theta}}
        - \pdv{F\qty\big({\bm{v}'}^{(i)})}{\bm{\theta}}.
\end{equation}
其中的 ${\bm{v}'}^{(i)}$ 表示经过 Gibbs 采样生成的可见层单元。

\subsection{梯度的计算}

首先来计算式~\eqref{eq:hidden-layer-dist}、\eqref{eq:visible-layer-dist} 中出现的条件概率
$P(\bm{h}|\bm{v})$。利用乘法公式,有 \cite{zhouzhihua,rbmonline}
\begin{align}
  \label{eq:conditional-probability}
  P(\bm{h}|\bm{v})
  &= \frac{P(\bm{v},\,\bm{h})}{P(\bm{v})}
   = \frac{\ee^{-E(\bm{v},\,\bm{h})}}%
          {\sum_{\tilde{\bm{h}}} \ee^{-E(\bm{v},\,\tilde{\bm{h}})}} \notag \\
  &= \frac{\exp\qty\big(\bm{v}^\trans\bm{W}\bm{h} + \bm{b}^\trans\bm{v} + \bm{c}^\trans\bm{h})}%
          {\sum_{\tilde{\bm{h}}} \exp\qty\big(\bm{v}^\trans\bm{W}\tilde{\bm{h}}
           + \bm{b}^\trans\bm{v} + \bm{c}^\trans\tilde{\bm{h}})}
   = \frac{\exp\qty\big[\qty\big(\bm{v}^\trans\bm{W} + \bm{c}^\trans)\bm{h}\,]}%
          {\sum_{\tilde{\bm{h}}} \exp\qty\big[\qty\big(\bm{v}^\trans\bm{W} + \bm{c}^\trans)
                                              \tilde{\bm{h}}\,]} \notag \\
  &= \frac{\exp\qty\big[\,\sum_j h_j \qty\big(c_j + \sum_i W_{ij}v_i)]}%
          {\sum_{\q{\tilde{h}_i}} \exp\qty\big[\,\sum_j \tilde{h}_j
                                               \qty\big(c_j + \sum_i W_{ij}v_i)]} \notag \\
  &= \prod_j \frac{\exp\qty\big[h_j \qty\big(c_j + \sum_i W_{ij}v_i)]}%
                  {\sum_{\q{\tilde{h}_i}} \exp\qty\big[\tilde{h}_j
                                               \qty\big(c_j + \sum_i W_{ij}v_i)]}
   = \prod_j P(h_j|\bm{v}).
\end{align}
考虑到 $h_j$ 只能取 0 或 1,又有 \cite{rbmonline,hinton2012practical,andrzejewski2009training}
\begin{align}
  P(h_j|\bm{v})
   = P(h_j=1|\bm{v})
  &= \frac{\exp 1 \cdot \qty\big(c_j + \sum_i W_{ij}v_i)}%
          {\exp 0 \cdot \qty\big(c_j + \sum_i W_{ij}v_i)
           + 1 \cdot \qty\big(c_j + \sum_i W_{ij}v_i)} \notag \\
  &= \frac{\exp\qty\big(c_j + \sum_i W_{ij}v_i)}{1 + \exp\qty\big(c_j + \sum_i W_{ij}v_i)}
   = \sigma\qty\big(c_j + \sum_i W_{ij}v_i).
\end{align}
同理,
\begin{equation}
  P(\bm{v}|\bm{h}) = \prod_i P(v_i=1|\bm{h}) = \prod_i \sigma\qty\big(b_i + \sum_j W_{ij}h_j).
\end{equation}
利用概率近似代替生成的 $\bm{h}$ 和 $\bm{v}'$,我们有
\begin{equation}
  \label{eq:h-v-approx}
  h_j  \approx \sigma\qty\big(c_j + \sum_i W_{ij}v_i) \qc
  v'_i \approx \sigma\qty\big(b_i + \sum_j W_{ij}h_j).
\end{equation}

接下来计算式~\eqref{eq:rmb-gradient-approx} 中自由能的偏导数。代入 Boltzmann 机的能量函数
\eqref{eq:rbm-energy}~式,可以把自由能写成
\begin{align}
  F(\bm{v})
  &= - \ln \sum_{\q{h_i}}
       \exp\qty\Big(\sum_{i,\,j} W_{ij} v_i h_j + \sum_i b_i v_i + \sum_j c_j h_j) \notag \\
  &= - \sum_i b_i v_i
     - \ln \sum_{\q{h_i}} \exp\qty\bigg[\sum_j h_j \qty\Big(c_j + \sum_i W_{ij} v_i)] \notag \\
  &= - \sum_i b_i v_i
     - \ln \prod_j \sum_{h_j} \exp\qty\bigg[h_j \qty\Big(c_j + \sum_i W_{ij} v_i)] \notag \\
  &= - \sum_i b_i v_i
     - \sum_j \ln \sum_{h_j} \ee^{h_j \qty\big(c_j + \sum_i W_{ij} v_i)}.
\end{align}
利用 $h_j$ 的布尔特性,可以将上式化简:
\begin{align}
  F(\bm{v})
  &= - \sum_i b_i v_i - \sum_j \ln
       \qty\bigg[  \ee^{0 \cdot \qty\big(c_j + \sum_i W_{ij} v_i)}
                 + \ee^{1 \cdot \qty\big(c_j + \sum_i W_{ij} v_i)}] \notag \\
  &= - \sum_i b_i v_i - \sum_j \ln \qty\Big(1 + \ee^{c_j + \sum_i W_{ij} v_i}).
\end{align}
此时可以计算出自由能关于各参量的偏导数 \cite{rbmonline}:
\begin{align}
  \pdv{F(\bm{v})}{W_{ij}} &= - \pdv{W_{ij}} \ln \qty\Big(1+\ee^{c_j+\sum_iW_{ij}v_i})
                           = - v_i \cdot \frac{\ee^{c_j+W_{ij}v_i}}{1+\ee^{c_j+W_{ij}v_i}}
                           = - v_i \cdot \sigma\qty\Big(c_j+\sum_iW_{ij}v_i); \\
  \pdv{F(\bm{v})}{b_i}    &= - v_i; \\
  \pdv{F(\bm{v})}{c_j}    &= - \pdv{c_j} \ln \qty\Big(1+\ee^{c_j+\sum_iW_{ij}v_i})
                           = - \frac{\ee^{c_j+\sum_iW_{ij}v_i}}{1+\ee^{c_j+\sum_iW_{ij}v_i}}
                           = - \sigma\qty\Big(c_j+\sum_iW_{ij}v_i).
\end{align}
代入式~\eqref{eq:h-v-approx},并改写成矩阵形式:
\begin{equation}
  \pdv{F(\bm{v})}{\bm{W}} = - \bm{v} \bm{h}^\trans \qc
  \pdv{F(\bm{v})}{\bm{b}} = - \bm{v} \qc
  \pdv{F(\bm{v})}{\bm{c}} = - \bm{h}.
\end{equation}
此即式~\eqref{eq:rmb-gradient-approx} 中的正相部分。

通过 Gibbs 采样,负相部分的形式与正相部分完全一致,只要把 $\bm{v}$、$\bm{h}$ 换成 $\bm{v}'$、
$\bm{h}'$,因此有
\begin{equation}
  \pdv{F(\bm{v}')}{\bm{W}} = - \bm{v}' {\bm{h}'}^\trans \qc
  \pdv{F(\bm{v}')}{\bm{b}} = - \bm{v}' \qc
  \pdv{F(\bm{v}')}{\bm{c}} = - \bm{h}'.
\end{equation}
至此,我们便得到了梯度下降法所需要的参数更新公式 \cite{zhouzhihua,hinton2012practical}:
\begin{equation}
  \left\{
  \begin{aligned}
    \bm{W} \! &\gets \bm{W} \! - \alpha \qty\big(\bm{v} \bm{h}^\trans
                                                 - \bm{v}' {\bm{h}'}^\trans), \\
    \bm{b} \, &\gets \, \bm{b} - \alpha \qty\big(\bm{v} - \bm{v}'), \\
    \bm{c} \, &\gets \, \bm{c} - \alpha \qty\big(\bm{h} - \bm{h}').
  \end{aligned}
  \right.
\end{equation}
式中的 $\alpha$ 为学习率。

\section{RBM 的应用:手写数字}
\label{sec:rbm-mnist}

本节我们将利用 MNIST 数据集训练 RBM,并基于已有数据生成新的数据,即让计算机学会“书写”数字
\cite{rbmonline,lyy1994rbm}。

MNIST 数据集 \num{60000} 张灰度图片组成,每张图片表示一个手写数字,其大小被统一为 $28 \times 28$
像素,并保证数字处于居中位置 \cite{mnist}。RBM 的训练需要使用布尔型数据,我们也据此将图片进行了
二值化处理,如图~\ref{fig:mnist-image} 所示。

\begin{figure}[htb]
  \centering
  \begin{subfigure}[b]{0.35\textwidth}
    \centering
    \imageinput[width=2cm]{mnist-image.pdf}
    \caption{}
  \end{subfigure}
  \begin{subfigure}[b]{0.35\textwidth}
    \centering
    \imageinput[width=2cm]{mnist-image-binary.pdf}
    \caption{}
  \end{subfigure}
  \caption{MNIST 数据集中的首张图片,对应数字 5。
    (a)~原始图像,灰度值从白色到黑色分别对应 0--255;
    (b)~二值化后的图像,只包含值 0(白色)和 1(黑色)}
  \label{fig:mnist-image}
\end{figure}

具体的训练参数如下:

\begin{enumerate}
  \item 可见层单元数量:784($=28 \times 28$),即把图片压平为一维向量;
  \item 隐藏层单元数量:100;
  \item 批次(batch)大小:64,表示每次同时计算的样本数量;
  \item 训练轮次:20,一个轮次表示将整个数据集训练一遍;
  \item \CDk{} 中的参数 $k=30$;
  \item 学习率 $\alpha=0.1$。
\end{enumerate}

在训练过程中,我们需要监测损失函数的变化。从图~\ref{fig:learning-curve} 中可以看到,随着步数的
增加,损失函数迅速下降,并很快趋于平稳。这说明 RBM 的训练情况良好。

\begin{figure}[htb]
  \centering
  \imageinput{learning-curve.pdf}
  \caption{学习曲线。这里的步数(steps)等于批次数与训练轮次的乘积,注意取了对数坐标}
  \label{fig:learning-curve}
\end{figure}

\begin{figure}[htb]
  \centering
  \begin{subfigure}[b]{0.3\textwidth}
    \centering
    \imageinput[width=3cm]{mnist-weight-epoch-1.pdf}
    \caption{}
    \label{fig:mnist-weight-epoch-a}
  \end{subfigure}
  \begin{subfigure}[b]{0.3\textwidth}
    \centering
    \imageinput[width=3cm]{mnist-weight-epoch-6.pdf}
    \caption{}
    \label{fig:mnist-weight-epoch-b}
  \end{subfigure}
  \begin{subfigure}[b]{0.3\textwidth}
    \centering
    \imageinput[width=3cm]{mnist-weight-epoch-20.pdf}
    \caption{}
    \label{fig:mnist-weight-epoch-c}
  \end{subfigure}
  \caption{训练过程中的权重矩阵(部分)。权重矩阵 $\bm{W}$ 的形状为 $100 \times 784$,这里将其
    变形为 100 个 $28 \times 28$ 的子矩阵,以便更加清晰地呈现出从可见层到隐藏层的映照关系。
    (a)~第 1 个训练轮次;(b)~第 6 个训练轮次;(c)~第 20 个训练轮次}
  \label{fig:mnist-weight-epoch}
\end{figure}

与此同时,我们还需要监测权重矩阵的变化情况 \cite{lyy1994rbm}。初始时,权重矩阵 $\bm{W}$ 中的所有
元素均取为 0 到 1之间的随机数。经过一个轮次的训练后,$\bm{W}$ 中出现了一些明暗变化(图~%
\ref{fig:mnist-weight-epoch-a});到第 6 个轮次,可以明显地看到 $\bm{W}$ 中的展现出了笔画的特征,
但仍有较大噪音(图~\ref{fig:mnist-weight-epoch-b});到第 20 轮即训练结束时,各子权重矩阵均显著
出现了明暗对比,具有清晰的笔画线条,并且噪音较小(图~\ref{fig:mnist-weight-epoch-c}
和图~\ref{fig:mnist-weight})。注意到图~\ref{fig:mnist-weight} 中仍有一些子权重矩阵的训练效果不
理想,出现了较大的涨落。但可以预料,随着训练轮次的继续增加,涨落还会逐渐减小。

\begin{figure}[!htb]
  \centering
  \imageinput[width=10cm]{mnist-weight.pdf}
  \caption{最终生成的权重矩阵}
  \label{fig:mnist-weight}
\end{figure}

\begin{figure}[!htb]
  \centering
  \imageinput[width=12cm]{mnist-samples.pdf}
  \caption{由训练完成的 RBM 生成的数字图像。第一行为原始图像,之后每一行增加 \num{500} 次 Gibbs
    采样,到最后一行共进行了 \num{4000} 次采样}
  \label{fig:mnist-samples}
\end{figure}

最后,我们根据训练得到的数据生成新的手写数字。首先从 MNIST 数据集中随机选取一些样本图像作为可见层,
利用 Gibbs 采样生成隐藏层,再以同样方式返回到可见层,以此类推进行迭代,结果见图~%
\ref{fig:mnist-samples}。经过若干次采样之后,RBM 确实给出了不同于原始图像的分布,即学会了“书写”
数字。对于笔画比较简单的数字,如 1、7 等,学习效果较好;但对于稍复杂的数字,如 3、6 等,训练后的
RBM 给出了错误的结果;另外有些数字,如 5 和 9,RBM 只给出了模糊的阴影,并没有很好地复现原始图像。
这说明 RBM 训练算法仍有可以提高的余地。

\section{RBM 与重整化群的对应关系}

RBM 的思想与重整化群是十分类似的。在 RBM 中,我们把可见层中的信息编码到了隐藏层中;而在重整化群
方法中,我们把自旋进行粗粒近似,得到重整化后的 Hamilton 量。如果把 RBM 中的可见层理解为原始自旋
点阵,而把隐藏层理解为粗粒近似下的 Kadanoff 集团,我们便可以得到 RBM 与重整化群的对应关系。
在式~\eqref{eq:Z-H-renormalization} 中,我们给出了重整化前后 Hamilton 量的对应关系 \cite{exact}:
\begin{equation}
    \ee^{-H^{\text{RG}}(\bm{h})}
  = \sum_{\bm{v}} \ee^{\opT(\bm{v},\,\bm{h})-H(\bm{v})}.
\end{equation}
这里我们已经把自旋 $\sigma_i'$、$\sigma_i'$ 用 RBM 的语言 $\bm{v}$、$\bm{h}$ 进行表述,
并把 $\beta$ 归入了 $H$。接下来我们令
\begin{equation}
  \opT(\bm{v},\,\bm{h}) = H(\bm{v}) - E(\bm{v},\,\bm{h}).
\end{equation}
代入 \eqref{eq:rbm-probability-h}~式,可得
\begin{align}
  &\mathrel{\phantom{\implies}}
    \frac{1}{Z} \ee^{-H^{\text{RG}}(\bm{h})}
  = \frac{1}{Z} \sum_{\bm{v}} \ee^{-E(\bm{v},\,\bm{h})}
  = P(\bm{h})
  = \frac{1}{Z} \ee^{-H^{\text{RBM}}(\bm{h})} \notag \\
  &\implies H^{\text{RG}}(\bm{h}) = H^{\text{RBM}}(\bm{h}).
\end{align}
由此就建立了 RBM 与重整化群的严格映照。

利用式~\eqref{eq:rbm-probability-v}、\eqref{eq:conditional-probability} 进行计算,可以证明
\cite{exact}
\begin{equation}
  \ee^{\opT(\bm{v},\,\bm{h})} = P(\bm{h}|\bm{v}) \ee^{H(\bm{v})-H^{\text{RBM}}(\bm{v})}.
\end{equation}
当重整化变换 $\opT(\bm{v},\,\bm{h})$ 可以使得自旋集团严格重现原始自旋的自由能时,我们有
\begin{equation}
  \sum_{\bm{h}} \ee^{\opT(\bm{v},\,\bm{h})} = 1;
\end{equation}
另一方面,给定 $\bm{v}$ 下的条件概率 $P(\bm{h}|\bm{v})$ 满足
\begin{equation}
  \sum_{\bm{h}} P(\bm{h}|\bm{v}) = 1.
\end{equation}
这样可以得到
\begin{equation}
  H^{\text{RBM}}(\bm{v}) = H(\bm{v}),
\end{equation}
这说明经由 RBM 获得的概率分布严格等价于原始数据的分布。

\section{利用 RBM 研究 Ising 模型}

鉴于 RBM 与重整化群之间存在一一对应的关系,利用 Ising 模型点阵数据训练 RBM 之后,我们得到的结果
就相当于进行一次粗粒近似的重整化自旋集团。

为了与 MNIST 数据集保持一致,我们采用由 Metropolis 算法生成的 $28 \times 28$ 点阵。采样温度为
$\Tc/J+\num{0.03}$,外磁场为零。我们一共生成了 \num{60000} 组点阵数据,其中前 \num{59000} 组作为
训练集,其余作为测试集。RBM 的训练参数与 \ref{sec:rbm-mnist}~节中相同,除了训练轮次改为 50,
以得到更好的训练结果。待训练结束后,给出权重矩阵(图~\ref{fig:ising-weight})以及由 RBM 重建
而成的 Ising 模型点阵(图~\ref{fig:ising-samples})。

\begin{figure}[htb]
  \centering
  \imageinput[width=10cm]{ising-weight.pdf}
  \caption{训练结束时的权重矩阵}
  \label{fig:ising-weight}
\end{figure}

\begin{figure}[htb]
  \centering
  \imageinput[width=12cm]{ising-samples.pdf}
  \caption{由训练完成的 RBM 重建的 Ising 模型点阵。第一行为原始数据,之后每一行增加 \num{500}
    次 Gibbs 采样,到最后一行共进行了 \num{4000} 次采样}
  \label{fig:ising-samples}
\end{figure}

与 MNIST 数据集上的训练结果类似,权重矩阵给出了数据集的一些特征信息。一些子权重矩阵中出现了明暗相间
的条纹(如 6 行 8 列、6 行 10 列、8 行 1 列等),这种权重可以识别出不同相的边界;更多的子矩阵中
则出现了暗点(如 1 行 2 列、3 行 7 列、7 行 5 列等),它们反映了自旋的局域性信息,即在一个自旋集团
中,只有中心位置的原始自旋会对粗粒近似下的有效自旋产生贡献,而其余位置的原始自旋则贡献很小。由 RBM
所重建出的 Ising 模型点阵大致刻画了原始点阵的性质。对于明显具有标度变换不变性的结构(如第 6 列),
重建点阵能够保持住原有的“花斑”图样。但与 MNIST 类似,RBM 的学习效果并不十分理想。例如第 4 列,重建
点阵中存在明显的斑块,而这在原始数据中是不存在的。

\begin{figure}[htb]
  \centering
  \imageinput[width=12cm]{ising-hidden.pdf}
  \caption{RBM 所给出的隐藏层。这是在图~\ref{fig:ising-samples} (除第一行)的基础上,额外进行一次
    由可见层 $\bm{v}$ 到隐藏层 $\bm{h}$ 的 Gibbs 采样所得到的结果}
  \label{fig:ising-hidden}
\end{figure}

我们还给出了由 RBM 生成的隐藏层(图~\ref{fig:ising-hidden})。事实上,该图并不能反映出实质性的
物理信息。根据重整化群的思想,粗粒近似需要不断重复进行直至不动点的产生。因此我们可以推断,单层 RBM
所揭示的物理图像并不完善,而只有 RBM 组合成多层神经网络[称为深度信念网络(deep belief network,
DBN)]\cite{exact},才可以比较全面地展现出与重整化群的严格对应关系。

\section{卷积层}

\subsection{卷积}

在训练 RBM 的过程中,原始数据中的二维矩阵(MNIST 中的二维图像,或二维 Ising 模型的自旋构型)都被
压平成了一维向量。因此,原始数据中有关局域性(locality)的信息便会被全部丢失,没有被机器学习
利用起来。

为了克服这一问题,我们可以引入卷积层(convolution layer)。数学上,一维体系下的卷积可以定义如下
\cite{wiki:convolution}:
\begin{equation}
  \label{eq:convolution-continuous}
  (f \ast g) (t) = \int_{-\infty}^{+\infty} f(\tau) \, g(t-\tau) \dd{\tau}.
\end{equation}
对于离散情况,有
\begin{equation}
  (f \ast g) [n] = \sum_{k=-\infty}^{+\infty} f[k] \, g[k-n] \qc n,\,k \in \Z.
\end{equation}

卷积可以理解为一种“加权平均”。将卷积核 $f$ 与相同长度的“信号” $g$ 的片段逐项相乘,并依次放置在
相应位置,即完成了 $f$ 与 $g$ 的卷积。

将以上定义推广至高维,可得
\begin{equation}
  (f \ast g) [\bm{n}] = \sum_{\bm{k}} f[\bm{k}] \, g[\bm{k}-\bm{n}] \qc
  n=\mqty[n_1 \\ n_2 \\ \vdots \\ n_N], \, k=\mqty[k_1 \\ k_2 \\ \vdots \\ k_N] \in \Z^N.
\end{equation}
对于二维图像处理的问题,可以把卷积核 $f$ 理解为“滤镜”(filter),其过程如图~\ref{fig:convolution}
所示。事实上,如果取卷积核为 Gauss 函数
\begin{equation}
  f(\bm{x}) = \frac{1}{2\pp\sigma^2} \exp(-\frac{\norm{\bm{x}}^2}{2\sigma^2}),
\end{equation}
与待处理的图像进行卷积时,每次做“加权平均”,由于 Gauss 函数离中心较远处收敛迅速,所以中心像素
所占权重最大;而随着到中心的距离增加,各像素贡献也逐渐减小。整体来看,每个像素均用其周围一部分像素
的加权平均来代替,因而最终的结果就是对原始数据的 Gauss 模糊。

\begin{figure}[htb]
  \centering
  \imageinput{convolution.pdf}
  \caption{二维卷积示意图 \cite{liamconvolution}。随着卷积核(“滤镜”)在原始图像上移动,最终得到
    结果的就是以各像素为中心的局部加权平均}
  \label{fig:convolution}
\end{figure}

\subsection{小波变换}

类似于 Fourier 变换,小波变换(wavelet transform)也是一种函数变换。Fourier 变换将函数从时域转化为
频域(物理上等价于从坐标空间变换到动量空间),即用具有不同频率的基函数(三角函数)将原始函数展开。
与之类似,小波变换用一组母小波(mother wavelet)来展开原始函数。不同于三角函数,母小波具有收敛
迅速的特点,因此利用母小波的平移,在给出频域特征的同时还可以保留一定的时域特征 \cite{wiki:wavelet}。

在 $d$ 维空间中,利用母小波 $\psi$ 的缩放和平移可以张成一组基 \cite{wiki:wavelet,padic}:
\begin{equation}
  \label{eq:daughter-wavelet}
  \psi_{\bm{a},\,s}(\bm{x}) = \frac{1}{s^{d/2}} \psi\qty(\frac{\bm{x}-\bm{a}}{s}),
\end{equation}
其中的 $s$ 和 $\bm{a}$ 分别表征了 $\psi$ 的缩放和平移。这组基称为子小波(daughter wavelet)。
任意函数可以利用子小波展开:
\begin{equation}
  W_f(\bm{a},\,s) = \int \dd[d]{x} f(\bm{x}) \psi^\dagger_{\bm{a},\,s}(\bm{x}).
\end{equation}
这里 $\psi^\dagger$ 是 $\psi$ 的对偶小波。其逆变换为
\footnote{由于数学上的某些限制,逆变换未必存在。}
\begin{equation}
  f(\bm{x}) = \frac{1}{C_\psi} \int_0^\infty \frac{\dd{s}}{s^{d+1}}
              \int \dd[d]{a} W_f(\bm{a},\,s) \psi_{\bm{a},\,s}(\bm{x}).
\end{equation}
式中的 $C_\psi$ 为一系数。

与 \eqref{eq:convolution-continuous}~式进行对照,可以看到由于平移的存在,小波变换也可以理解为
一种卷积,它同样包括了卷积核(小波基)在原始函数上平移之后求和(积分)的操作。

\subsection{\AdSCFT{} 与小波变换}

在 \AdSCFT{} 中,区块(bulk)中可规范化的局域场 $\phi^i(\bm{x},z)$ 可以利用边界算符
(boundary operator)进行重建 \cite{padic}:
\begin{align}
  &\mathrel{\phantom{=}} \phi^i(\bm{x},z) \notag \\
  &= \int \dd[d]{y} K_i(\bm{x},z|\bm{y})  \opO^i(\bm{y}) \notag \\
  &+ \sum_{j,k} \frac{\lambda^i_{jk}}{N}
       \int \dd[d]{x'} \dd{z'} G_i(\bm{x},z|\bm{x}'\!,z')
       \int \dd[d]{y_1} K_j(\bm{x}'\!,z'|\bm{y}_1)  \opO^j(\bm{y}_1)
       \int \dd[d]{y_2} K_k(\bm{x}'\!,z'|\bm{y}_2)  \opO^k(\bm{y}_2) \notag \\
  &+ \bigO\qty(\frac{\lambda^2}{N^2}) \int \cdots
\end{align}
其中的 $\opO^i$ 是与区块中场 $\phi^i$ 对偶的边界算符:
\begin{equation}
  \lim_{z \to 0} \phi^i(\bm{x},z) = z^{-\Delta_i} \opO^i(\bm{x}),
\end{equation}
$\Delta_i$ 是 $\opO^i$ 的共形维数(conformal dimension);$K_i(\bm{x},z|\bm{y})$ 是 $\opO^i$
的边界—区块核(boundary--bulk kernel),称为模糊函数(smearing function),其定义为
\begin{equation}
  \label{eq:boundary-bulk-kernel}
  K_i(\bm{x},z|\bm{y})
  = \qty[\frac{z}{z^2-\norm{\bm{x}-\bm{y}}^2}]^{d-\Delta_i} \,
    \Theta\qty\big(z^2-\norm{\bm{x}-\bm{y}}^2);
\end{equation}
其中的 $\Theta$ 表示阶跃函数;$G_i(\bm{x},z|\bm{x}'\!,z')$ 是区块—区块核。
边界—区块核与区块—区块核有如下关系:
\begin{equation}
  \lim_{z' \to 0} {z'}^{\Delta_i-d} G_i(\bm{x},z|\bm{x}'\!,z') \sim K_i(\bm{x},z|\bm{x}').
\end{equation}
以上 $G^i$、$K_i$ 均可理解为传播子(propagator),或 Green 函数。

注意到式~\eqref{eq:boundary-bulk-kernel} 中 $K_i$ 的形式与小波变换非常类似 \cite{padic}。
我们取母小波为
\begin{equation}
  \psi_\Delta(\bm{x}) = \qty(\frac{1}{1-\norm{\bm{x}}^2})^{d-\Delta} \,
                        \Theta\qty\big(1-\norm{\bm{x}}^2)
\end{equation}
根据式~\eqref{eq:daughter-wavelet} 生成子小波,有
\begin{equation}
  \psi_{\bm{a},\,s}(\bm{x})
  = \frac{1}{z^{d/2}} \psi\qty(\frac{\bm{y}-\bm{x}}{z})
  = \frac{1}{z^{d/2}} \qty(\frac{1}{1-\norm{\frac{\bm{y}-\bm{x}}{z}}^2})^{d-\Delta}
    \Theta\qty(1-\norm{\frac{\bm{y}-\bm{x}}{z}}^2).
\end{equation}
这样即可得到
\begin{equation}
  K(\bm{x},z|\bm{y}) = z^{\Delta-d/2} \psi_{\bm{x},\,z}(\bm{y}).
\end{equation}

场 $\phi^i(\bm{x},z)$ 同样可由边界算符 $\opO_\Delta$ 的小波变换表示:
\begin{equation}
  \phi^i(\bm{x},z) = z^{\Delta-d/2} W_{\opO}(\bm{x},z).
\end{equation}
此处,Poincaré 坐标中的 $z$ 由缩放参数决定。

\section{卷积限制 Boltzmann 机(CRBM)}

利用 RBM 处理 Ising 模型,只能给出与重整化群相对应的定性关系,而不能给出具有较为深刻物理背景的
定量结果。考虑到 Ising 模型作为二维共形场论中最简单的最小模型(minimal model),我们可以从
\AdSCFT{} 的角度出发做一点讨论。

前面已经指出,\AdSCFT{} 提供了传播子 $K_i$ 与小波变换的对应关系,而小波变换则有可以被理解为某种
形式的卷积。因此我们认为,利用卷积网络处理 Ising 模型,则机器所学习到的卷积核就可以给出传播子
$K_i$ 的定量结果。

\subsection{网络构建}

在图~\ref{fig:rbm} 中,我们给出了 RBM 的网络构建。与图~\ref{fig:neural-net} 比较,可以看出二者
十分相似。事实上,RBM 正是基于线性网络的能量模型,它的核心是在线性映照
$\bm{x}\to\bm{W}\bm{x}+\bm{b}$ 基础上给出的能量函数 \eqref{eq:rbm-energy}~式。

\begin{figure}[htb]
  \centering
  \imageinput{crbm.pdf}
  \caption{卷积限制 Boltzmann 机示意图。绿色菱形表示卷积核,它们作用在可见层(红色圆圈)上,
    并生成隐藏层(蓝色方块)。与图~\ref{fig:rbm} 不同,这里单元之间的连线并不表示权重。不同深浅的
    连线展现了卷积核与可见层不同位置的对应。需要注意,可见层、隐藏层与卷积核实际上均是二维矩阵,
    它们相比该示意图额外还有一个垂直纸面的维度}
  \label{fig:crbm}
\end{figure}

下面我们需要把线性网络改变成卷积网络,称为卷积 RBM(CRBM),如图~\ref{fig:crbm} 所示
\cite{lee2009convolutional,lee2011unsupervised}。线性网络中,可见层与隐藏层之间有如下关系:
\begin{equation}
  \bm{h}+\bm{c} \sim \bm{W} (\bm{v}+\bm{b}).
\end{equation}
写出分量形式,为
\begin{equation}
  h_j+c_j \sim \sum_i W_{ij} (v_i+b_i).
\end{equation}
这里用“$\sim$”表示其间还可以增加激活函数,如 $\sigma$ 函数等。在卷积网络中,可见层与隐藏层
则通过下式相连接:
\begin{equation}
  h_{\bm{j}} + c_{\bm{j}} \sim \sum_{\bm{i}} W_{\bm{i}} (v_{\bm{i}+\bm{j}} + b_{\bm{i}+\bm{j}}).
\end{equation}
注意这里的指标 $\bm{i}$、$\bm{j}$ 均有两个维度,而不再是一个整数,且求和需要对两个维度分别进行。
类比 \eqref{eq:rbm-energy}~式,我们可以写出 CRBM 的能量函数
\cite{lee2009convolutional,lee2011unsupervised}:
\begin{equation}
  E(\bm{v},\,\bm{h})
  = - \sum_{\bm{i},\,\bm{j}} W_{\bm{i}} v_{\bm{i}+\bm{j}} h_{\bm{j}}
    - \sum_{\bm{i}} b_{\bm{i}} v_{\bm{i}} - \sum_{\bm{j}} c_{\bm{j}} h_{\bm{j}}.
\end{equation}
除了第一项,CRBM 与 RBM 的能量函数在形式上几乎完全一致。但由于参数空间增加了一个维度,这种改变
实际上是非平庸的。

\subsection{训练方法}

CRBM 的损失函数仍采用 \eqref{eq:ebm-loss-function}~式的形式,因此与 RBM 相同,也可以使用 \CDk{}
算法训练。仿照 \ref{sec:CDk-algorithm}~节的推导过程,我们可以计算得到 CRBM 的参数更新公式
\cite{lee2009convolutional,lee2011unsupervised}:
\begin{equation}
  \label{eq:crbm-parameters-update}
  \left\{
  \begin{aligned}
    W_{\bm{i}}    &\gets W_{\bm{i}} - \alpha
                         \qty\Big(  \sum_{\bm{j}} v_{\bm{i}+\bm{j}} h_{\bm{j}}
                                  - \sum_{\bm{j}} v'_{\bm{i}+\bm{j}} h'_{\bm{j}}), \\
    b_{\bm{i}} \! &\gets \! b_{\bm{i}} - \alpha \qty\big(v_{\bm{i}} - v'_{\bm{i}}),   \\
    c_{\bm{j}} \! &\gets \! c_{\bm{j}} - \alpha \qty\big(h_{\bm{j}} - h'_{\bm{j}}).
  \end{aligned}
  \right.
\end{equation}
式中的 $\alpha$ 为学习率。利用梯度下降法,即可据此进行训练。

由于参数空间提高到了二维,训练参数的数目大大增加。根据 \ref{subsec:regularization}~小节中的讨论,
为了缓解过拟合问题,我们有必要在损失函数中引入正则项。一种有效的手段是强制参数稀疏化,它只需要修改
式~\eqref{eq:crbm-parameters-update} 中参数 $\bm{b}$ 的更新规则
\cite{lee2009convolutional,lee2011unsupervised}:
\begin{equation}
  \incr{b_{\bm{i}}} = \incr{b_{\bm{i}}}^0 + \incr{b}^{\text{sparsity}},
\end{equation}
其中 $\incr{b_{\bm{i}}}^0$ 仍按式~\eqref{eq:crbm-parameters-update} 选取,而稀疏化项则为
\begin{equation}
  \incr{b}^{\text{sparsity}}
  \propto p - \frac{1}{N_{\text{hidden}}} \sum_{\bm{j}} P\qty\big(h_{\bm{j}}=1|\bm{v}).
\end{equation}

\chapter{结论}

本文在经典 Ising 模型的基础上,引入了现代机器学习手段对其进行了一些研究,主要包含两个方面:

一、
利用监督学习对 Ising 模型的有序相和无序相进行分类,这里采用的主要是线性分类模型,以及由多个线性层
组合而成的神经网络。我们发现,最简单的线性分类模型就一个可以给出比较准确的定性结果;而利用神经网络,
还可以获得一些定量结果,如计算得到临界温度 $\Tc$。

二、
探讨 RBM 等基于能量的模型与重整化群的联系。RBM 是生成型神经网络的代表,以 MNIST 手写数字数据集为例,
我们展示了 RBM 的学习能力。类似地,将其用来学习 Ising 模型的点阵数据,给出的训练结果也与粗粒近似
的思想相吻合。鉴于 RBM 无法给出定量结果,也就难以开展进一步的研究,我们还引入了卷积层,试图从理论
上建立起与 \AdSCFT{} 之间的对应关系。


\backmatter
\ctexset{chapter/numbering=false}
\chapter{附注}

本文所使用代码可在 \url{https://github.com/Stone-Zeng/ising} 中获得。
该代码按照 MIT 开源许可证公开发布。

\nocite{*}
\printbibliography
\input{chapters/acknowledgements}

\end{document}
