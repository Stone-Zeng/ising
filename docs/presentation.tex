\PassOptionsToPackage{log-declarations=false}{xparse}
\PassOptionsToPackage{no-math}{fontspec}
%\PassOptionsToPackage{draft}{graphicx}
\documentclass[aspectratio=169]{beamer}
\usepackage{fontspec,amsmath,unicode-math,physics,siunitx,subcaption}

\useoutertheme{metropolis}
\useinnertheme{metropolis}
\usecolortheme{metropolis}
\usefonttheme{professionalfonts}

\setbeamerfont{title}{size=\LARGE, series=\bfseries}
\setbeamerfont{frametitle}{series=\bfseries}
\setbeamerfont{caption}{size=\footnotesize}
\setbeamerfont{caption name}{series=\bfseries}
\setbeamerfont{footnote}{size=\tiny}
\setbeamertemplate{title page}{
  \begin{minipage}[b]{\textwidth}
    \mbox{}\vfill
    \usebeamertemplate*{title}
    \usebeamertemplate*{title separator}
    \usebeamertemplate*{author}
    \usebeamertemplate*{date}
    \usebeamertemplate*{institute}
    \mbox{}\hfill\inserttitlegraphic
    \vspace{-1.5cm}
  \end{minipage}
}
\setbeamertemplate{section in toc}[sections numbered]
\makeatletter
\AtBeginSection{
  \ifbeamer@inframe
    \sectionpage
  \else
    \frame[plain]{\sectionpage}
  \fi
}
\makeatother

\setsansfont{FiraGO}[BoldFont=* SemiBold]
\setmonofont{Iosevka}
\setmathfont{fira-math.otf}[Path=fonts/, BoldFont=*, math-style=ISO, bold-style=ISO, mathrm=sym]

\captionsetup[sub]{font=footnotesize}

% Constants
\def\ee{{\symup{e}}}
\def\Tc{T_{\!\symup{c}}}
% Operators
\def\incr{{\symup{\Delta}}}
\def\trans{{\symup{T}}}
\def\opT{{\symbfup{T}}}
\def\opO{O}
\newcommand\trR[2][]{\symbfup{R}^{\,#1}_{#2}}
% Set
\def\domD{D}
\def\domN{{N_{\symup{s}}}}
% Decorations
\def\bm#1{{\symbf{#1}}}
\def\nearest#1{\langle#1\rangle}
\def\q#1{{\{#1\}}}
% Text
\def\const{\text{const}}

\newcommand\imageinput[2][]{\includegraphics[#1]{figures/#2}}

\def\CDk{CD-$k$}
\def\AdSCFT{AdS/CFT}

\title{Machine Learning and Ising Model}
\date{\today}
\author{Xiangdong Zeng}
\institute{Department of Physics, Fudan University}
\titlegraphic{\includegraphics[height=1.2cm]{figures/fudan-emblem.pdf}}
\date{June 4, 2018}

\begin{document}

\maketitle

\begin{frame}{Table of contents}
\begin{itemize}
  \item \textbf{Ising model}
    \begin{itemize}
      \item Statistical physics, critical behaviors, etc
      \item Monte Carlo simulation
    \end{itemize}
  \item \textbf{Machine learning}
    \begin{itemize}
      \item Basic ideas
      \item Linear model
      \item Neural network
    \end{itemize}
  \item \textbf{Machine learning and renormalization group (RG)}
    \begin{itemize}
      \item Boltzmann machine and restricted Boltzmann machine (RBM)
      \item Contrastive divergence (\CDk{}) algorithm
      \item RBM for MNIST and Ising model
      \item Convolution, wavelet transform, etc
    \end{itemize}
\end{itemize}
\end{frame}

\section{Part 1. Ising model}

\begin{frame}{Ising model}
\begin{itemize}
  \item ``Binary'' spins ($+1$ or $-1$) arranged in a lattice
  \item Simplest model for (anti)-ferromagnetism and phase transition
  \item Hamiltonian:
    \[
      H(\q{\sigma_i}) = -\sum_{\nearest{ij}} J_{ij}\sigma_i\sigma_j - \mu \sum_{i} B_i\sigma_i
                      = -J \sum_{\nearest{ij}} \sigma_i\sigma_j - \mu B \sum_{i} \sigma_i
    \]
  \item Probability of configuration $\q{\sigma_i}$:
    \[
      P(\q{\sigma_i}) = \frac{\ee^{-\beta H(\q{\sigma_i})}}{Z_N} \qc
      Z_N = \sum_\q{\sigma_i} \ee^{-\beta H(\q{\sigma_i})}
    \]
\end{itemize}
\end{frame}

\begin{frame}{Exact solutions}
\begin{itemize}
  \item One dimension: no phase transition
  \item Two dimension
    \begin{itemize}
      \item Critical temperature:
        \[ \frac{\Tc}{J} = \frac{2}{\ln(1+\sqrt{2})} \approx \num{2.2692} \]
      \item Heat capacity (logarithmic divergence):
        \[ \frac{C}{N} \simeq \num{-0.4945} \ln\abs{1-\frac{T}{\Tc}} + \const \]
      \item Spontaneous magnetization:
        \[
          \frac{\bar{M}}{N\mu} =
          \begin{cases}
            \qty(1 - \sinh^{-4} 2\beta J)^{1/8}, & T<\Tc \\
            0, & T>\Tc
          \end{cases}
        \]
    \end{itemize}
\end{itemize}
\end{frame}

\begin{frame}{Critical behaviors and RG}
\begin{itemize}
  \item Universality: $\xi\to\infty$ at critical point
  \item Partition function is scale-invariant
    \begin{itemize}
      \item Partition function:
        \[ Z_N = \sum_\q{\sigma_i} \ee^{-\beta H(\q{\sigma_i},\,\bm{K})} \]
      \item Hamiltonian:
        \[
            H(\q{\sigma_i},\,\bm{K})
          = - \sum_i K^{(1)}_i \sigma_i
            - \sum_{\nearest{ij}}  K^{(2)}_{ij}  \sigma_i\sigma_j
            - \sum_{\nearest{ijk}} K^{(3)}_{ijk} \sigma_i\sigma_j\sigma_k - \cdots
        \]
      \item Scale transformation:
        \[ N' = L^{-d} N \qc \xi' = L^{-1} \xi \]
      \item Summation over Kadanoff's block $\q{\sigma_i'}$:
        \[
          Z_{N'}
          = \sum_\q{\sigma'_i} \ee^{-\beta H^{\text{RG}}(\q{\sigma_i'},\,\bm{K}')}
          = \sum_\q{\sigma'_i} \sum_\q{\sigma_i}
            \ee^{-\beta \qty\big[H(\q{\sigma_i},\,\bm{K})
                 -\opT_\lambda(\q{\sigma_i},\,\q{\sigma_i'})]} \qc
          \bm{K}' = \trR{L}(\bm{K})
        \]
    \end{itemize}
\end{itemize}
\end{frame}

\begin{frame}{Monte Carlo simulation}
\small
\begin{itemize}
  \item Basic idea: average over samples instead of all configurations
  \item Distribution evolution:
    \begin{align*}
      P\qty\big(\q{\sigma_i},\,t+1) = P\qty\big(\q{\sigma_i},\,t)
      &+ \sum_{\q{\sigma'_i}} P\qty\big(\q{\sigma'_i},\,t) W\qty\big(\q{\sigma'_i}\to\q{\sigma_i})
         \notag \\
      &- \sum_{\q{\sigma'_i}} P\qty\big(\q{\sigma_i},\,t)  W\qty\big(\q{\sigma_i}\to\q{\sigma'_i})
    \end{align*}
  \item Detailed balance condition:
    \[
        P_{\text{eq}}(\q{\sigma_i})  W\qty\big(\q{\sigma_i}\to\q{\sigma'_i})
      = P_{\text{eq}}(\q{\sigma'_i}) W\qty\big(\q{\sigma'_i}\to\q{\sigma_i})
    \]
  \item Metropolis transition rates:
    \[
      W\qty\big(\q{\sigma_i}\to\q{\sigma'_i}) =
      \begin{cases}
        1, & \incr{E} \leqslant 0 \\
        \ee^{-\beta \incr{E}}, & \incr{E} > 0
      \end{cases}
    \]
\end{itemize}
\end{frame}

\begin{frame}{Algorithm}
\begin{enumerate}
  \item Initialize all the spins: set spins to be 0 or 1 \emph{randomly}.
  \item Generate a trial state $\q{\sigma'_i}$ that is near the current state $\q{\sigma_i}$
    \begin{itemize}
      \item Usually we can flip a single spin.
    \end{itemize}
  \item Calculate $\incr{E}$ and accept probability $P$. If random number $\xi<P$, then flip the
    corresponding spin.
  \item Perform step 2 and 3 for all the lattice (one Monte Carlo step, MCS).
    \begin{itemize}
      \item To maintain detailed balance, we should choose the spin \emph{randomly}.
      \item Flipping one-by-one is also acceptable for efficiency.
    \end{itemize}
  \item Repeat step 2--4 for some time, then we can measure the thermodynamic observables and
    calculate their mean values and standard errors.
\end{enumerate}
\end{frame}

\begin{frame}{Simulation results (1)}
\begin{figure}
  \centering
  \begin{subfigure}[b]{0.32\textwidth}
    \centering
    \imageinput[width=3cm]{ising-lattice-1-0.pdf}
    \caption{}
  \end{subfigure}
  \begin{subfigure}[b]{0.32\textwidth}
    \centering
    \imageinput[width=3cm]{ising-lattice-2-3.pdf}
    \caption{}
  \end{subfigure}
  \begin{subfigure}[b]{0.32\textwidth}
    \centering
    \imageinput[width=3cm]{ising-lattice-3-6.pdf}
    \caption{}
  \end{subfigure}
  \caption{$64 \times 64$ Ising lattice at (a) $\Tc/J=1.0$; (b) $\Tc/J=2.3$; (c) $\Tc/J=3.6$.}
\end{figure}
\end{frame}

\begin{frame}{Simulation results (2)}
\begin{figure}
  \centering
  \begin{subfigure}[b]{0.32\textwidth}
    \centering
    \imageinput[height=3.4cm]{ising-energy.pdf}
  \end{subfigure}
  \begin{subfigure}[b]{0.32\textwidth}
    \centering
    \imageinput[height=3.4cm]{ising-magnet.pdf}
  \end{subfigure}
  \begin{subfigure}[b]{0.32\textwidth}
    \centering
    \imageinput[height=3.4cm]{ising-cv.pdf}
  \end{subfigure}
  \caption{(a) Energy; (b) spontaneous magnetization; (c) heat capacity.
    Sampling density is increased near $\Tc$. Error bar means one $\sigma$.}
\end{figure}
\end{frame}

\section{Part 2. Machine learning}

\begin{frame}{Basic ideas}
\begin{itemize}
  \item Extract features from data, build models, make predictions or decisions with the help of
    computers
  \item Give machine the ability to ``learn''
  \item Categories:
    \begin{itemize}
      \item Supervised learning: regression, classification, etc
      \item Unsupervised learning: clustering, dimensionality reduction, etc
    \end{itemize}
  \item Main steps:
    \begin{enumerate}
      \item Build a reasonable model based on data's feature;
      \item Give the \alert{loss function};
      \item Train on the dataset (training set), find the parameters to minimize the loss function;
      \item Validate the model and parameters, fine tuning (validation set);
      \item Test the trained model on \emph{new} dataset (test set).
    \end{enumerate}
\end{itemize}  
\end{frame}

\begin{frame}{Linear regression}
\begin{itemize}
  \item Hypothesis:
    \[ h_\bm{\theta} (\bm{x}) = \sum_{j=0}^n \theta_j x_j = \bm{\theta}^\trans \bm{x} \]
  \item Loss function:
    \[
      L(\bm{\theta})
      = \frac{1}{2} \sum_{i=1}^m \qty\Big[y^{(i)} - h_\bm{\theta}\qty\big(\bm{x}^{(i)})]^2
    \]
  \item Exact solution:
    \[ \hat{\bm{\theta}} = \qty\big(\bm{X}^\trans\bm{X})^{-1} \bm{X}^\trans \bm{y} \]
  \item Not realistic with large dataset --- \alert{gradient descent}
    \[ \bm{\theta} \coloneq \bm{\theta} - \alpha \, \nabla\bm{\theta} \]
\end{itemize}
\end{frame}

\begin{frame}{Linear classifier (logistic regression)}
\begin{columns}[t]
  \begin{column}{0.52\textwidth}
    \begin{itemize}
      \item Hypothesis:
        \[ h_\bm{\theta} (\bm{x}) = \sigma(\bm{\theta}^\trans \bm{x}) \]
      \item Activation function:
        \[ \sigma(z) = \frac{1}{1+\ee^{-z}} \]
      \item Loss function:
        \[
          L(\bm{\theta})
          = \sum_{i=1}^m \log\qty\Big{1+\exp[-y^{(i)} h_\bm{\theta}\qty\big(\bm{x}^{(i)})]}
        \]
    \end{itemize}
  \end{column}
  \begin{column}{0.44\textwidth}
    \begin{figure}
      \centering
      \imageinput[width=0.9\textwidth]{linear-layer.pdf}
      \caption{Network structure of the linear classifier}
    \end{figure}
  \end{column}
\end{columns}
\end{frame}

\begin{frame}{Neural network}
\begin{columns}[t]
  \begin{column}{0.44\textwidth}
    \begin{figure}
      \centering
      \imageinput[height=3.8cm]{multipolar-neuron.png}
      \caption{A multipolar neuron \footnotemark}
    \end{figure}
  \end{column}
  \begin{column}{0.54\textwidth}
    \begin{figure}
      \centering
      \imageinput[height=3.8cm]{neural-net.pdf}
      \caption{A two-layer neural network (perceptron)}
    \end{figure}
  \end{column}
\end{columns}
\footnotetext{Source: \url{https://en.wikipedia.org/wiki/File:Blausen_0657_MultipolarNeuron.png}}
\end{frame}

\begin{frame}{Neural network --- details}
\begin{columns}[t]
  \makebox[0.52\textwidth][l]{%
    \begin{column}{0.6\textwidth}
      \begin{itemize}
        \item Loss function (mean squared error):
          \[ L = \frac{1}{2} \sum_{i=1}^m \qty\big[y^{(i)} - \hat{y}^{(i)}]^2 \]
        \item Loss function (cross-entropy):
          \[
            L = - \frac{1}{m} \sum_{i=1}^m
                  \qty\Big[y^{(i)}\ln\hat{y}^{(i)}
                           - \qty\big(1-y^{(i)})\ln\qty\big(1-\hat{y}^{(i)})]^2
          \]
        \item Training algorithm: backpropagation
      \end{itemize}
    \end{column}}
  \begin{column}{0.48\textwidth}
    \begin{itemize}
      \item Regularization
        \begin{itemize}
          \item Prevent overfitting
          \item A kind of ``penalty'': force the parameters to be sparse
          \item Weight decay ($L^2$ reg\-u\-lar\-iza\-tion):
            \[ \frac{\lambda}{2m} \sum_k \norm{\bm{\theta}_k}^2 \]
        \end{itemize}
    \end{itemize}
  \end{column}
\end{columns}
\end{frame}

\begin{frame}{Training results (linear model)}
\begin{figure}
  \centering
  \begin{subfigure}[b]{0.45\textwidth}
    \centering
    \imageinput[width=4.5cm]{ising-learning-linear.pdf}
  \end{subfigure}
  \begin{subfigure}[b]{0.45\textwidth}
    \centering
    \imageinput[width=4.5cm]{ising-learning-linear-scaled.pdf}
  \end{subfigure}
  \caption{Predict results on Ising lattice. (a) Original; (b) after scale transformation.}
\end{figure}
\end{frame}

\begin{frame}{Training results (neural network)}
\begin{figure}
  \centering
  \begin{subfigure}[b]{0.45\textwidth}
    \centering
    \imageinput[width=4.5cm]{ising-learning-net.pdf}
  \end{subfigure}
  \begin{subfigure}[b]{0.45\textwidth}
    \centering
    \imageinput[width=4.5cm]{ising-learning-net-scaled.pdf}
  \end{subfigure}
  \caption{Predict results on Ising lattice. (a) Original; (b) after scale transformation.}
\end{figure}
\end{frame}

\section{Part 3. Machine learning and RG}

\begin{frame}{Energy-based model}
\begin{columns}[t]
  \makebox[0.42\textwidth][l]{%
    \begin{column}{0.48\textwidth}
      \begin{itemize}
        \item Boltzmann machine
          \begin{itemize}
            \item Energy function:
              \begin{align*}
                E(\bm{s}) &= E(\q{s_i}) \\
                          &= - \sum_{i<j} W_{ij} s_i s_j - \sum_i \theta_i s_i
              \end{align*}
            \item Structure:
              \begin{figure}
                \centering
                \imageinput[scale=0.6]{boltzmann-machine.pdf}
              \end{figure}
          \end{itemize}
      \end{itemize}
    \end{column}}
  \begin{column}{0.56\textwidth}
    \begin{itemize}
      \item Restricted Boltzmann machine (RBM)
        \begin{itemize}
          \item Energy function:
            \begin{align*}
              E(\bm{v},\,\bm{h})
              &= - \bm{v}^\trans \bm{W} \bm{h} - \bm{b}^\trans \bm{v} - \bm{c}^\trans \bm{h} \\
              &= - \sum_{i,\,j} W_{ij} v_i h_j - \sum_i b_i v_i - \sum_j c_j h_j
            \end{align*}
          \item Structure:
            \begin{figure}
              \centering
              \imageinput[scale=0.6]{rbm.pdf}
            \end{figure}
        \end{itemize}
    \end{itemize}
  \end{column}
\end{columns}
\end{frame}

\begin{frame}{CD-\textit{k} algorithm (1)}
\begin{itemize}
  \item Loss function:
    \[
      L(\bm{\theta})
      = - \frac{1}{\abs{\domD}} \sum_{\bm{x}^{(i)}\in\domD} \ln P\qty\big(\bm{x}^{(i)})
    \]
  \item Gradient:
    \[
        - \pdv{\bm{\theta}} \ln P\qty\big(\bm{x}^{(i)})
      = - \pdv{\bm{\theta}} \ln P\qty\big(\bm{v}^{(i)})
      = \pdv{F(\bm{v}^{(i)})}{\bm{\theta}}
      - \sum_{\bm{v}} P(\bm{v}) \pdv{F(\bm{v})}{\bm{\theta}}
    \]
    where $F$ is the free energy
    \[ F(\bm{v}) = - \ln \sum_{\bm{h}} \ee^{-E(\bm{v},\,\bm{h})} \]
  \item Positive phase and negative phase (difficult)
\end{itemize}
\end{frame}

\begin{frame}{CD-\textit{k} algorithm (2)}
\begin{columns}
  \begin{column}{0.49\textwidth}
    \begin{itemize}
      \item Monte Carlo again:
        \[
          \sum_{\bm{v}} P(\bm{v}) \pdv{F(\bm{v})}{\bm{\theta}}
          \approx \frac{1}{\abs{\domN}} \sum_{\bm{v}\in\domN} \pdv{F(\bm{v})}{\bm{\theta}}
        \]
      \item Gibbs sampling:
        %\[
        %  \bm{h}  \sim P(\bm{h}\big|\bm{v}) \qc
        %  \bm{v}' \sim P(\bm{v}\big|\bm{h})
        %\]
        \begin{figure}
          \centering
          \imageinput[scale=0.64]{gibbs-sampling.pdf}
        \end{figure}
    \end{itemize}
  \end{column}
  \begin{column}{0.49\textwidth}
    \begin{itemize}
      \item Gradient:
        \[
          - \pdv{\bm{\theta}} \ln P\qty\big(\bm{v}^{(i)})
          \approx \pdv{F(\bm{v}^{(i)})}{\bm{\theta}}
                - \pdv{F({\bm{v}'}^{(i)})}{\bm{\theta}}
        \]
      \item Update parameters:
        \[
          \left\{
          \begin{aligned}
            \bm{W} \! &\gets \bm{W} \! - \alpha \, (  \bm{v} \bm{h}^\trans
                                                    - \bm{v}' {\bm{h}'}^\trans) \\
            \bm{b}    &\gets \, \bm{b} - \alpha \, (\bm{v} - \bm{v}') \\
            \bm{c}    &\gets \, \bm{c} - \alpha \, (\bm{h} - \bm{h}')
          \end{aligned}
          \right.
        \]
    \end{itemize}
  \end{column}
\end{columns}
\end{frame}

\begin{frame}{RBM trained on MNIST}
\begin{columns}
  \begin{column}{0.44\textwidth}
    \begin{figure}
      \centering
      \imageinput[height=5.2cm]{mnist-weight.pdf}
      \caption{Weight matrix (reshaped)}
    \end{figure}
  \end{column}
  \begin{column}{0.55\textwidth}
    \begin{figure}
      \centering
      \imageinput[height=5.2cm]{mnist-samples.pdf}
      \caption{Reconstructed images}
    \end{figure}
  \end{column}
\end{columns}
\end{frame}

\begin{frame}{RBM trained on Ising lattice}
\begin{columns}
  \begin{column}{0.44\textwidth}
    \begin{figure}
      \centering
      \imageinput[height=5.2cm]{ising-weight-raw.pdf}
      \caption{Weight matrix (reshaped)}
    \end{figure}
  \end{column}
  \begin{column}{0.55\textwidth}
    \begin{figure}
      \centering
      \imageinput[height=5.2cm]{ising-samples.pdf}
      \caption{Reconstructed images}
    \end{figure}
  \end{column}
\end{columns}
\end{frame}

\begin{frame}{Exact mapping between RBM and RG}
\begin{itemize}
  \item Partition function under coarse graining:
    \[
      Z_{N'}
      = \sum_\q{\sigma'_i} \ee^{-\beta H^{\text{RG}}(\q{\sigma_i'},\,\bm{K}')}
      = \sum_\q{\sigma'_i} \sum_\q{\sigma_i}
        \ee^{-\beta \qty\big[H(\q{\sigma_i},\,\bm{K})
              -\opT_\lambda(\q{\sigma_i},\,\q{\sigma_i'})]}
    \]
  \item Express in RBM:
    \[
        \ee^{-H^{\text{RG}}(\bm{h})}
      = \sum_{\bm{v}} \ee^{\opT(\bm{v},\,\bm{h})-H(\bm{v})}
    \]
  \item Let $\opT(\bm{v},\,\bm{h}) = H(\bm{v}) - E(\bm{v},\,\bm{h})$, then
    \[
        \frac{1}{Z} \ee^{-H^{\text{RG}}(\bm{h})}
      = \frac{1}{Z} \sum_{\bm{v}} \ee^{-E(\bm{v},\,\bm{h})}
      = P(\bm{h})
      = \frac{1}{Z} \ee^{-H^{\text{RBM}}(\bm{h})}
      \implies \alert{H^{\text{RG}}(\bm{h}) = H^{\text{RBM}}(\bm{h})}
    \]
\end{itemize}
\end{frame}

\begin{frame}{Convolution and wavelet transform}
\begin{itemize}
  \item Convolution:
    \[ (f \ast g) (t) = \int_{-\infty}^{+\infty} f(\tau) \, g(t-\tau) \dd{\tau} \]
  \item A kind of ``weighted average''
  \item Expand function with \emph{daughter wavelet}:
    \[ \psi_{\bm{a},\,s}(\bm{x}) = \frac{1}{s^{d/2}} \psi\qty(\frac{\bm{x}-\bm{a}}{s}) \]
  \item Wavelet transform:
    \[ W_f(\bm{a},\,s) = \int \dd[d]{x} f(\bm{x}) \psi^\dagger_{\bm{a},\,s}(\bm{x}) \]
\end{itemize}
\end{frame}

\begin{frame}{\AdSCFT{} and wavelet transform}
\begin{itemize}
  \item Reconstruct fields in the bulk from boundary operators:
    \begingroup\footnotesize
      \begin{align*}
           \phi^i(\bm{x},z)
        &= \int \dd[d]{y} K_i(\bm{x},z|\bm{y}) \opO^i(\bm{y}) \\
        &+ \sum_{j,k} \frac{\lambda^i_{jk}}{N}
             \int \dd[d]{x'} \dd{z'} G_i(\bm{x},z|\bm{x}'\!,z')
             \int \dd[d]{y_1} K_j(\bm{x}'\!,z'|\bm{y}_1) \opO^j(\bm{y}_1)
             \int \dd[d]{y_2} K_k(\bm{x}'\!,z'|\bm{y}_2) \opO^k(\bm{y}_2) + \cdots
      \end{align*}
    \endgroup
  \item Boundary--bulk kernel:
    \begingroup\footnotesize
      \[
        K_i(\bm{x},z|\bm{y})
        = \qty[\frac{z}{z^2-\norm{\bm{x}-\bm{y}}^2}]^{d-\Delta_i} \,
          \Theta\qty\big(z^2-\norm{\bm{x}-\bm{y}}^2)
      \]
    \endgroup
  \item Mother and daughter wavelets:
    \begingroup\footnotesize
      \[
        \psi_\Delta(\bm{x}) = \qty(\frac{1}{1-\norm{\bm{x}}^2})^{d-\Delta} \,
                              \Theta\qty\big(1-\norm{\bm{x}}^2) \qc
        \psi_{\bm{a},\,s}(\bm{x})
        = \frac{1}{z^{d/2}} \psi\qty(\frac{\bm{y}-\bm{x}}{z})
      \]
    \endgroup
  \item Then
    \begingroup\footnotesize
      \[ K(\bm{x},z|\bm{y}) = z^{\Delta-d/2} \psi_{\bm{x},\,z}(\bm{y}) \]
    \endgroup
\end{itemize}
\end{frame}

\begin{frame}{Convolutional RBM (CRBM)}
\begin{columns}[t]
  \begin{column}{0.68\textwidth}
    \begin{itemize}
      \item Linear to convolutional:
        \[
          h_j+c_j \sim \sum_i W_{ij} (v_i+b_i) \to
          h_{\bm{j}} + c_{\bm{j}} \sim
            \sum_{\bm{i}} W_{\bm{i}} (v_{\bm{i}+\bm{j}} + b_{\bm{i}+\bm{j}})
        \]
      \item Energy function:
        \[
          E(\bm{v},\,\bm{h})
          = - \sum_{\bm{i},\,\bm{j}} W_{\bm{i}} v_{\bm{i}+\bm{j}} h_{\bm{j}}
            - \sum_{\bm{i}} b_{\bm{i}} v_{\bm{i}} - \sum_{\bm{j}} c_{\bm{j}} h_{\bm{j}}
        \]
      \item Our conjecture: filter $\bm{W}$ in CRBM is the Green's function $K$ in \AdSCFT{}
    \end{itemize}
  \end{column}
  \begin{column}{0.3\textwidth}
    \begin{figure}
      \centering
      \imageinput[width=\textwidth]{crbm.pdf}
    \end{figure}
  \end{column}
\end{columns}
\end{frame}

\section{Conclusion}

\begin{frame}{Conclusion}
\begin{itemize}
  \item Supervised learning can be used to classify phases of Ising model
  \item RBM can be considered as an implementation of renormalization
  \item Convolution and wavelet transform that root in \AdSCFT{} can be mapped to
    machine learning (e.g. CRBM)
\end{itemize}
\end{frame}

\begin{frame}[standout]
\LARGE\itshape
Thank you!
\end{frame}

\end{document}
